\documentclass{article}
\pdfinfoomitdate=1
\pdftrailerid{}

\usepackage{float}
\restylefloat{table}

\usepackage{booktabs}

\title{Team Contributions: POC\\\progname}

\author{\authname}

\date{}

\input{../Comments}
%% Common Parts

\newcommand{\progname}{RoCam} % PUT YOUR PROGRAM NAME HERE
\newcommand{\authname}{Team \#3, SpaceY
  \\ Zifan Si
  \\ Jianqing Liu
  \\ Mike Chen
  \\ Xiaotian Lou} % AUTHOR NAMES                  

\usepackage{hyperref}
\hypersetup{colorlinks=true, linkcolor=blue, citecolor=blue, filecolor=blue,
  urlcolor=blue, unicode=false}
\urlstyle{same}

\usepackage{indentfirst}
\usepackage{graphicx}

\usepackage{titling}

\pretitle{\begin{center}\includegraphics[width=0.5\textwidth]{../../assets/logo/black.png}\\[0.75em]\LARGE}
    \posttitle{\par\end{center}}

\usepackage[letterpaper, portrait, margin=1in]{geometry}

\usepackage{placeins}
\usepackage{float}

\begin{document}

\maketitle

This document summarizes the contributions of each team member up to the POC
Demo. The time period of interest is the time between the beginning of the term
and the POC demo.

\section{Demo Plans}

We will present an end-to-end proof-of-concept for \progname{} (“RoCam”),
spanning video acquisition, real-time target detection/tracking, closed-loop
gimbal control, and the operator UI. During the demo, a Jetson ingests a live
720p camera feed; an on-device preview window on the Jetson displays the stream
together with key telemetry (FPS, end-to-end latency, gimbal angles/commands)
and overlays (bounding box and confidence). The YOLO-series tracker outputs the
target location; with a PID controller we issue smoothed follow commands so the
target remains near the frame center. When the printed rocket drifts away from
center, the gimbal moves the camera and continues tracking. The demo includes
one manual takeover and one autonomous tracking segment to demonstrate the
implemented core functionality.

We will verify three quantitative metrics: (1) real-time performance, with
end-to-end latency of approximately 100--500\,ms; (2) stability, where for at
least 10\,s of continuous tracking the target center remains within $\pm 15\%$
of the frame center; and (3) closed-loop accuracy, with gimbal pointing error
not exceeding $3^\circ$--$5^\circ$. We will also record and display the
processing frame rate (targeting $\geq 15$\,FPS). After the demo we will submit
an evidence package including screen recordings/videos, UI screenshots with
overlays, and time-stamped logs (FPS, latency, angles, commands) for
reproducibility and verification.

The demo environment is a Jetson Orin with a custom PCB and a custom gimbal;
the target is a printed rocket image. We connect to the Jetson over SSH to
launch the programs. For degradation/fallback, when poor lighting or weak
texture reduces detection confidence, we lower the detection threshold (and, if
needed, relax smoothing) to maintain tracking; the operator can temporarily
take manual control and the system will resume closed-loop tracking when the
target is re-acquired.

\section{Team Meeting Attendance}

\wss{For each team member how many team meetings have they attended over the
  time period of interest.  This number should be determined from the meeting
  issues in the team's repo.  The first entry in the table should be the total
  number of team meetings held by the team.}

\begin{table}[H]
  \centering
  \begin{tabular}{ll}
    \toprule
    \textbf{Student} & \textbf{Meetings} \\
    \midrule
    Total            & 17                \\
    Xiaotian Lou     & 17                \\
    Zifan Si         & 17                \\
    Jianqing Liu     & 17                \\
    Shike Chen       & 17                \\
    \bottomrule
  \end{tabular}
\end{table}

\wss{If needed, an explanation for the counts can be provided here.}

\section{Supervisor/Stakeholder Meeting Attendance}

\wss{For each team member how many supervisor/stakeholder team meetings have
  they attended over the time period of interest.  This number should be determined
  from the supervisor meeting issues in the team's repo.  The first entry in the
  table should be the total number of supervisor and team meetings held by the
  team.  If there is no supervisor, there will usually be meetings with
  stakeholders (potential users) that can serve a similar purpose.}

\noindent \textbf{Supervisor's Name: } Sirouspour, Shahin

\begin{table}[H]
  \centering
  \begin{tabular}{ll}
    \toprule
    \textbf{Student} & \textbf{Meetings} \\
    \midrule
    Total            & 3                 \\
    Xiaotian Lou     & 3                 \\
    Zifan Si         & 3                 \\
    Jianqing Liu     & 3                 \\
    Shike Chen       & 3                 \\
    \bottomrule
  \end{tabular}
\end{table}

\wss{If needed, an explanation for the counts can be provided here.}

\section{Lecture Attendance}

\wss{For each team member how many lectures have they attended over the time
  period of interest.  This number should be determined from the lecture issues in
  the team's repo. You can find the number of lectures in the time period of
  interest by looking at the
  \href{https://calendar.google.com/calendar/u/0/embed?src=rnboqiaki1k2la7rpu3bn0um58@group.calendar.google.com&ctz=America/Toronto}
  {Google calendar} for the capstone course.}

\wss{NOTE: There will be approximately 13 lectures between the start of class
  and the POC demos}

\begin{table}[H]
  \centering
  \begin{tabular}{ll}
    \toprule
    \textbf{Student} & \textbf{Lectures} \\
    \midrule
    Total            & 14                \\
    Xiaotian Lou     & 8                 \\
    Zifan Si         & 6                 \\
    Jianqing Liu     & 5.5               \\
    Shike Chen       & 5                 \\
    \bottomrule
  \end{tabular}
\end{table}

\wss{If needed, an explanation for the lecture attendance can be provided here.}

\section{TA Document Discussion Attendance}

\wss{For each team member how many of the informal document discussion meetings
  with the TA were attended over the time period of interest.}

\noindent \textbf{TA's Name: } Tanya Djavaherpour

\begin{table}[H]
  \centering
  \begin{tabular}{ll}
    \toprule
    \textbf{Student} & \textbf{Lectures} \\
    \midrule
    Total            & 3                 \\
    Xiaotian Lou     & 3                 \\
    Zifan Si         & 3                 \\
    Jianqing Liu     & 3                 \\
    Shike Chen       & 3                 \\
    \bottomrule
  \end{tabular}
\end{table}

\wss{If needed, an explanation for the attendance can be provided here.}

\section{Commits}

\wss{For each team member how many commits to the main branch have been made
  over the time period of interest.  The total is the total number of commits for
  the entire team since the beginning of the term.  The percentage is the
  percentage of the total commits made by each team member.}

\begin{table}[H]
  \centering
  \begin{tabular}{lll}
    \toprule
    \textbf{Student} & \textbf{Commits} & \textbf{Percent} \\
    \midrule
    Total            & 280              & 100\%            \\
    Xiaotian Lou     & 67               & 23.9\%           \\
    Zifan Si         & 88               & 31.4\%           \\
    Jianqing Liu     & 100              & 35.7\%           \\
    Shike Chen       & 25               & 8.9\%            \\
    \bottomrule
  \end{tabular}
\end{table}

\noindent\textit{Notes.} Commit counts reflect merges to the \texttt{main} branch only. During the POC period, Shike Chen and Xiaotian Lou focused on model R\&D and large-scale training/ablation (e.g., DINO and YOLO variants). Much of this work involved exploratory code, training scripts, and logs that reside in experimental/backup branches and on the university H100 “Grace” cluster rather than the repository’s \texttt{main}; as a result, their commit totals under-represent effort. We trained YOLO weights for $>30$ hours and accumulated substantial debugging/tuning experience across software, hardware, and environment layers.

We therefore interpret the raw commit distribution with caution. When
accounting for model R\&D and training conducted outside \texttt{main}, the
overall contribution across team members is roughly balanced. Reproducible
training scripts and experiment summaries will be upstreamed in subsequent PRs.

\section{Issue Tracker}

\wss{For each team member how many issues have they authored (including open and
  closed issues (O+C)) and how many have they been assigned (only counting closed
  issues (C only)) over the time period of interest.}

\begin{table}[H]
  \centering
  \begin{tabular}{lll}
    \toprule
    \textbf{Student} & \textbf{Authored (O+C)} & \textbf{Assigned (C only)} \\
    \midrule
    Xiaotian Lou     & 17                      & 10                         \\
    Zifan Si         & 13                      & 16                         \\
    Jianqing Liu     & 46                      & 20                         \\
    Shike Chen       & 2                       & 6                          \\
    \bottomrule
  \end{tabular}
\end{table}

\noindent\textit{Notes.}
\begin{itemize}
  \item \textbf{Counting rule:} “Authored (O+C)” counts issues opened by a member (open or closed). “Assigned (C only)” counts issues where the member was the assignee at closure.
  \item \textbf{Role-based workflow:} During the POC period, Jianqing Liu (project manager, “Pega”) centrally opened, assigned, and closed most issues based on team decisions captured in meeting notes; hence his “Authored” count is higher.
  \item \textbf{Credit vs.\ logistics:} Centralized opening/closing is administrative. Actual implementation credit is reflected in commits, PRs, and code reviews; an issue can have multiple contributors beyond the final assignee.
  \item \textbf{Assignment semantics:} Ownership may change during an issue’s lifetime; we attribute “Assigned (C only)” to the final assignee at closure to represent delivery responsibility.
\end{itemize}

\section{CICD}

We use GitHub Actions with four workflows that cover documentation, firmware,
and the operator UI.

\paragraph{build-tex (push to \texttt{main}, \texttt{docs/**}, or manual)}
Compiles LaTeX with TeX~Live~2025 on Ubuntu and commits the built PDFs back to the repository.
The job runs \texttt{make -B} under \texttt{docs/} and then auto-commits any updated \texttt{.pdf} files
(\texttt{contents: write}).

\paragraph{check-tex (pull requests touching \texttt{docs/**})}
Validates that the LaTeX compiles on PRs, uploads any changed PDFs as an artifact, then formats all
\texttt{.tex} sources with \texttt{latexindent} (80-column wrapping) and commits the formatting changes.
This keeps the documentation reproducible and consistently formatted.

\paragraph{firmware-ci (push/PR under \texttt{src/pcb-firmware/**})}
Builds the embedded firmware with Cargo on Ubuntu and uploads the resulting binary as an artifact.
The job also enforces Rust source formatting via \texttt{cargo fmt --check}.

\paragraph{react-ci (push/PR for the React app)}
Builds and smoke-tests the operator UI on Node~20. Dependencies are installed
with \texttt{npm ci}; a lightweight \texttt{ci:smoke} runs on every change.
Unit tests and production build currently run in \emph{best-effort} mode
(non-blocking), and any build artifacts are uploaded when present.

\paragraph{Near-term roadmap (to be enforced via branch protection)}
\begin{itemize}
  \item \textbf{Formatting gates:} add \texttt{black --check} for Python sources and \texttt{prettier --check} for the React codebase. If formatting is not clean, the PR will fail and be blocked from merging.
  \item \textbf{Cross-platform test matrix:} add a Python test job that runs the full unit test suite (15 tests) on
        \{\texttt{ubuntu-latest}, \texttt{macos-latest}, \texttt{windows-latest}\} $\times$
        Python \{3.9, 3.10, 3.11, 3.12, 3.13\}. The job will install dependencies and execute the tests; failures will block merges.
\end{itemize}

\section{Team Charter Trigger Items}

\subsection*{Quantified Triggers (per Team Charter)}
\begin{itemize}
  \item \textbf{Meeting cadence \& attendance:} weekly team meetings; prior notice required for any absence; target attendance $\geq$85\%.
  \item \textbf{Preparation \& quality:} come to meetings with concise updates, blockers, and options; maintain decision logs and lightweight RACI notes.
  \item \textbf{PR/issue hygiene:} every task has a GitHub issue with an owner and estimate; all code changes go through PRs that reference an issue; no direct pushes to \texttt{main}.
  \item \textbf{Review SLA \& CI gates:} at least one reviewer approval; address review comments within 7 days (unless a different SLA is agreed); CI must be green (docs build, firmware build, UI smoke).
  \item \textbf{Documentation standards:} LaTeX compiles reproducibly on CI; PDFs are published on \texttt{main}; \texttt{latexindent} enforces consistent formatting on PRs.
  \item \textbf{Runtime KPIs (demo readiness):} track end-to-end latency (p50/p95), FPS, target re-acquisition time, crash-free session rate, and UI task success; for the POC demo we target $\sim$100--500\,ms latency, $\geq$15\,FPS, and $\leq$2\,s re-acquisition.
  \item \textbf{Underperformance trigger:} if a member underdelivers for $>$3 weeks, initiate pairing/guardrails and, if needed, escalate to TA/instructor.
\end{itemize}

\subsection*{Violations Observed During the POC Period}
None observed. All members met the charter triggers: weekly meetings were held
with advance notice for any conflicts; PRs referenced issues and passed CI;
review comments were addressed within the 7-day SLA; documentation built
reproducibly; runtime KPIs were recorded for the demo.

\subsection*{Plan to Sustain Compliance}
\begin{itemize}
  \item Keep branch protection (no direct pushes to \texttt{main}; $\geq$1 review; CI
        required).
  \item Promote formatting to required checks: \texttt{black --check} (Python) and
        \texttt{prettier --check} (React).
  \item Add a cross-platform Python test matrix (Ubuntu/macOS/Windows;
        Python~3.9--3.13) as a required status check.
  \item Use a meeting-issue template (owner, ETA, blockers) and maintain the decision
        log to preserve cadence and traceability.
  \item Clarify KPI thresholds for upcoming milestones and monitor p50/p95 latency,
        FPS, and re-acquisition time on each demo run.
\end{itemize}

\section{Additional Productivity Metrics}

\wss{If your team has additional metrics of productivity, please feel free to
  add them to this report.}
our team has no additional productivity metrics to report at this time.
\end{document}
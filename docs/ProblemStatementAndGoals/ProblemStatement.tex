\documentclass{article}

\usepackage{tabularx}
\usepackage{booktabs}

\title{Problem Statement and Goals\\\progname}

\author{\authname}

\date{}

\input{../Comments}
%% Common Parts

\newcommand{\progname}{RoCam} % PUT YOUR PROGRAM NAME HERE
\newcommand{\authname}{Team \#3, SpaceY
  \\ Zifan Si
  \\ Jianqing Liu
  \\ Mike Chen
  \\ Xiaotian Lou} % AUTHOR NAMES                  

\usepackage{hyperref}
\hypersetup{colorlinks=true, linkcolor=blue, citecolor=blue, filecolor=blue,
  urlcolor=blue, unicode=false}
\urlstyle{same}

\usepackage{indentfirst}
\usepackage{graphicx}

\usepackage{titling}

\pretitle{\begin{center}\includegraphics[width=0.5\textwidth]{../../assets/logo/black.png}\\[0.75em]\LARGE}
    \posttitle{\par\end{center}}

\usepackage[letterpaper, portrait, margin=1in]{geometry}

\usepackage{placeins}
\usepackage{float}

\begin{document}

\maketitle

\begin{table}[hp]
\caption{Revision History} \label{TblRevisionHistory}
\begin{tabularx}{\textwidth}{llX}
\toprule
\textbf{Date} & \textbf{Developer(s)} & \textbf{Change}\\
\midrule
Sept. 8, 2025 & Jianqing Liu & Initial Draft\\
... & ... & ...\\
\bottomrule
\end{tabularx}
\end{table}

\section{Problem Statement}

\wss{You should check your problem statement with the
\href{https://github.com/smiths/capTemplate/blob/main/docs/Checklists/ProbState-Checklist.pdf}
{problem statement checklist}.} 

\wss{You can change the section headings, as long as you include the required
information.}

\subsection{Problem}

In model rocketry, staging fails and parachute tangling has been the hardest engineering problems 
to solve due to difficulties to observe what actually happens when the rocket is in the air.

Model rockets fly extremely fast and high, some exceeding mach 3 and reach an altitude over 100km (TODO: citation here).
This makes manual camera tracking basically impossible.

Tracing a small model rocket can be more challenging than big rockets like the Falcon 9, because the
launch pad environment is more unpredictable and the target is smaller, which makes it harder to identify 
and track.

Some commercial tracking camera solutions exist (TODO: citation here), but they do not have the performance and 
accuracy required to track a small and fast accelerating model rocket.

This project aims to develop a production-ready software stack for the tracking camera with a modular 
interface for the gimbal mechanism.

\subsection{Inputs and Outputs}

\wss{Characterize the problem in terms of ``high level'' inputs and outputs.  
Use abstraction so that you can avoid details.}

\begin{itemize}
    \item Inputs
    \begin{itemize}
        \item 1080p 60fps camera feed
    \end{itemize}
    \item Outputs
    \begin{itemize}
        \item Gimbal movement commands
        \item Real-time 1080p 60fps digitally zoomed and stablized video feed
        \item Web management portal
    \end{itemize}
\end{itemize}

\subsection{Stakeholders}

\begin{itemize}
    \item Model Rocket Engineers
    \item TODO: more?
\end{itemize}

\subsection{Environment}

\wss{Hardware and software environment}

\subsection{Gimbal}

An off-the-shelf cheap gimbal is used.

A custom developed PCB is used to adapt the gimbal to the Jetson.

\subsection{Computer Vision}

Nvidia Jetson Orin Nano Super

\subsection{Web Management Portal}

Runs on any recent web browser

\section{Goals}

Track small-scale rocket launches (apogee < 200m)

\section{Stretch Goals}

connected to a full-size gimbal developed by the McMaster Rocketry Team to track high-powered rocket launches (apogee 3km+)

\section{Extras}

\wss{Teams may wish to include extras as either potential bonus grades, or to
make up for a less advanced challenge level.  Potential extras include usability
testing, code walkthroughs, user documentation, formal proof, GenderMag
personas, Design Thinking, etc.  Normally the maximum number of extras will be
two.  Approval of the extras will be part of the discussion with the instructor
for approving the project.  The extras, with the approval (or request) of the
instructor, can be modified over the course of the term.}

\subsection{Circuit Design}

\subsection{TODO: Extra 2}


\newpage{}

\section*{Appendix --- Reflection}

\wss{Not required for CAS 741}

\input{../Reflection.tex}

\begin{enumerate}
    \item What went well while writing this deliverable? 
    \item What pain points did you experience during this deliverable, and how
    did you resolve them?
    \item How did you and your team adjust the scope of your goals to ensure
    they are suitable for a Capstone project (not overly ambitious but also of
    appropriate complexity for a senior design project)?
\end{enumerate}  

\end{document}
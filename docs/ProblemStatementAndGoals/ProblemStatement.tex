\documentclass{article}
\pdfinfoomitdate=1
\pdftrailerid{}

\usepackage{tabularx}
\usepackage{booktabs}
\usepackage[round]{natbib}
\title{Problem Statement and Goals\\\progname}

\author{\authname}

\date{}

\input{../Comments}
%% Common Parts

\newcommand{\progname}{RoCam} % PUT YOUR PROGRAM NAME HERE
\newcommand{\authname}{Team \#3, SpaceY
  \\ Zifan Si
  \\ Jianqing Liu
  \\ Mike Chen
  \\ Xiaotian Lou} % AUTHOR NAMES                  

\usepackage{hyperref}
\hypersetup{colorlinks=true, linkcolor=blue, citecolor=blue, filecolor=blue,
  urlcolor=blue, unicode=false}
\urlstyle{same}

\usepackage{indentfirst}
\usepackage{graphicx}

\usepackage{titling}

\pretitle{\begin{center}\includegraphics[width=0.5\textwidth]{../../assets/logo/black.png}\\[0.75em]\LARGE}
    \posttitle{\par\end{center}}

\usepackage[letterpaper, portrait, margin=1in]{geometry}

\usepackage{placeins}
\usepackage{float}

\begin{document}

\maketitle

\begin{table}[hp]
  \caption{Revision History} \label{TblRevisionHistory}
  \begin{tabularx}{\textwidth}{llX}
    \toprule
    \textbf{Date}  & \textbf{Developer(s)} & \textbf{Change}                   \\
    \midrule
    Sept. 8, 2025  & Jianqing Liu          & Initial Draft                     \\
    Sept. 9, 2025  & Zifan Si              & Details Elaboration               \\
    Sept. 10, 2025 & Zifan Si              & Fix based on teammate suggestion  \\
    Sept. 10, 2025 & Jianqing Liu          & Refine inputs, outputs, and goals \\
    Sept. 12, 2025 & Jianqing Liu          & Update Depolyment Environment     \\
    Sept. 13, 2025 & Xiaotian Lou          & Update Citation                   \\
    \bottomrule
  \end{tabularx}
\end{table}

\newpage{}

\section{Problem Statement}
This section outlines the motivation and context behind our project, ROCAM
(Rocket Optical Camera and Motion tracking). It defines the specific
engineering problem of achieving reliable, real-time visual tracking for
small-scale model rockets using a gimbal-mounted optical camera system.
It also identifies the inputs and outputs of ROCAM, such as live camera
feeds, gimbal state data, and stabilized video streams, and states the
performance objectives and key stakeholders involved in its development.
The goal is to enable accurate mid-flight observation for analysis of
staging and parachute deployment, thereby improving both safety and
performance in future rocket launches.


\wss{You should check your problem statement with the
  \href{https://github.com/smiths/capTemplate/blob/main/docs/Checklists/ProbState-Checklist.pdf}
  {problem statement checklist}.}

\wss{You can change the section headings, as long as you include the required
  information.}

\subsection{Problem}

In model rocketry, two of the hardest engineering problems are staging failures
and parachute tangling. These problems are hard to study because they happen in
mid-flight, where direct observation is limited. Without reliable tracking,
engineers cannot fully understand these failures or design better solutions. As
a result, recurring issues remain unresolved, slowing progress in both safety
and performance of future rockets.

Model rockets travel at very high speeds, sometimes faster than Mach 3 and over
100 km in altitude \citep{SpaceConcordiaRocketry2025}. Under these conditions,
manual camera tracking is not possible.

Tracking a small model rocket is even harder than tracking large rockets such as
the Falcon 9. Smaller size and uncontrolled launch conditions make accurate
detection and continuous tracking much more difficult.

Some commercial tracking systems exist \citep{AverTR3XX2021}, but they lack the
speed and precision needed for small, fast-moving rockets. To address this gap,
our project, ROCAM, aims to develop a real-time camera tracking system that can
automatically follow model rockets during flight for clear mid-air observation.


\subsection{Inputs and Outputs}
\begin{itemize}
  \item \textbf{Inputs}
        \begin{itemize}
          \item Realtime camera feed of the rocket during flight
          \item State information from the camera gimbal (a motorized mount that stabilizes and aims the camera)
          \item Manual camera gimbal adjustment commands from user
        \end{itemize}

  \item \textbf{Outputs}
        \begin{itemize}
          \item Commands that move the camera gimbal to keep the rocket in frame
          \item Real time video preview for the operator
          \item Real time stabilized video output to the live streaming equipment
        \end{itemize}
\end{itemize}

\subsection{Performance Objectives}
\label{sec:performance}

\begin{itemize}
  \item The system should be able to process a 1080p 60 fps input video feed from the camera
  \item The System should be able to output 1080p 60fps to the live streaming equipment
  \item Real time video preview should be at least 15fps
\end{itemize}

\subsection{Stakeholders}

\subsubsection*{\color{blue}{Direct Stakeholders}}
\begin{enumerate}
    \item \textbf{McMaster Rocketry Team}: The primary end users who will deploy
    the system during launches. They rely on accurate real-time tracking to analyze
    staging events, parachute deployment, and overall flight performance.

    \item \textbf{Faculty Supervisor}: Provides technical guidance, project
    oversight, and mentorship. Dr.~Shahin Sirouspour serves in this role, ensuring
    that the project aligns with academic standards, engineering best practices,
    and capstone deliverable expectations.


\end{enumerate}

\subsubsection*{\color{blue}{Indirect Stakeholders}}
\begin{enumerate}
  \item \textbf{Aerospace Engineers and Researchers}: Benefit from
        high-quality flight footage to validate models, improve rocket
        designs, and support experimental research.

  \item \textbf{Event Organizers and Safety Officers}: Rely on
        reliable tracking for live monitoring of rocket flights,
        particularly for confirming parachute deployment and safe
        recovery during launch events.

  \item \textbf{Engineering and Robotics Community}: May adapt the
        system’s design principles for other domains requiring precise
        tracking of fast-moving objects, such as UAV navigation, sports
        analytics, or autonomous robotics.

  \item \textbf{Potential Commercial and Industrial Users}: Could
        adopt the system for broader applications in surveillance,
        wildlife monitoring, or industrial inspections where real-time
        vision-guided tracking is valuable.
\end{enumerate}

\subsection{Deployment Environment}

\begin{itemize}
  \item Outdoor rocket launch sites with variable lighting, wind, and dynamic
        backgrounds.
  \item Indoor lab-based testing and demonstration for development and evaluation.
\end{itemize}

\section{Goals}

The goal of this project is to design and implement ROCAM, a real-time camera
tracking system that automatically detects and follows small-scale model
rockets during launch, ascent, and descent (apogee < 200 m). The system aims to
maintain a stable visual lock on the rocket using a motorized gimbal, process a
1080p 60 fps video feed in real time, and output both live and recorded
stabilized footage. By meeting these goals, ROCAM will provide clear mid-flight
visual data to support analysis of staging events, parachute deployment, and
overall flight performance.


\section{Stretch Goals}

\begin{itemize}
  \item Handle 4K 60fps camera stream.
  \item Include functionalities to control the zoom and the focus of more advanced
        cameras.
  \item Integration with a full-size gimbal developed by the McMaster Rocketry Team.
  \item Track high-powered model rocket launches (apogee 3km+), while achieving all the
        performance objectives listed in Section \ref{sec:performance}.
\end{itemize}

\section{Extras}

\wss{Teams may wish to include extras as either potential bonus grades, or to
  make up for a less advanced challenge level.  Potential extras
  include usability
  testing, code walkthroughs, user documentation, formal proof, GenderMag
  personas, Design Thinking, etc.  Normally the maximum number of extras will be
  two.  Approval of the extras will be part of the discussion with
  the instructor
  for approving the project. The extras, with the approval (or request) of the
  instructor, can be modified over the course of the term.}

\subsection{Circuit Design}

As this project require to use of a mechanical gimbal, electrical circuit is
needed to interface between the host computer and the actuators. A seperate
document will be authored to detail the circuit design.

\subsection{User Instructional Video}

An user instructional video will be created to help potential users to
understand how to use the system.

\newpage{}

\section*{Appendix --- Reflection}

\wss{Not required for CAS 741}

\input{../Reflection.tex}

\begin{enumerate}
  \item What went well while writing this deliverable?
  \item What pain points did you experience during this deliverable, and how did you
        resolve them?
  \item How did you and your team adjust the scope of your goals to ensure they are
        suitable for a Capstone project (not overly ambitious but also of appropriate
        complexity for a senior design project)?
\end{enumerate}

\subsection*{\color{blue}{Jianqing Liu}}

I'm responsible for drafting most of the sections of this deliverable except
stakeholders and environment. Section 1.1 "Problem" is actually the most
difficult section to write, as the field of model rocketry is not familiar to
most people. I showed that section to multiple people and modified it based on
the feedback. I had to add a "Performance Objectives" section to further
specify the "high-performance" part of the project, which is not in the
original template, but I think it worked out well. I had some disagreements
with my teammates on the enviroument section, but we cleared up the differences
by talking to a TA. Luckly, no adjustments to the project goals were needed.
\subsection*{\color{blue}{Xiaotian Lou}}
\begin{description}
  \item[\textit{What went well while writing this deliverable?}]
        Our team started with a clear objective and a lightweight task
        management plan, which helped keep ownership visible and discussions
        focused. We also successfully connected with a faculty advisor relevant
        to our domain. Team communication was smooth and inclusive; ideas were
        surfaced early, discussed openly, and consolidated into a shared view
        without blocking progress.

  \item[\textit{What pain points did you experience during this deliverable, and how did you resolve them?}]
        Authoring in \LaTeX{} was initially time--consuming and a bit
        frustrating. We mitigated this by adopting editor plugins, templates,
        and a simple compile pipeline. Conceptually, a few sections felt vague
        at first; we resolved this through multiple short meetings, refining
        definitions, assumptions, and success criteria until the problem
        framing was concrete.

\end{description}

\subsection{\color{blue}{Shike Chen}}

The team is very cohesive with a clear goal on our project. It was very easy
for us to identify our objectives. The only struggle invovled in this
deliverable is the standards to "high-performance". The definition of
"high-performance" largely dictates the complexity of the project. In order to
constrain the scope of the project and ensure the project is suitable for a
Capstone project, we added some constrains and extra goals to confine the
domain of "high-performance".

\bibliographystyle{plainnat}
\bibliography{../../refs/References}

\end{document}
\documentclass{article}
\pdfinfoomitdate=1
\pdftrailerid{}

\usepackage{tabularx}
\usepackage{booktabs}
\usepackage[round]{natbib}
\title{Problem Statement and Goals\\\progname}

\author{\authname}

\date{}

\input{../Comments}
%% Common Parts

\newcommand{\progname}{RoCam} % PUT YOUR PROGRAM NAME HERE
\newcommand{\authname}{Team \#3, SpaceY
  \\ Zifan Si
  \\ Jianqing Liu
  \\ Mike Chen
  \\ Xiaotian Lou} % AUTHOR NAMES                  

\usepackage{hyperref}
\hypersetup{colorlinks=true, linkcolor=blue, citecolor=blue, filecolor=blue,
  urlcolor=blue, unicode=false}
\urlstyle{same}

\usepackage{indentfirst}
\usepackage{graphicx}

\usepackage{titling}

\pretitle{\begin{center}\includegraphics[width=0.5\textwidth]{../../assets/logo/black.png}\\[0.75em]\LARGE}
    \posttitle{\par\end{center}}

\usepackage[letterpaper, portrait, margin=1in]{geometry}

\usepackage{placeins}
\usepackage{float}

\begin{document}

\maketitle

\begin{table}[hp]
  \caption{Revision History} \label{TblRevisionHistory}
  \begin{tabularx}{\textwidth}{llX}
    \toprule
    \textbf{Date}  & \textbf{Developer(s)} & \textbf{Change}                   \\
    \midrule
    Sept. 8, 2025  & Jianqing Liu          & Initial Draft                     \\
    Sept. 9, 2025  & Zifan Si              & Details Elaboration               \\
    Sept. 10, 2025 & Zifan Si              & Fix based on teammate suggestion  \\
    Sept. 10, 2025 & Jianqing Liu          & Refine inputs, outputs, and goals \\
    Sept. 12, 2025 & Jianqing Liu          & Update Depolyment Environment     \\
    Sept. 13, 2025 & Xiaotian Lou          & Update Citation                   \\
    Oct.  4,  2025 & Zifan Si              & Fix based on TA feedback          \\
    \bottomrule
  \end{tabularx}
\end{table}

\newpage{}

\section{Problem Statement}
The problem statement outlines the motivation and context behind our project,
ROCAM (High Performance Vision-Guided Rocket Tracker). It defines the specific
engineering problem of achieving reliable, real-time visual tracking for
small-scale model rockets using a gimbal-mounted optical camera system. It also
identifies the inputs and outputs of ROCAM, such as live camera feeds, gimbal
state data, and stabilized video streams, and states the performance objectives
and key stakeholders involved in its development. The goal is to enable
accurate mid-flight observation for analysis of staging and parachute
deployment, thereby improving both safety and performance in future rocket
launches.

\wss{You should check your problem statement with the
  \href{https://github.com/smiths/capTemplate/blob/main/docs/Checklists/ProbState-Checklist.pdf}
  {problem statement checklist}.}

\wss{You can change the section headings, as long as you include the required
  information.}

\subsection{Problem}

In model rocketry, two of the hardest engineering problems are staging failures
and parachute tangling. These problems are hard to study because they happen in
mid-flight, where direct observation is limited. Without reliable tracking,
engineers cannot fully understand these failures or design better solutions. As
a result, recurring issues remain unresolved, slowing progress in both safety
and performance of future rockets.

Model rockets travel at very high speeds, sometimes faster than Mach 3 and over
100 km in altitude \citep{SpaceConcordiaRocketry2025}. Under these conditions,
manual camera tracking is not possible.

Tracking a small model rocket is even harder than tracking large rockets such
as the Falcon 9. Smaller size and uncontrolled launch conditions make accurate
detection and continuous tracking much more difficult.

Some commercial tracking systems exist \citep{AverTR3XX2021}, but they lack the
speed and precision needed for small, fast-moving rockets. To address this gap,
our project, ROCAM, aims to develop a real-time camera tracking system that can
automatically follow model rockets during flight for clear mid-air observation.

\subsection{Inputs and Outputs}
\begin{itemize}
  \item \textbf{Inputs}
        \begin{itemize}
          \item Realtime camera feed of the rocket during flight
          \item State information from the camera gimbal (a motorized mount that stabilizes and
                aims the camera)
          \item Manual camera gimbal adjustment commands from user
        \end{itemize}

  \item \textbf{Outputs}
        \begin{itemize}
          \item Commands that move the camera gimbal to keep the rocket in frame
          \item Real time video preview for the operator
          \item Real time stabilized video output to the live streaming equipment
        \end{itemize}
\end{itemize}

\subsection{Performance Objectives}
\label{sec:performance}

\begin{itemize}
  \item The system should be able to process a 1080p 60 fps input video feed from the
        camera
  \item The System should be able to output 1080p 60fps to the live streaming equipment
  \item Real time video preview should be at least 15fps
\end{itemize}

\subsection{Stakeholders}

\subsubsection*{Direct Stakeholders}
\begin{enumerate}
  \item \textbf{McMaster Rocketry Team}: The primary end users who will deploy
        the system during launches. They rely on accurate real-time tracking to analyze
        staging events, parachute deployment, and overall flight performance.

  \item \textbf{Faculty Supervisor}: Provides technical guidance, project
        oversight, and mentorship. Dr.~Shahin Sirouspour serves in this role, ensuring
        that the project aligns with academic standards, engineering best practices,
        and capstone deliverable expectations.

\end{enumerate}

\subsubsection*{Indirect Stakeholders}
\begin{enumerate}
  \item \textbf{Aerospace Engineers and Researchers}: Benefit from
        high-quality flight footage to validate models, improve rocket
        designs, and support experimental research.

  \item \textbf{Event Organizers and Safety Officers}: Rely on
        reliable tracking for live monitoring of rocket flights,
        particularly for confirming parachute deployment and safe
        recovery during launch events.

  \item \textbf{Engineering and Robotics Community}: May adapt the
        system’s design principles for other domains requiring precise
        tracking of fast-moving objects, such as UAV navigation, sports
        analytics, or autonomous robotics.

  \item \textbf{Potential Commercial and Industrial Users}: Could
        adopt the system for broader applications in surveillance,
        wildlife monitoring, or industrial inspections where real-time
        vision-guided tracking is valuable.
\end{enumerate}

\subsection{Deployment Environment}

\begin{itemize}
  \item Outdoor rocket launch sites with variable lighting, wind, and dynamic
        backgrounds.
  \item Indoor lab-based testing and demonstration for development and evaluation.
\end{itemize}

\section{Goals}

The goal of this project is to design and implement ROCAM, a real-time camera
tracking system that automatically detects and follows small-scale model
rockets during launch, ascent, and descent (apogee < 200 m). The system aims to
maintain a stable visual lock on the rocket using a motorized gimbal, process a
1080p 60 fps video feed in real time, and output both live and recorded
stabilized footage. By meeting these goals, ROCAM will provide clear mid-flight
visual data to support analysis of staging events, parachute deployment, and
overall flight performance.

\section{Stretch Goals}

\begin{itemize}
  \item Handle 4K 60fps camera stream.
  \item Include functionalities to control the zoom and the focus of more advanced
        cameras.
  \item Integration with a full-size gimbal developed by the McMaster Rocketry Team.
  \item Track high-powered model rocket launches (apogee 3km+), while achieving all the
        performance objectives listed in Section \ref{sec:performance}.
\end{itemize}

\section{Extras}
In addition to the core goals of ROCAM, our team plans to include a few
value-added components that enhance the overall quality and usability of the
system. These extras aim to demonstrate practical integration between hardware
and software, improve accessibility for end users, and provide clearer insight
into the system’s design and operation. The selected extras include the design
of the control circuit used to drive the gimbal actuators and the creation of a
user instructional video for system setup and operation.

\wss{Teams may wish to include extras as either potential bonus grades, or to
  make up for a less advanced challenge level.  Potential extras
  include usability
  testing, code walkthroughs, user documentation, formal proof, GenderMag
  personas, Design Thinking, etc.  Normally the maximum number of extras will be
  two.  Approval of the extras will be part of the discussion with
  the instructor
  for approving the project. The extras, with the approval (or request) of the
  instructor, can be modified over the course of the term.}

\subsection{Circuit Design}

As this project requires the use of a mechanical gimbal, an electrical circuit
is needed to interface between the host computer and the gimbal actuators. The
circuit will handle motor control, feedback sensing, and communication with the
main software to enable precise and stable camera movement.

\subsection{User Instructional Video}

An user instructional video will be created to help potential users to
understand how to use the system.

\newpage{}

\section*{Appendix --- Reflection}

\subsection*{Team Reflection (Q1--Q3)}
\begin{enumerate}
  \item What went well while writing this deliverable?
  \item What pain points did you experience during this deliverable, and how did you resolve them?
  \item How did you and your team adjust the scope of your goals to ensure they are suitable for a Capstone project (not overly ambitious but also of appropriate complexity for a senior design project)?
\end{enumerate}

\paragraph{Team Summary}
Our team collaborated effectively on this deliverable, maintaining clear communication and giving constructive feedback to improve each section. The writing process went smoothly overall, and adding the ``Performance Objectives'' section helped clarify what we meant by ``high performance'' and made our project goals more specific. Some challenges arose, such as defining what ``high performance'' should mean for our system and deciding how complex the implementation should be, but we discussed these issues together and reached a clear, realistic definition supported by measurable goals. Learning to write in \LaTeX{} and follow the template format also took time, but sharing examples and helping one another fixed most formatting issues quickly. As the main editor, keeping everyone’s updates synchronized before the deadline was difficult, so we set small internal deadlines and used GitHub to track edits efficiently. Overall, the team worked cohesively, divided tasks fairly, and refined the document through supervisor and TA feedback to produce a clear and well-structured document.

\subsection*{Individual Reflections}

\subsubsection*{\color{blue}{Jianqing Liu from hardware}}
\begin{enumerate}
  \item \textbf{Q1 (What went well?):} I drafted most sections and kept edits moving fast by sharing early versions and asking for quick comments. Section~1.1 (\emph{Problem}) improved a lot after I rewrote it with simpler words and a short example of why manual tracking fails. Adding the \emph{Performance Objectives} made our goals clearer to non-rocket readers. The team gave focused feedback, and I merged changes the same day to keep one clean version. Using Git branches for small edits avoided merge conflicts.
  \item \textbf{Q2 (Pain points \& resolution):} Defining “high performance” was hard because it could mean many things. We listed what the operator actually needs in the field (smooth video, low delay, stable tracking) and turned that into numbers (60\,fps, $<$120\,ms latency, $\le 1.5^\circ$ error). I also struggled with LaTeX formatting at first; setting up a simple compile script and a small style guide fixed most issues. When we disagreed on the environment constraints, we asked a TA to confirm what is realistic for a capstone. After that, writing went smoother.
  \item \textbf{Q3 (Scope adjustment):} I pushed to keep the core hardware simple: a two-axis gimbal with encoders and a stable mount. We moved custom PCB work and advanced sensors to “Extras” so our main goal stays on time. We agreed to prove value with one camera, one tracker, and clean video first. That plan still shows strong engineering work but avoids risky detours. If we finish early, we can add the extra hardware later.
\end{enumerate}

\subsubsection*{\color{blue}{Xiaotian Lou from QA}}
\begin{enumerate}
  \item \textbf{Q1 (What went well?):} Clear task ownership and short check-ins helped us finish drafts without last-minute rush. I focused on quality: making sure terms were consistent and each claim matched the numbers in the objectives. We kept a simple checklist for each section (goal, assumptions, metric, test), which reduced back-and-forth. Getting a faculty advisor early also helped us avoid unrealistic promises. Overall, our writing became more direct and easy to read.
  \item \textbf{Q2 (Pain points \& resolution):} LaTeX slowed us down at first, especially tables and references. I set up a minimal template, editor plugins, and a compile command so anyone could build the PDF the same way. Some sections were vague (e.g., “robustness”), so I ran short working sessions to turn them into testable statements (temperature, wind, glare). I also added a quick pass for grammar and formatting to keep the tone consistent. These steps made the document feel unified.
  \item \textbf{Q3 (Scope adjustment):} From a QA view, success must be testable. I proposed acceptance tests tied to each metric: lock percentage during ascent, glass-to-glass latency, and reacquire time. We added operating bounds to avoid undefined cases in testing (e.g., temperature range, wind limit). I recommended deferring multi-target tracking and night IR, since they would add many new tests. This keeps the test plan realistic and still rigorous.
\end{enumerate}

\subsubsection*{\color{blue}{Shike Chen from CV}}
\begin{enumerate}
  \item \textbf{Q1 (What went well?):} Team alignment made the outline simple, which helped me describe the vision pipeline clearly. We agreed to start with a reliable detector and clean data path before trying advanced models. I wrote short explanations of bbox, centroid, and control loop so the reader can follow without CV background. The link between CV output and gimbal command is now easy to see. Early diagrams and plain words reduced later rewrites.
  \item \textbf{Q2 (Pain points \& resolution):} “High performance” meant different things to us at first. I translated CV needs into numbers: minimum fps, acceptable delay from frame to command, and how much jitter is okay. We added a small telemetry log requirement so we can verify results after a test. I also simplified the wording in the CV section to avoid heavy jargon. That made reviews faster and less confusing.
  \item \textbf{Q3 (Scope adjustment):} To keep work realistic, I suggested we focus on a single-rocket day-time tracker with good reacquire, instead of multi-target or long-range RF links. We fixed baseline targets (60\,fps, $<$120\,ms latency, $\le 1.5^\circ$ error) and kept model choice flexible as long as it meets those numbers. Hard features like night scenes or strong smoke are now future work. This plan lets us deliver a strong demo and solid data first.
\end{enumerate}

\subsubsection*{\color{blue}{Zifan Si from software}}
\begin{enumerate}
  \item \textbf{Q1 (What went well?):} Separating the “why” (motivation) from the “how” (design) made the writing clearer. I tied software pieces to user needs: live preview, start/stop, reacquire, and clear errors. Linking each feature to a metric or acceptance test kept the document consistent. I also helped set up a simple folder structure and a build script so teammates could update the PDF easily. Small, frequent commits kept everyone in sync.
  \item \textbf{Q2 (Pain points \& resolution):} Balancing ambition with field limits was hard. I proposed explicit operating bounds (temperature, wind, sun angle) and a reacquire target so the scope is realistic. We wrote down latency sources (capture, processing, encode, UI) to keep the end-to-end delay under control. LaTeX details took time, but shared examples and a short style guide fixed most issues. After that, edits were faster and more consistent.
  \item \textbf{Q3 (Scope adjustment):} I recommended keeping the core deliverable to autonomous tracking plus stabilized recording with a simple operator UI. More advanced items—custom control PCB, complex RF, or multi-target—were moved to Extras. We also agreed to log key timings and states so we can prove performance. This plan keeps risk low while still leaving room for stretch goals if time allows.
\end{enumerate}

\bibliographystyle{plainnat}
\bibliography{../../refs/References}

\end{document}
\documentclass{article}

\usepackage{tabularx}
\usepackage{booktabs}

\title{Problem Statement and Goals\\\progname}

\author{\authname}

\date{}

\input{../Comments}
%% Common Parts

\newcommand{\progname}{RoCam} % PUT YOUR PROGRAM NAME HERE
\newcommand{\authname}{Team \#3, SpaceY
  \\ Zifan Si
  \\ Jianqing Liu
  \\ Mike Chen
  \\ Xiaotian Lou} % AUTHOR NAMES                  

\usepackage{hyperref}
\hypersetup{colorlinks=true, linkcolor=blue, citecolor=blue, filecolor=blue,
  urlcolor=blue, unicode=false}
\urlstyle{same}

\usepackage{indentfirst}
\usepackage{graphicx}

\usepackage{titling}

\pretitle{\begin{center}\includegraphics[width=0.5\textwidth]{../../assets/logo/black.png}\\[0.75em]\LARGE}
    \posttitle{\par\end{center}}

\usepackage[letterpaper, portrait, margin=1in]{geometry}

\usepackage{placeins}
\usepackage{float}

\begin{document}

\maketitle

\begin{table}[hp]
    \caption{Revision History} \label{TblRevisionHistory}
    \begin{tabularx}{\textwidth}{llX}
        \toprule
        \textbf{Date}  & \textbf{Developer(s)} & \textbf{Change}                   \\
        \midrule
        Sept. 8, 2025  & Jianqing Liu          & Initial Draft                     \\
        Sept. 9, 2025  & Zifan Si              & Details Elaboration               \\
        Sept. 10, 2025 & Zifan Si              & Fix based on teammate suggestion  \\
        Sept. 10, 2025 & Jianqing Liu          & Refine inputs, outputs, and goals \\
        Sept. 12, 2025 & Jianqing Liu          & Update Depolyment Environment     \\
        \bottomrule
    \end{tabularx}
\end{table}

\section{Problem Statement}

\wss{You should check your problem statement with the
    \href{https://github.com/smiths/capTemplate/blob/main/docs/Checklists/ProbState-Checklist.pdf}
    {problem statement checklist}.}

\wss{You can change the section headings, as long as you include the required
    information.}

\subsection{Problem}

In model rocketry, two of the hardest engineering problems are staging failures
and parachute tangling. These problems are hard to study because they happen in
mid-flight, where direct observation is limited. Without reliable tracking,
engineers cannot fully understand these failures or design better solutions. As
a result, recurring issues remain unresolved, slowing progress in both safety
and performance of future rockets.

Model rockets travel at very high speeds, sometimes faster than Mach 3 and over
100 km in altitude (TODO: citation). Under these conditions, manual camera
tracking is not possible.

Tracking a small model rocket is even harder than tracking large rockets suchas
the Falcon 9. Smaller size and uncontrolled launch conditions make accurate
detection and continuous tracking much more difficult.

Some commercial tracking systems exist (TODO: citation), but they do not have
the accuracy or speed needed to follow small, fast-moving rockets. This gap
shows the need for a dedicated system that can provide clear, real-time
observation of rocket flights.

\subsection{Inputs and Outputs}
\begin{itemize}
    \item \textbf{Inputs}
          \begin{itemize}
              \item Realtime camera feed
              \item State information from gimbal
              \item Manual camera gimbal adjustment commands from user
          \end{itemize}

    \item \textbf{Outputs}
          \begin{itemize}
              \item Camera gimbal adjustment commands to maintain visual lock of the rocket
              \item Realtime video preview via ethernet
              \item Realtime stabilized video output via HDMI with the rocket centered in frame
              \item Recorded video with the rocket centered in frame
          \end{itemize}
\end{itemize}

\subsection{Performance Objectives}
\label{sec:performance}

\begin{itemize}
    \item System should be able to handle 1080p 60fps camera feed
    \item System should be able to output 1080p 60fps video via HDMI
    \item Realtime video preview should be at least 15fps
\end{itemize}

\subsection{Stakeholders}

\subsubsection*{\color{blue}{Direct Stakeholders}}
\begin{enumerate}
    \item \textbf{McMaster Rocketry Team}: The primary end users who
          will deploy the system during launches. They depend on accurate
          real-time tracking to analyze staging, parachute deployment, and
          overall flight performance.

    \item \textbf{Dr. Shahin Sirouspour (Supervisor)}: Provides
          technical guidance, project oversight, and mentorship. Ensures
          the project aligns with academic standards, engineering best
          practices, and capstone deliverable expectations.
\end{enumerate}

\subsubsection*{\color{blue}{Indirect Stakeholders}}
\begin{enumerate}
    \item \textbf{Aerospace Engineers and Researchers}: Benefit from
          high-quality flight footage to validate models, improve rocket
          designs, and support experimental research.

    \item \textbf{Event Organizers and Safety Officers}: Rely on
          reliable tracking for live monitoring of rocket flights,
          particularly for confirming parachute deployment and safe
          recovery during launch events.

    \item \textbf{Engineering and Robotics Community}: May adapt the
          system’s design principles for other domains requiring precise
          tracking of fast-moving objects, such as UAV navigation, sports
          analytics, or autonomous robotics.

    \item \textbf{Potential Commercial and Industrial Users}: Could
          adopt the system for broader applications in surveillance,
          wildlife monitoring, or industrial inspections where real-time
          vision-guided tracking is valuable.
\end{enumerate}

\subsection{Environment}
\subsubsection*{\color{blue}{Deployment Environment}}
\begin{itemize}
    \item Outdoor rocket launch sites with variable lighting, wind, and dynamic
          backgrounds.
    \item Initial testing on small-scale model rockets (apogees around 200 m).
    \item Scalable to high-powered launches by the McMaster Rocketry Team
          (apogees exceeding 3 km).
    \item Indoor lab-based testing and demonstration for development and
          evaluation.
\end{itemize}

\subsubsection*{\color{blue}{Execution Environment}}
There are four pieces of the hardware related to this project:

\begin{enumerate}
    \item A gimbal with a camera mounted on it
    \item An STM32 microcontroller
    \item An Nvidia Jetson prosessing module
    \item A user laptop running a browser connected to the Jetson
\end{enumerate}

Three pieces of software will be developed as a part of this project:

\begin{enumerate}
    \item Baremetal code that runs on the STM32 microcontroller
    \item Code that runs on the Jetson, which expects an linux environment with JetPack 6
    \item Code that runs on the user browser, which expects any recent versions of major
          browsers
\end{enumerate}

\section{Goals}

The goal of the project is to track small-scale model rocket launches with
apogee \textless 200m, while achieving all the performance objectives listed in
Section \ref{sec:performance}.

\section{Stretch Goals}

\begin{itemize}
    \item Handle 4K 60fps camera stream.
    \item Include functionalities to control the zoom and the focus of more advanced
          cameras.
    \item Integration with a full-size gimbal developed by the McMaster Rocketry Team.
    \item Track high-powered model rocket launches (apogee 3km+), while achieving all the
          performance objectives listed in Section \ref{sec:performance}.
\end{itemize}

\section{Extras}

\wss{Teams may wish to include extras as either potential bonus grades, or to
    make up for a less advanced challenge level.  Potential extras
    include usability
    testing, code walkthroughs, user documentation, formal proof, GenderMag
    personas, Design Thinking, etc.  Normally the maximum number of extras will be
    two.  Approval of the extras will be part of the discussion with
    the instructor
    for approving the project. The extras, with the approval (or request) of the
    instructor, can be modified over the course of the term.}

\subsection{Circuit Design}

As this project require to use of a mechanical gimbal, electrical circuit is
needed to interface between the host computer and the actuators. A seperate
document will be authored to detail the circuit design.

\subsection{User Instructional Video}

An user instructional video will be created to help potential users to
understand how to use the system.

\newpage{}

\section*{Appendix --- Reflection}

\wss{Not required for CAS 741}

\input{../Reflection.tex}

\begin{enumerate}
    \item What went well while writing this deliverable?
    \item What pain points did you experience during this deliverable, and how did you
          resolve them?
    \item How did you and your team adjust the scope of your goals to ensure they are
          suitable for a Capstone project (not overly ambitious but also of appropriate
          complexity for a senior design project)?
\end{enumerate}

\end{document}
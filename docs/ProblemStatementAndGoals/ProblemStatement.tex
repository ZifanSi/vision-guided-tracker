\documentclass{article}

\usepackage{tabularx}
\usepackage{booktabs}

\title{Problem Statement and Goals\\\progname}

\author{\authname}

\date{}

\input{../Comments}
%% Common Parts

\newcommand{\progname}{RoCam} % PUT YOUR PROGRAM NAME HERE
\newcommand{\authname}{Team \#3, SpaceY
  \\ Zifan Si
  \\ Jianqing Liu
  \\ Mike Chen
  \\ Xiaotian Lou} % AUTHOR NAMES                  

\usepackage{hyperref}
\hypersetup{colorlinks=true, linkcolor=blue, citecolor=blue, filecolor=blue,
  urlcolor=blue, unicode=false}
\urlstyle{same}

\usepackage{indentfirst}
\usepackage{graphicx}

\usepackage{titling}

\pretitle{\begin{center}\includegraphics[width=0.5\textwidth]{../../assets/logo/black.png}\\[0.75em]\LARGE}
    \posttitle{\par\end{center}}

\usepackage[letterpaper, portrait, margin=1in]{geometry}

\usepackage{placeins}
\usepackage{float}

\begin{document}

\maketitle

\begin{table}[hp]
\caption{Revision History} \label{TblRevisionHistory}
\begin{tabularx}{\textwidth}{llX}
\toprule
\textbf{Date} & \textbf{Developer(s)} & \textbf{Change}\\
\midrule
Sept. 8, 2025 & Jianqing Liu & Initial Draft\\
Sept. 9, 2025 & Zifan Si & Details Elaboration\\
Sept. 10, 2025 & Zifan Si & Fix based on teammate suggestion\\
... & ... & ...\\
\bottomrule
\end{tabularx}
\end{table}

\section{Problem Statement}

\wss{You should check your problem statement with the
\href{https://github.com/smiths/capTemplate/blob/main/docs/Checklists/ProbState-Checklist.pdf}
{problem statement checklist}.} 

\wss{You can change the section headings, as long as you include the required
information.}

\subsection{Problem}

In model rocketry, two of the hardest engineering problems are staging failures
and parachute tangling. These problems are hard to study because they happen in
mid-flight, where direct observation is limited. Without reliable tracking,
engineers cannot fully understand these failures or design better solutions. As
a result, recurring issues remain unresolved, slowing progress in both safety
and performance of future rockets.

Model rockets travel at very high speeds, sometimes faster than Mach 3 and over
100 km in altitude (TODO: citation). Under these conditions, manual camera
tracking is not possible.

Tracking a small model rocket is even harder than tracking large rockets such as
the Falcon 9. Smaller size and uncontrolled launch conditions make accurate
detection and continuous tracking much more difficult.

Some commercial tracking systems exist (TODO: citation), but they do not have
the accuracy or speed needed to follow small, fast-moving rockets. This gap
shows the need for a dedicated system that can provide clear, real-time
observation of rocket flights.

\subsection{Inputs and Outputs}
\begin{itemize}
    \item \textbf{Inputs}
    \begin{itemize}
        \item 1080p 60fps camera feed of a rocket
        \item Basic system state information for tracking
    \end{itemize}

    \item \textbf{Outputs}
    \begin{itemize}
        \item Camera orientation adjustments to maintain object lock
        \item Stabilized video stream with the object centered in frame
        \item User interface for monitoring and control
    \end{itemize}
\end{itemize}


\subsection{Stakeholders}

\subsubsection*{\color{blue}{Direct Stakeholders}}
\begin{enumerate}
    \item \textbf{Capstone Development Team}: Designs, implements, and validates the vision-guided tracking system. They are responsible for delivering a production-ready, reliable, and well-documented solution that meets course and project requirements.  

    \item \textbf{McMaster Rocketry Team}: The primary end users who will deploy the system during launches. They depend on accurate real-time tracking to analyze staging, parachute deployment, and overall flight performance.  

    \item \textbf{Dr. Shahin Sirouspour (Supervisor)}: Provides technical guidance, project oversight, and mentorship. Ensures the project aligns with academic standards, engineering best practices, and capstone deliverable expectations.  
\end{enumerate}

\subsubsection*{\color{blue}{Indirect Stakeholders}}
\begin{enumerate}
    \item \textbf{Aerospace Engineers and Researchers}: Benefit from high-quality flight footage to validate models, improve rocket designs, and support experimental research.  

    \item \textbf{Event Organizers and Safety Officers}: Rely on reliable tracking for live monitoring of rocket flights, particularly for confirming parachute deployment and safe recovery during launch events.  

    \item \textbf{Engineering and Robotics Community}: May adapt the system’s design principles for other domains requiring precise tracking of fast-moving objects, such as UAV navigation, sports analytics, or autonomous robotics.  

    \item \textbf{Potential Commercial and Industrial Users}: Could adopt the system for broader applications in surveillance, wildlife monitoring, or industrial inspections where real-time vision-guided tracking is valuable.  
\end{enumerate}


\subsection{Environment}
\textbf{Development Frameworks and Tools:}
\begin{enumerate}

    \item \textit{GitLab} will be used for version control, project management, and CI/CD pipelines to automate testing and deployment.
    
    \item \textit{Visual Studio Code} will serve as the primary IDE for software development across the embedded system, computer vision pipeline, and web application components.  

    \item \textit{GitHub Actions} (or GitLab CI) will provide automated unit/integration testing and continuous integration workflows for reliable code validation.  

    \item \textit{NVIDIA Jetson Orin Nano SDK} will be used for computer vision model execution, GPU acceleration, and performance optimization.  

    \item \textit{STM32 Development Tools} (STM32CubeIDE, OpenOCD, or equivalent) will support programming and debugging of the embedded gimbal motion controller.  

\end{enumerate}


\subsection{Gimbal}

An off-the-shelf cheap gimbal is used.

A custom developed PCB is used to adapt the gimbal to the Jetson.

\subsection{Computer Vision}

Nvidia Jetson Orin Nano Super

\subsection{Web Management Portal}

Runs on any recent web browser

\section{Goals}

Track small-scale rocket launches (apogee < 200m)

\section{Stretch Goals}

connected to a full-size gimbal developed by the McMaster Rocketry Team to track high-powered rocket launches (apogee 3km+)

\section{Extras}

\wss{Teams may wish to include extras as either potential bonus grades, or to
make up for a less advanced challenge level.  Potential extras include usability
testing, code walkthroughs, user documentation, formal proof, GenderMag
personas, Design Thinking, etc.  Normally the maximum number of extras will be
two.  Approval of the extras will be part of the discussion with the instructor
for approving the project.  The extras, with the approval (or request) of the
instructor, can be modified over the course of the term.}

\subsection{Circuit Design}

\subsection{TODO: Extra 2}


\newpage{}

\section*{Appendix --- Reflection}

\wss{Not required for CAS 741}

\input{../Reflection.tex}

\begin{enumerate}
    \item What went well while writing this deliverable? 
    \item What pain points did you experience during this deliverable, and how
    did you resolve them?
    \item How did you and your team adjust the scope of your goals to ensure
    they are suitable for a Capstone project (not overly ambitious but also of
    appropriate complexity for a senior design project)?
\end{enumerate}  

\end{document}
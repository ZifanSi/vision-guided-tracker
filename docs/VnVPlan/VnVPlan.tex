\documentclass[12pt, titlepage]{article}
\pdfinfoomitdate=1
\pdftrailerid{}

\usepackage{booktabs}
\usepackage{tabularx}
\usepackage{hyperref}
\hypersetup{
    colorlinks,
    citecolor=blue,
    filecolor=black,
    linkcolor=red,
    urlcolor=blue
}
\usepackage[round]{natbib}
\usepackage{pifont}

\input{../Comments}
%% Common Parts

\newcommand{\progname}{RoCam} % PUT YOUR PROGRAM NAME HERE
\newcommand{\authname}{Team \#3, SpaceY
  \\ Zifan Si
  \\ Jianqing Liu
  \\ Mike Chen
  \\ Xiaotian Lou} % AUTHOR NAMES                  

\usepackage{hyperref}
\hypersetup{colorlinks=true, linkcolor=blue, citecolor=blue, filecolor=blue,
  urlcolor=blue, unicode=false}
\urlstyle{same}

\usepackage{indentfirst}
\usepackage{graphicx}

\usepackage{titling}

\pretitle{\begin{center}\includegraphics[width=0.5\textwidth]{../../assets/logo/black.png}\\[0.75em]\LARGE}
    \posttitle{\par\end{center}}

\usepackage[letterpaper, portrait, margin=1in]{geometry}

\usepackage{placeins}
\usepackage{float}

\begin{document}

\title{System Verification and Validation Plan for \progname{}}
\author{\authname}
\date{\today}

\maketitle

\pagenumbering{roman}

\section*{Revision History}

\begin{tabularx}{\textwidth}{p{3cm}p{2cm}X}
  \toprule {\bf Date} & {\bf Version} & {\bf Notes}   \\
  \midrule
  Date 1              & 1.0           & Initial Draft \\
  \bottomrule
\end{tabularx}

~\\
\wss{The intention of the VnV plan is to increase confidence in the software.
  However, this does not mean listing every verification and validation technique
  that has ever been devised.  The VnV plan should also be a \textbf{feasible}
  plan. Execution of the plan should be possible with the time and team available.
  If the full plan cannot be completed during the time available, it can either be
  modified to ``fake it'', or a better solution is to add a section describing
  what work has been completed and what work is still planned for the future.}

\wss{The VnV plan is typically started after the requirements stage, but before
  the design stage.  This means that the sections related to unit testing cannot
  initially be completed.  The sections will be filled in after the design stage
  is complete.  the final version of the VnV plan should have all sections filled
  in.}

\newpage

\tableofcontents

\section{Symbols, Abbreviations, and Acronyms}

\renewcommand{\arraystretch}{1.2}
\begin{tabular}{l l}
  \toprule
  \textbf{symbol} & \textbf{description}                                             \\
  \midrule
  Yolo            & You only look once -- an existing object detection model         \\
  CNN             & Convolutional Neural Network                                     \\
  Jetson          & abbreviatio for specifcally Jetson Nano Orin, which is the
  proposed board for this project
  \\
  ST32            & abbreviation for specifically ST32 microcontroller, which is the
  proposed microcontroller for this project                                          \\

  \bottomrule
\end{tabular}\\

\wss{symbols, abbreviations, or acronyms --- you can simply reference the SRS
  \citep{SRS} tables, if appropriate}

\wss{Remove this section if it isn't needed}

\newpage

\pagenumbering{arabic}

This Verification and Validation (VnV) plan establishes the evidence that our
RoCam model-rocket tracking system is both built right and the right build.
Verification confirms each SRS and design requirement—camera and gimbal specs,
timing/latency budgets, fault handling, and code quality—via reviews, static
analysis, unit/integration tests, bench and simulation runs, with full
requirement-to-test traceability. Validation demonstrates the system fulfills
user and mission needs in field conditions by acquiring the rocket at launch,
maintaining continuous tracking through boost/coast and occlusions, and
supporting a safe, simple operator workflow. Success is judged against
quantitative targets and operational criteria. Deliverables include test
plans/cases, automated reports, calibration and operating procedures,
field-test results, and a validation summary with stakeholder sign-off.

\hl{TODO: tidy up deliverables list}

\wss{provide an introductory blurb and roadmap of the
  Verification and Validation plan}

\section{General Information}

\subsection{Summary}

The RoCam system is a ground-based camera and gimbal assembly designed to
autonomously track model rockets during launch and flight. It utilizes a
high-resolution camera coupled with a motorized gimbal to maintain visual lock
on the rocket, providing real-time data and video feed. The system consists of
a STM32 based motion controller that interfaces with the physical gimbal; a
Jetson compute module that processes the camera feed; and a react frontend that
displays the video preview and allows the user to control the gimbal. The
primary purpose of RoCam is to replace the traditional unreliable manual
tracking methods with an automated solution to produce high-quality flight data
and video footage. To better assist model rocket teams across Canada to conduct
successful launches by providing an affordable and reliable system that can be
easily track launching movements.

\wss{Say what software is being tested.  Give its name and a brief overview of
  its general functions.}

\subsection{Objectives}
The primary objective of the system is to \textbf{demonstrate
  reliable, real-time visual tracking of small-scale model rockets} through an
autonomous, vision-guided camera system. The project aims to achieve
\textbf{high accuracy}, \textbf{low latency}, and \textbf{stable video
  performance}, ensuring that the rocket remains centered in frame throughout its
flight. By focusing on \ \textbf{system performance}, \textbf{tracking
  precision}, and \textbf{robustness under real launch conditions}, RoCam seeks
to provide high-quality flight footage that supports post-launch analysis and
safety validation.

\subsection{In-Scope Objectives}
Verification and validation efforts will prioritize the following qualities:
\begin{itemize}
  \item \textbf{Correctness:} Ensure the reliability of tracking and control
        algorithms through testing and validation.
  \item \textbf{Performance:} Achieve real-time processing of 1080p 60\,fps
        video and maintain minimal gimbal control latency.
  \item \textbf{Reliability and Safety:} Guarantee that the system operates s
        afely and consistently under outdoor launch conditions.
\end{itemize}

\subsection{Out-of-Scope Objectives}
The following objectives are excluded from this project due to time, budget,
and resource constraints:
\begin{itemize}
  \item \textbf{Comprehensive Usability Testing:} User experience will be evaluated
        informally via team and client feedback rather than large-scale formal studies.
  \item \textbf{Third-Party Component Verification:} External hardware such as the
        gimbal controller and camera are assumed to perform according to their
        manufacturer specifications.
  \item \textbf{Scalability Beyond Small-Scale Rockets:} Tracking a
        bullet or extremely difficult
        targets beyond a model rocket is outside the scope of this project.
\end{itemize}

By defining these priorities and exclusions, the RoCam project ensures that
available effort is directed toward validating the most critical system
qualities: \textbf{performance, accuracy, and operational safety}.

\wss{State what is intended to be accomplished.  The objective will be around
  the qualities that are most important for your project.  You might have
  something like: ``build confidence in the software correctness,''
  ``demonstrate adequate usability.'' etc.  You won't list all of the qualities,
  just those that are most important.}

\wss{You should also list the objectives that are out of scope.  You don't have
  the resources to do everything, so what will you be leaving out.  For instance,
  if you are not going to verify the quality of usability, state this.  It is also
  worthwhile to justify why the objectives are left out.}

\wss{The objectives are important because they highlight that you are aware of
  limitations in your resources for verification and validation.
  You can't do everything, so what are you going to prioritize?
  As an example, if your system depends on an external library,
  you can explicitly state that you will assume that external library
  has already been verified by its implementation team.}

\subsection{Challenge Level and Extras}

The \textbf{RoCam} project is an engineering heavy capstone project. One of the
major challenges of this project is the integration of multiple complex
subsystems, including \textbf{real-time computer vision},
\textbf{hardware-software interfacing}, and \textbf{autonomous motion control}.
The project requires the implementation of algorithms capable of detecting,
tracking, and maintaining a visual lock on fast-moving model rockets, while
ensuring real-time performance at 1080p\,60\,fps and stable gimbal control
under dynamic outdoor conditions. In addition, the system's design encompasses
both software engineering and embedded hardware domains, further contributing
to its advanced technical scope.

\subsubsection*{Extras}
Beyond the core objectives, the RoCam project includes several
\textbf{approved extras} that enhance its quality and usability:
\begin{itemize}
  \item \textbf{STM32 control circuit design:} Development of a
        custom control circuit to interface the gimbal actuators with
        the host computer, enabling precise motion control and feedback sensing.
  \item \textbf{User Instructional Video:} Creation of a short instructional
        video demonstrating system setup, calibration, and operation to support
        new users and ensure safe deployment.
  \item \textbf{Usability Evaluation:} Informal field-based usability
        evaluation with feedback from the McMaster Rocketry Team to validate
        ease of use and interface clarity.
\end{itemize}

These extras aim to improve the overall usability, accessibility, and
documentation quality of the final deliverable, reinforcing RoCam's emphasis on
practical integration between computer vision software, mechanical control, and
end-user experience.

\wss{State the challenge level (advanced, general, basic) for your project.
  Your challenge level should exactly match what is included in your problem
  statement.  This should be the challenge level agreed on between you and the
  course instructor.  You can use a pull request to update your challenge level
  (in TeamComposition.csv or Repos.csv) if your plan changes as a result of the
  VnV planning exercise.}

\wss{Summarize the extras (if any) that were tackled by this project.  Extras
  can include usability testing, code walkthroughs, user documentation, formal
  proof, GenderMag personas, Design Thinking, etc.  Extras should have already
  been approved by the course instructor as included in your problem statement.
  You can use a pull request to update your extras (in TeamComposition.csv or
  Repos.csv) if your plan changes as a result of the VnV planning exercise.}

\subsection{Relevant Documentation}

IDK What to do here \wss{Reference relevant documentation. This will definitely
  include your SRS and your other project documents (design documents, like MG,
  MIS, etc). You can include these even before they are written, since by the
  time the project is done, they will be written. You can create BibTeX entries
  for your documents and within those entries include a hyperlink to the
  documents.}

\citet{SRS}

\wss{Don't just list the other documents.  You should explain why they are relevant and
  how they relate to your VnV efforts.}

\section{Plan}

\wss{Introduce this section.  You can provide a roadmap of the sections to
  come.}

\subsection{Verification and Validation Team}

\wss{Your teammates.  Maybe your supervisor.
  You should do more than list names.  You should say what each person's role is
  for the project's verification.  A table is a good way to summarize this information.}

\subsection{SRS Verification}

The verification and validation (V\&V) process for the \textbf{RoCam} project
is carried out collaboratively by all team members under the supervision of the
faculty advisor. Each member is responsible for verifying specific components
of the system based on their primary area of contribution, ensuring that both
software and hardware subsystems meet performance, safety, and usability
requirements. Table~\ref{tab:vnv_team} summarizes the V\&V responsibilities for
each team member.

\begin{table}[H]
  \centering
  \caption{Verification and Validation Team Responsibilities}
  \label{tab:vnv_team}
  \begin{tabular}{|p{3.5cm}|p{3.5cm}|p{8cm}|}
    \hline
    \textbf{Name}                  & \textbf{Role}                     & \textbf{V\&V Responsibilities} \\ \hline

    \textbf{Zifan Si}              & Team member                       &
    Responsible for verifying and validating the usability and functionality of the
    react frontend interface.
    Ensure usability, robustness of frontend features.                                                  \\ \hline

    \textbf{Jianqing Liu}          & Team Lead                         &
    Conduct integration test and system validation. Validate the functionality and
    performance of the system against requirements and use cases.                                       \\ \hline

    \textbf{Mike Chen}             & Hardware and Control Lead         &
    Verify and validate computer vision pipeline performance, including object detection
    and tracking accuracy.                                                                              \\ \hline

    \textbf{Xiaotian Lou}          & Software Quality and Testing Lead &
    Implement automated testing frameworks in all software components.                                  \\ \hline

    \textbf{Dr. Shahin Sirouspour} & Faculty Supervisor                &
    Provides oversight and ensures that verification and validation procedures
    align with academic and engineering standards.
    Reviews test methodologies, validates safety-critical procedures, and
    confirms that deliverables meet course and institutional requirements.                              \\ \hline

    \textbf{Launch Canada}         & Stakeholder                       &
    Provides feedback on system performance during field tests.
    \\ \hline

  \end{tabular}
\end{table}

Together, this team structure ensures that verification and validation
activities are distributed across all critical system components—computer
vision, control, integration, and software reliability—resulting in a robust
and well-tested final product.

\wss{List any approaches you intend to use for SRS verification.  This may
  include ad hoc feedback from reviewers, like your classmates (like your
  primary reviewer), or you may plan for something more rigorous/systematic.}

\wss{If you have a supervisor for the project, you shouldn't just say they will
  read over the SRS.  You should explain your structured approach to the review.
  Will you have a meeting?  What will you present?  What questions will you ask?
  Will you give them instructions for a task-based inspection?  Will you use your
  issue tracker?}

\wss{Maybe create an SRS checklist?}

\subsection{Design Verification}

The design verification process for the \textbf{RoCam} project ensures that the
proposed architecture, algorithms, and hardware interfaces satisfy all
functional and non-functional requirements before implementation. Design
verification activities will be conducted through structured design reviews,
peer evaluations, and the use of detailed verification checklists. The goal is
to confirm that the design meets the intended performance, reliability, and
safety objectives defined in the SRS.

\subsubsection*{Verification Plan}
The design verification will proceed in three main stages:
\begin{enumerate}
  \item \textbf{Internal Design Review:}
        Each subsystem (computer vision, control, and UI)
        will undergo an internal review by each team member. Each member must
        agree on the designed architecture, algorithms, and interfaces.
        Interfaces must be clearly defined for scalability.

  \item \textbf{Peer Review by Classmates:}
        A peer design review session will be conducted with other capstone
        teams to obtain external feedback.
        Reviewers will evaluate the system’s clarity, feasibility, and
        maintainability based on the shared design artifacts (block diagrams
        , interface specifications, and test plans).

  \item \textbf{Supervisor and TA Review:}
        The project supervisor and teaching assistants will review the
        finalized design to ensure compliance with academic standards,
        capstone expectations, and safety considerations.
\end{enumerate}

\subsubsection*{Verification Artifacts}
Design verification will produce the following artifacts:
\begin{itemize}
  \item Verified design diagrams and interface specifications.
  \item Annotated review feedback forms from classmates and supervisor.
  \item Updated design documents reflecting corrections and improvements.
  \item Completed design verification checklists (see below).
\end{itemize}

\subsubsection*{Design Verification Checklist}
The following checklist will be used during design reviews to ensure
completeness and consistency across all modules:

\newpage

\begin{table}[H]
  \centering
  \caption{RoCam Design Verification Checklist}
  \label{tab:design_verification_checklist}
  \begin{tabular}{|p{10cm}|c|}
    \hline
    \textbf{Verification Item}                              & \textbf{Status (\ding{51}/\ding{55})} \\ \hline

    \textbf{Requirements Traceability:} Each design module (vision,
    control, UI, data management) maps directly to specific functional and
    non-functional requirements in the SRS.                 &                                       \\ \hline

    \textbf{System Architecture Validation:} The proposed architecture
    supports concurrent data acquisition, vision inference, and gimbal
    control without violating latency constraints.          &                                       \\ \hline

    \textbf{Performance Objectives:} Design supports 1080p 60 fps video
    input/output, with an end-to-end latency target of less than 120 ms
    between camera frame and gimbal response.               &                                       \\ \hline

    \textbf{Rocket Tracking Algorithm:} The computer vision subsystem’s
    design ensures accurate rocket detection, tracking, and reacquisition
    under changing lighting and smoke conditions.           &                                       \\ \hline

    \textbf{Gimbal Control and Stability:} The control architecture
    provides smooth, stable movement
    and sufficient angular speed for small rocket tracking. &                                       \\ \hline

    \textbf{Hardware–Software Interface:} Communication between the Jetson
    platform, STM32 controller, and gimbal is well-defined, with verified
    protocol timing and error handling.                     &                                       \\ \hline

    \textbf{Safety and Fault Handling:} Safe-state transitions (Idle,
    Armed, Tracking) are fully defined and prevent uncontrolled gimbal
    motion on error or manual override.                     &                                       \\ \hline

    \textbf{Reliability and Robustness:} Design includes fail-safes for
    connection loss, power interruption, and unexpected frame drop events
    during operation.                                       &                                       \\ \hline

    \textbf{User Interface Design:} The UI supports outdoor readability,
    bilingual support (English/French), and visual status indicators for
    all operational states.                                 &                                       \\ \hline

    \textbf{Data Recording and Storage:} Recording subsystem design
    supports synchronized video and gimbal-angle logging at 1080p 60 fps,
    with enough local storage for a full launch day.        &                                       \\ \hline

    \textbf{Testing and Validation Plan Alignment:} Each subsystem has
    a defined verification procedure (unit tests, simulation, or field test)
    linked to corresponding SRS fit criteria.               &                                       \\ \hline

    \textbf{Peer Review and Feedback Resolution:} All feedback from peer
    design reviews and supervisor evaluations has been documented and
    addressed in the updated design artifacts.              &                                       \\ \hline

  \end{tabular}
\end{table}

\noindent
By following this structured verification plan and checklist, the team
ensures that the RoCam design is both technically sound and fully aligned
with its performance and safety goals prior to implementation.

\wss{Plans for design verification}

\wss{The review will include reviews by your classmates}

\wss{Create a checklists?}

\subsection{Verification and Validation Plan Verification}

\wss{The verification and validation plan is an artifact that should also be
  verified.  Techniques for this include review and mutation testing.}

\wss{The review will include reviews by your classmates}

\wss{Create a checklists?}

\subsection{Implementation Verification}

The implementation verification for the \textbf{RoCam} project ensures that the
developed software and integrated hardware components correctly realize the
verified design. This phase confirms that the implemented system satisfies the
functional and non-functional requirements defined in the SRS through a
combination of structured testing, static verification, and peer review.

\subsubsection*{Dynamic Verification}
Dynamic verification activities focus on confirming that the system
behaves as intended under realistic operating conditions.
They include the following elements:

\begin{itemize}
  \item \textbf{Unit Testing:}
        Each software module (vision processing, gimbal control, data
        recording, and UI management) will undergo unit testing to verify
        functional correctness in isolation.

  \item \textbf{Integration Testing:}
        Once individual modules are verified, integration tests will
        confirm proper data flow between subsystems (camera input
        $\rightarrow$ vision algorithm $\rightarrow$ gimbal control
        $\rightarrow$ UI).
        Tests will ensure synchronization, timing correctness, and
        robustness under real-time constraints.

  \item \textbf{System Testing:}
        Full-system field tests will validate end-to-end performance,
        including real-time video tracking, system state transitions
        (Idle, Armed, Tracking), and video recording accuracy at 1080p\,60\,fps.
        These tests correspond directly to the verification items in
        the SRS functional and performance requirements.
\end{itemize}

\subsubsection*{Static Verification}
Static verification will be performed throughout the implementation
phase to ensure that the codebase meets quality, safety,
nd maintainability standards before execution.
Planned techniques include:

\begin{itemize}
  \item \textbf{Code Walkthroughs:}
        Conducted during the final CAS~741 presentation, where each
        subsystem lead presents core implementation logic and design decisions.
        This serves as a semi-formal walkthrough for peer and instructor
        review, enabling early identification of potential design or coding issues.

  \item \textbf{Code Inspection:}
        Team members will review each other’s pull requests prior to
        merging into the main branch.
        Reviews will focus on coding standard compliance, potential
        logic errors, exception handling, and safety-critical sections
        (e.g., gimbal actuation and state transitions).

  \item \textbf{Static Analysis Tools:}
        Tools such as \texttt{clang-tidy}, \texttt{cppcheck}, and
        \texttt{pylint} will be used to detect memory leaks, uninitialized variables,
        and other common implementation defects.
        Results from static analysis will be logged and reviewed
        as part of the verification record.

  \item \textbf{Documentation Review:}
        The team will verify that all inline comments, module-level
        documentation, and API references are consistent with the design
        specification and user documentation requirements.
\end{itemize}

\subsubsection*{Presentation and Usability Review}
The final CAS~741 presentation will serve as both a
\textbf{code walkthrough} and a partial \textbf{usability evaluation}.
During the presentation, peers and instructors will observe live
system operation, assess user interface clarity, and provide
feedback on usability, responsiveness, and overall system behavior.
This qualitative feedback will be incorporated into the final
verification summary and used to refine user-facing components
prior to final submission.

\noindent
Together, these verification activities ensure that the RoCam
implementation is thoroughly tested, statically verified, and
validated against both its design and real-world performance expectations.

\subsection{Automated Testing and Verification Tools}

For Phase~1, the implementation is \textbf{Python-first} (OpenCV pipeline,
control logic, and operator UI services). Our automation focuses on Python and
the web UI; any C/C++ tools will be introduced only in a later phase if needed.

\subsubsection*{Unit and Integration Testing}
\begin{itemize}
  \item \textbf{Python:} \texttt{pytest} with \texttt{pytest-cov}; fixtures
        simulate camera frames (OpenCV \texttt{VideoCapture} stubs), gimbal feedback,
        and UI requests.
  \item \textbf{Web UI (if applicable):} Jest + Testing Library for component
        tests; Playwright for smoke-level end-to-end checks covering the operator
        workflow (arm~$\rightarrow$~track~$\rightarrow$~stop~$\rightarrow$~export).
\end{itemize}

\subsubsection*{Static Analysis, Linters, and Style}
\begin{itemize}
  \item \textbf{Python:} \texttt{flake8} and \texttt{pylint} (lint),
        \texttt{mypy} (type checking for critical paths), \texttt{bandit} (basic security checks).
  \item \textbf{Formatting:} \texttt{black} + \texttt{isort}; enforced
        via pre-commit hooks and CI.
  \item \textbf{Web UI:} ESLint with project rules; Prettier for formatting.
\end{itemize}

\subsubsection*{Coverage and Build Automation}
\begin{itemize}
  \item \textbf{Coverage (Python):} \texttt{coverage.py} via \texttt{pytest-cov};
        HTML reports published by CI.
  \item \textbf{Coverage (Web UI):} Istanbul/NYC for statements/branches/functions/lines.
  \item \textbf{Build/Tasks:} \texttt{pip} + \texttt{pyproject.toml}/\texttt{requirements.txt};
        \texttt{npm} scripts for the UI.
  \item \textbf{CI/CD:} GitLab CI pipelines run on every push/MR; jobs enforce: lints clean,
        type check passes, unit tests pass, and minimum coverage gates.
\end{itemize}

\subsubsection*{Profiling and Runtime Diagnostics}
\begin{itemize}
  \item \textbf{Python:} \texttt{cProfile}, \texttt{line\_profiler}, and
        \texttt{memory\_profiler} on CV hot paths; timestamped logs to measure
        camera$\rightarrow$inference$\rightarrow$gimbal command latency.
  \item \textbf{Observability:} Structured logs (JSON) with frame IDs and timing;
        optional Prometheus-style counters for latency histograms (median/95th).
\end{itemize}

\subsubsection*{Coverage Reporting Plan}
\begin{itemize}
  \item \textbf{Thresholds:} Line coverage $\geq 80\%$ for safety-critical
        modules (state machine, tracking loop, command arbitration); $\geq 70\%$ for other
        modules. Branch coverage is reported for state-transition logic.
  \item \textbf{Reports:} CI publishes HTML artifacts and a coverage badge; weekly trend
        tracked and cited in the V\&V report.
  \item \textbf{Scope Notes:} Hardware I/O shims (e.g., serial/gimbal stubs) are measured
        separately; excluded/generated code is documented.
\end{itemize}

\wss{What tools are you using for automated testing.  Likely a unit testing
  framework and maybe a profiling tool, like ValGrind.  Other possible tools
  include a static analyzer, make, continuous integration tools, test coverage
  tools, etc.  Explain your plans for summarizing code coverage metrics.
  Linters are another important class of tools.  For the programming language
  you select, you should look at the available linters.  There may also be tools
  that verify that coding standards have been respected, like flake9 for
  Python.}

\wss{If you have already done this in the development plan, you can point to
  that document.}

\wss{The details of this section will likely evolve as you get closer to the
  implementation.}

\subsection{Software Validation}

Software validation confirms that \textbf{RoCam} satisfies stakeholder needs
and accurately implements the intended use cases (``building the right
system''). Validation complements verification and is referenced to the SRS
goals nd stakeholder roles.

\subsubsection*{Stakeholder Reviews and Task-Based Inspection}
\begin{itemize}
  \item \textbf{Requirements Walkthroughs:} Scheduled reviews of the SRS
        and UI wireframes with the McMaster Rocketry Team (client) to confirm
        scope, priorities, and acceptance criteria.
  \item \textbf{Task-Based Inspection:} Stakeholders perform
        representative tasks (arm system, initiate tracking, stop tracking, export
        footage) while observers record issues, ambiguities, and unmet expectations.
  \item \textbf{Peer Reviews:} Classmate reviews focus on clarity,
        testability, and completeness (per CAS~741 guidance); all feedback and
        resolutions are logged.
\end{itemize}

\subsubsection*{Field and Demo-Based Validation}
\begin{itemize}
  \item \textbf{Rev~0/PoC Demo:} Used to validate that early
        functionality aligns with stakeholder expectations (e.g., preview
        latency, arm/track state semantics). Post-demo debrief captures change
        requests.
  \item \textbf{Supervisor Validation:} A focused session shortly
        after Rev~0 to confirm that the demonstrable behaviors match the
        SRS’s product use cases and performance intent.
  \item \textbf{User (Operator) Sessions:} ``Hallway'' usability
        checks with camera operators validate readability in outdoor-like
        conditions and UI flow (single-page ops without scrolling on 1920$\times$1080).
\end{itemize}

\subsubsection*{External Data and Bench Validation}
\begin{itemize}
  \item \textbf{Reference Footage:} Where live launches are unavailable,
        public model-rocket videos and internally recorded dry-runs are used
        to validate detection/tracking logic and reacquisition behavior.
  \item \textbf{Synthetic Scenarios:} Scripted clips with
        occlusion/smoke/backlight and varying apparent rocket sizes validate
        robustness claims prior to field tests.
\end{itemize}

\subsubsection*{Acceptance and Traceability}
\begin{itemize}
  \item \textbf{Acceptance Criteria:} Derived from the SRS (e.g.,
        sustained 1080p\,60\,fps I/O, end-to-end latency $\leq 120$\,ms,
        successful tracking through nominal flight phases, recording with a
        ngle overlays).
  \item \textbf{Traceability:} Validation tests map to SRS goals and
        stakeholder needs; a table links each validation activity to the
        corresponding SRS item(s) and planned evidence (videos, logs, checklists).
\end{itemize}

\noindent
If no suitable external datasets are available for certain scenarios,
this limitation will be stated explicitly; in those cases, validation
will rely on stakeholder task-based inspections and controlled synthetic
footage. This section also references the SRS verification section for
consistency checks between validated behaviors and specified requirements.

\wss{If there is any external data that can be used for validation, you should
  point to it here.  If there are no plans for validation, you should state that
  here.}

\wss{You might want to use review sessions with the stakeholder to check that
  the requirements document captures the right requirements.  Maybe task based
  inspection?}

\wss{For those capstone teams with an external supervisor, the Rev 0 demo should
  be used as an opportunity to validate the requirements.  You should plan on
  demonstrating your project to your supervisor shortly after the scheduled Rev 0 demo.
  The feedback from your supervisor will be very useful for improving your project.}

\wss{For teams without an external supervisor, user testing can serve the same purpose
  as a Rev 0 demo for the supervisor.}

\wss{This section might reference back to the SRS verification section.}

\section{System Tests}

\wss{There should be text between all headings, even if it is just a roadmap of
  the contents of the subsections.}

\subsection{Tests for Functional Requirements}

\wss{Subsets of the tests may be in related, so this section is divided into
  different areas.  If there are no identifiable subsets for the tests, this
  level of document structure can be removed.}

\wss{Include a blurb here to explain why the subsections below
  cover the requirements.  References to the SRS would be good here.}

\subsubsection{Area of Testing1}

\wss{It would be nice to have a blurb here to explain why the subsections below
  cover the requirements.  References to the SRS would be good here.  If a section
  covers tests for input constraints, you should reference the data constraints
  table in the SRS.}

\paragraph{Title for Test}

\begin{enumerate}

  \item{test-id1\\}

        Control: Manual versus Automatic

        Initial State:

        Input:

        Output: \wss{The expected result for the given inputs. Output is not how you
          are going to return the results of the test. The output is the expected
          result.}

        Test Case Derivation: \wss{Justify the expected value given in the Output
          field}

        How test will be performed:

  \item{test-id2\\}

        Control: Manual versus Automatic

        Initial State:

        Input:

        Output: \wss{The expected result for the given inputs}

        Test Case Derivation: \wss{Justify the expected value given in the Output
          field}

        How test will be performed:

\end{enumerate}

\subsubsection{Area of Testing2}

...

\subsection{Tests for Nonfunctional Requirements}

\wss{The nonfunctional requirements for accuracy will likely just reference the
  appropriate functional tests from above.  The test cases should mention
  reporting the relative error for these tests.  Not all projects will
  necessarily have nonfunctional requirements related to accuracy.}

\wss{For some nonfunctional tests, you won't be setting a target threshold for
  passing the test, but rather describing the experiment you will do to measure
  the quality for different inputs.  For instance, you could measure speed versus
  the problem size.  The output of the test isn't pass/fail, but rather a summary
  table or graph.}

\wss{Tests related to usability could include conducting a usability test and
  survey.  The survey will be in the Appendix.}

\wss{Static tests, review, inspections, and walkthroughs, will not follow the
  format for the tests given below.}

\wss{If you introduce static tests in your plan, you need to provide details.
  How will they be done?  In cases like code (or document) walkthroughs, who will
  be involved? Be specific.}

\subsubsection{Area of Testing1}

\paragraph{Title for Test}

\begin{enumerate}

  \item{test-id1\\}

        Type: Functional, Dynamic, Manual, Static etc.

        Initial State:

        Input/Condition:

        Output/Result:

        How test will be performed:

  \item{test-id2\\}

        Type: Functional, Dynamic, Manual, Static etc.

        Initial State:

        Input:

        Output:

        How test will be performed:

\end{enumerate}

\subsubsection{Area of Testing2}

...

\subsection{Traceability Between Test Cases and Requirements}

\wss{Provide a table that shows which test cases are supporting which
  requirements.}

\section{Unit Test Description}

\wss{This section should not be filled in until after the MIS (detailed design
  document) has been completed.}

\wss{Reference your MIS (detailed design document) and explain your overall
  philosophy for test case selection.}

\wss{To save space and time, it may be an option to provide less detail in this section.
  For the unit tests you can potentially layout your testing strategy here.  That is, you
  can explain how tests will be selected for each module.  For instance, your test building
  approach could be test cases for each access program, including one test for normal behaviour
  and as many tests as needed for edge cases.  Rather than create the details of the input
  and output here, you could point to the unit testing code.  For this to work, you code
  needs to be well-documented, with meaningful names for all of the tests.}

\subsection{Unit Testing Scope}

\wss{What modules are outside of the scope.  If there are modules that are
  developed by someone else, then you would say here if you aren't planning on
  verifying them.  There may also be modules that are part of your software, but
  have a lower priority for verification than others.  If this is the case,
  explain your rationale for the ranking of module importance.}

\subsection{Tests for Functional Requirements}

\wss{Most of the verification will be through automated unit testing.  If
  appropriate specific modules can be verified by a non-testing based
  technique.  That can also be documented in this section.}

\subsubsection{Module 1}

\wss{Include a blurb here to explain why the subsections below cover the module.
  References to the MIS would be good.  You will want tests from a black box
  perspective and from a white box perspective.  Explain to the reader how the
  tests were selected.}

\begin{enumerate}

  \item{test-id1\\}

        Type: \wss{Functional, Dynamic, Manual, Automatic, Static etc. Most will be
          automatic}

        Initial State:

        Input:

        Output: \wss{The expected result for the given inputs}

        Test Case Derivation: \wss{Justify the expected value given in the Output
          field}

        How test will be performed:

  \item{test-id2\\}

        Type: \wss{Functional, Dynamic, Manual, Automatic, Static etc. Most will be
          automatic}

        Initial State:

        Input:

        Output: \wss{The expected result for the given inputs}

        Test Case Derivation: \wss{Justify the expected value given in the Output
          field}

        How test will be performed:

  \item{...\\}

\end{enumerate}

\subsubsection{Module 2}

...

\subsection{Tests for Nonfunctional Requirements}

\wss{If there is a module that needs to be independently assessed for
  performance, those test cases can go here.  In some projects, planning for
  nonfunctional tests of units will not be that relevant.}

\wss{These tests may involve collecting performance data from previously
  mentioned functional tests.}

\subsubsection{Module ?}

\begin{enumerate}

  \item{test-id1\\}

        Type: \wss{Functional, Dynamic, Manual, Automatic, Static etc. Most will be
          automatic}

        Initial State:

        Input/Condition:

        Output/Result:

        How test will be performed:

  \item{test-id2\\}

        Type: Functional, Dynamic, Manual, Static etc.

        Initial State:

        Input:

        Output:

        How test will be performed:

\end{enumerate}

\subsubsection{Module ?}

...

\subsection{Traceability Between Test Cases and Modules}

\wss{Provide evidence that all of the modules have been considered.}

\bibliographystyle{plainnat}

\bibliography{../../refs/References}

\newpage

\section{Appendix}

This is where you can place additional information.

\subsection{Symbolic Parameters}

The definition of the test cases will call for SYMBOLIC\_CONSTANTS. Their
values are defined in this section for easy maintenance.

\subsection{Usability Survey Questions?}

\wss{This is a section that would be appropriate for some projects.}

\newpage{}
\section*{Appendix --- Reflection}

\wss{This section is not required for CAS 741}

The information in this section will be used to evaluate the team members on
the graduate attribute of Lifelong Learning.

\input{../Reflection.tex}

\begin{enumerate}
  \item What went well while writing this deliverable?
  \item What pain points did you experience during this deliverable, and how did you
        resolve them?
  \item What knowledge and skills will the team collectively need to acquire to
        successfully complete the verification and validation of your project? Examples
        of possible knowledge and skills include dynamic testing knowledge, static
        testing knowledge, specific tool usage, Valgrind etc. You should look to
        identify at least one item for each team member.
  \item For each of the knowledge areas and skills identified in the previous question,
        what are at least two approaches to acquiring the knowledge or mastering the
        skill? Of the identified approaches, which will each team member pursue, and
        why did they make this choice?
\end{enumerate}

\end{document}
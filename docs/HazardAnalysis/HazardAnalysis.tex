\documentclass{article}
\pdfinfoomitdate=1
\pdftrailerid{}
\usepackage{booktabs}
\usepackage{tabularx}
\usepackage{hyperref}
\usepackage{graphicx}
\usepackage{float}
\usepackage{array}
\usepackage{makecell}
\usepackage{ragged2e} % for RaggedRight in X columns
\usepackage{tabularray}
\usepackage{tabularx,booktabs,array,xcolor}
\usepackage{longtable}
\newcolumntype{L}[1]{>{\raggedright\arraybackslash}p{#1}}
\definecolor{mylight}{RGB}{255,255,180} % optional highlight

\newcolumntype{Y}{>{\RaggedRight\arraybackslash}X}
\newcolumntype{C}[1]{>{\centering\arraybackslash}p{#1}}

\hypersetup{
    colorlinks=true,       % false: boxed links; true: colored links
    linkcolor=red,          % color of internal links (change box color with linkbordercolor)
    citecolor=green,        % color of links to bibliography
    filecolor=magenta,      % color of file links
    urlcolor=cyan           % color of external links
}

\title{Hazard Analysis\\\progname}

\author{\authname}

\date{}

\input{../Comments}
%% Common Parts

\newcommand{\progname}{RoCam} % PUT YOUR PROGRAM NAME HERE
\newcommand{\authname}{Team \#3, SpaceY
  \\ Zifan Si
  \\ Jianqing Liu
  \\ Mike Chen
  \\ Xiaotian Lou} % AUTHOR NAMES                  

\usepackage{hyperref}
\hypersetup{colorlinks=true, linkcolor=blue, citecolor=blue, filecolor=blue,
  urlcolor=blue, unicode=false}
\urlstyle{same}

\usepackage{indentfirst}
\usepackage{graphicx}

\usepackage{titling}

\pretitle{\begin{center}\includegraphics[width=0.5\textwidth]{../../assets/logo/black.png}\\[0.75em]\LARGE}
    \posttitle{\par\end{center}}

\usepackage[letterpaper, portrait, margin=1in]{geometry}

\usepackage{placeins}
\usepackage{float}

\begin{document}

\maketitle
\thispagestyle{empty}

~\newpage

\begin{table}[hp]
  \caption{Revision History} \label{TblRevisionHistory}
  \begin{tabularx}{\textwidth}{llX}
    \toprule
    \textbf{Date} & \textbf{Developer(s)} & \textbf{Change}        \\
    \midrule
    Date1         & Shike Chen              & Initial draft \\
    Date2         & Shike Chen              & Detailed draft \\
    \bottomrule
  \end{tabularx}
\end{table}

~\newpage

\tableofcontents

~\newpage

\wss{You are free to modify this template.}

\section{Introduction}
Hazard is generally defined as a harm or potential harm or effect caused by a 
system which may lead to personal injury, property damage, environmental damage 
etc. As an integrated system with both hardware and software components, hazards 
may arise due to improper design, malfunction of the system, unexpected user
behavior, etc. Therefore, it is critical to identify potential hazards and 
mitigate them in the early stage of the project. 
In the case of the RoCam project, the potential hazards could cause physical 
injury of the user or malfunction of a larger system.

\section{Scope and Purpose of Hazard Analysis}

The hazard analysis will identify any direct or potential hazards caused by the 
system which includes hazardous materials, design, user safety, etc. 
The hazard analysis will also address the general hazardous effects of this system 
which includes hazards effects in the events of malfunction, misuse or environmental
damage. However, since the project is open source, any use case that is beyond the
scope of this project will not be included in this hazard analysis.

Such hazard analysis will not only used to identify potential hazards caused by the 
system, but also to minimize risks related in unforeseen cases. The purpose of this 
hazard analysis is to properly address the potential harm and risks caused by the 
product to the user and the external environment.

\section{System Boundaries and Components}

\subsection{Hardware Components}
\subsubsection{Motion control modules}
The motion control module consists of a custom-designed PCB board utilizing an STM32 
microcontroller. This module is responsible for controlling the movement of the camera
gimbal. It receives commands from our Computer Vision module and translates them into
precise motor movements in real time. 
\subsubsection{Camera gimbal}
The camera gimbal used in this project is a existing Camera Gimbal designed by the 
McMaster Rocketry team. Since our team is not responsible for the design of the gimbal, 
we can only estimate the potential hazards based on similar products in the market. 
\subsubsection{Jetson Orin Nano}
The Jetson Orin Nano is small, lightweight AI computer designed by NVIDIA. It will
be used to deploy our computer vision moodule and communicate directly with the motion
control module.
\subsubsection{Camera}
The camera is a off the shelf camera that is compatible with the Jetson Orin Nano. It 
is able to provide 4k video input in real time. 

\subsection{Software Components}
\subsubsection{Computer Vision Module}
Our computer vision module will be developed in Python. The base architecture will be 
torch and pandas. It will be accerlerated with TensorRT in Jetson to achieve real time
stable performance. 
\subsubsection{User Interface Module}
The user interface module will be developed using React. It will provide a web-based
interface for users to interact with the system, view camera feeds, and control the
camera gimbal.

\wss{Dividing the system into components will help you brainstorm the hazards.
  You shouldn't do a full design of the components, just get a feel for the major
  ones.  For projects that involve hardware, the components will typically include
  each individual piece of hardware.  If your software will have a database, or an
  important library, these are also potential components.}

\section{Critical Assumptions}

\subsection{Assumptions about the Operating Environment}
The system will not operating in any extreme environment (e.g., extreme temperature, 
acidic environments, etc.). The system will be operating within 10 to 25 degree Celsius.
The system will operate with a stable power supply. 
\subsection{Assumptions about the User}
Anyone above 14 years old who uses electronic and exposed to software interface on a
daily basis for over 2 years with good intent. 
\subsection{Assumptions about general of the shelf hardware components}
All off-the-shelf hardware components will function as advertised and
will be compatible with our system. This includes the camera, gimbal, Jetson Orin Nano
etc. 
\subsection{Assumptions about the software libraries}
All software libraries used will not have any unknown security vulnerabilities.

\wss{These assumptions that are made about the software or system.  You should
  minimize the number of assumptions that remove potential hazards.  For instance,
  you could assume a part will never fail, but it is generally better to include
  this potential failure mode.}


\newpage
\section{Failure Mode and Effect Analysis}
\begin{table}[htbp]
    \caption{FMEA Table}\label{TblFMEALegend}
  \scalebox{0.6}{
  \begin{tblr}{
    colspec = {p{2cm} p{4cm} p{7cm} p{7cm} p{7cm}},
    row{1} = {font=\bfseries},
    hlines
  }
    Design Function & Failure Mode & Effect & Cause of Failure & Recommended Action\\
    Gimbal turn as CV control & 
    \begin{enumerate}
      \item Gimbal does not move
      \item Gimbal snaps
      \item Gimbal does not turn properly
    \end{enumerate} & 
    \begin{enumerate}
      \item \begin{enumerate}
        \item Camera loses the rocket
        \item user unable to debug the rocket launch
      \end{enumerate}
      \item \begin{enumerate}
        \item Camera loses the rocket
        \item Camera injures user (minor cuts and bruises)
        \item Camera and gimbal damages other property
        \item Camera, gimbal, motion control module, Jetson Orin Nano damages
      \end{enumerate}
      \item \begin{enumerate}
        \item Camera overshoots or loses the rocket
        \item Camera unable to track the rocket smoothly with high quality image
        \end{enumerate}
    \end{enumerate}
    & 
    \begin{enumerate}
        \item \begin{enumerate}
            \item failed in communication between the motion control module and the gimbal
            \item failed in communication then CV module and the motion control module
            \item firmware bug in the motion control module or compatibility issue
            \item fried electric components in the motion control module or gimbal
        \end{enumerate}
        \item \begin{enumerate}
            \item improper instruction from the motion control module
            \item rocket angular velocity to the gimbal excesses the design limit
        \end{enumerate}
        \item \begin{enumerate}
            \item improper instruction from the motion control module
            \item CV module fails to provide proper instruction to the motion control module
            \item CV module lost the rocket
            \item damaged components in the motion control module or gimbal
        \end{enumerate}
    \end{enumerate}
    &
    \begin{enumerate}
        \item \begin{enumerate}
            \item careful handling in transportation and assembly to avoid damage to the 
            components 
            \item Integration test before launch to ensure proper communication between 
            all modules
        \end{enumerate}
        \item \begin{enumerate}
            \item Set threshold of the maximum angular velocity to avoid gimbal snapping 
            \item Set threshold of the maximum angle to avoid gimbal snapping
            \item Calculate the expected angular velocity of the camera system and the (SR1)
            target to ensure
            they are within the design limit of the system
            \item safety goggles should be worn when operating the system (SR2)
        \end{enumerate}
        \item \begin{enumerate}
            \item design CV module to be robust in different environment 
            \item design motion control module and CV module speed to have self-correcting (SR3)
            mechanism
            \item design multiple tracking algorithms to avoid losing of the rocket

        \end{enumerate}
    \end{enumerate}
    \\
    User Interface control monitor tracking &
    \begin{enumerate}
      \item UI does not load
      \item UI loads but does not display camera feed
      \item User unable to understand and operate the UI
    \end{enumerate}
    &
    \begin{enumerate}
        \item \begin{enumerate}
            \item User unable to control the gimbal and camera
            \item Failed to monitor the rocket launch
            \item Failed to track the rocket launch
            \item User frustration
        \end{enumerate}
        \item \begin{enumerate}
                \item User unable to monitor rocket launch in real time
                \item User unable to track the rocket launch in real time
                \item User unable to select desired tracking mode and target
        \end{enumerate}
        \item \begin{enumerate}
                \item User unable to control the gimbal and camera
                \item Failed to monitor the rocket launch
                \item Failed to track the rocket launch
                \item User frustration
            \end{enumerate}
    \end{enumerate}
    &
    \begin{enumerate}
        \item \begin{enumerate}
            \item Bug in the UI software
            \item Issue in CD pipeline
            \item Browser compatibility issue
            \item OS environment issue
        \end{enumerate}
        \item \begin{enumerate}
                \item Issue in API between UI and CV module
                \item Camera or Jetson Orin Nano hardware issue
        \end{enumerate}
        \item \begin{enumerate}
                \item Poorly designed UI
                \item Lack of user manual and documentation
        \end{enumerate}
    \end{enumerate}
    &
    \begin{enumerate}
        \item \begin{enumerate}
            \item Thorough testing in different browsers and OS environment
            \item Proper CI/CD pipeline to ensure smooth deployment
            \item Provide user manual and documentation for installation and 
            troubleshooting
        \end{enumerate}
        \item \begin{enumerate}
                \item Thorough testing of the API between UI and CV module
                \item Provide user manual and documentation for installation and 
                troubleshooting
                \item Careful handling in transportation and assembly to avoid 
                damage to the components
            \end{enumerate}
        \item \begin{enumerate}
                \item User testing and feedback to improve UI design
                \item Provide user manual and documentation for installation and 
                troubleshooting
                \item Provide user with access to an AI Agent that read the user 
                manual and documentation
            \end{enumerate}
    \end{enumerate}
    
  \end{tblr}
  }
\end{table}

\wss{Include your FMEA table here. This is the most important part of this document.}
\wss{The safety requirements in the table do not have to have the prefix SR.
  The most important thing is to show traceability to your SRS. You might trace to
  requirements you have already written, or you might need to add new
  requirements.}
\wss{If no safety requirement can be devised, other mitigation strategies can be
  entered in the table, including strategies involving providing additional
  documentation, and/or test cases.}

\newpage
\section{Safety and Security Requirements}

\begin{enumerate}
    \item SR1: The system will be placed in a distance which the expected angular velocity 
    will not exceed the design limit of the gimbal.
    \item SR2: all operators must wear safety goggles when operating the system.
    \item SR3: The system will have a self-correcting mechanism to ensure the camera can 
    track the rocket smoothly in real time.
\end{enumerate}

\section{Roadmap}

\subsection{Capstone Implementation Plan (Sprints \& Milestones)}
\begin{table}[H]
\caption{Capstone Roadmap (8 sprints)}\label{tab:capstone-roadmap}
\renewcommand{\arraystretch}{1.15}
\begin{tabularx}{\textwidth}{C{1.2cm} L{4.2cm} Y Y Y}
\toprule
\textbf{Sprint} & \textbf{Primary Goals} & \textbf{Deliverables} & \textbf{Safety Focus (Traceability)} & \textbf{Verification / Evidence} \\
\midrule
1 &
System bring-up: Jetson Orin Nano, camera, repo CI. Define interfaces (UI \(\leftrightarrow\) CV \(\leftrightarrow\) Motion). &
Env setup scripts; interface spec; smoke tests. &
Baseline hazard review; confirm assumptions (power, temp). &
Checklist, boot logs, interface mock tests. \\
\midrule
2 &
Motion control MVP: gimbal command path; encoder/limits readout. &
Motor driver config; limit-switch handling; teleop tool. &
SR1 (angular velocity/angle limits) draft. &
Bench test: limit cut-offs; plotted step responses. \\
\midrule
3 &
CV tracking MVP on sample footage (offline \(\rightarrow\) near-real-time). &
Tracker module; target reacquisition logic. &
SR3 draft (self-correction, loss recovery). &
Replay tests; failure-injection (occlusion) logs. \\
\midrule
4 &
UI prototype: live feed, mode select, status \& alarms. &
Web UI with HUD; error banners; safe-state button. &
FMEA controls for “UI not loading/displaying”. &
Cross-browser tests; watchdog for API health. \\
\midrule
5 &
Closed-loop integration: CV \(\rightarrow\) Motion with rate limiting. &
End-to-end demo; rate limiter \& soft stops. &
SR1 formalized (hard/soft limits); map FMEA “gimbal snap”. &
Angular-rate sweep; saturation tests; limit breach proofs. \\
\midrule
6 &
Robustness: fallback trackers, loss detection, auto-recenter. &
Multi-strategy tracker; loss timers; homing routine. &
SR3 formalized (self-correction). &
Scenario matrix (lighting/occlusion); recovery time stats. \\
\midrule
7 &
Safety hardening: interlocks, E-stop, checklists, SOP. &
E-stop wiring/command; pre-launch checklist; SOP v1. &
SR2 (PPE), SR1 (interlocks), UI hazard cues. &
Dry-run with operators; checklist sign-offs; video evidence. \\
\midrule
8 &
Verification \& docs: finalize FMEA, traceability, test report. &
FMEA v2; SRS trace matrix; safety test report; demo video. &
All SRs: SR1–SR3 complete and traced. &
Pass/fail summary; residual-risk log; release tag. \\
\bottomrule
\end{tabularx}
\end{table}

\subsection{Milestones}
\begin{enumerate}
  \item M1 (End Sprint 2): Motion MVP with enforceable soft limits (SR1 draft).
  \item M2 (End Sprint 4): UI prototype with health/status and safe-state control.
  \item M3 (End Sprint 6): Closed-loop tracking with recovery/self-correction (SR3).
  \item M4 (End Sprint 8): Safety package complete (FMEA v2, SR1–SR3 verified, docs).
\end{enumerate}

\subsection{Traceability Plan}
\begin{itemize}
  \item FMEA \(\rightarrow\) SR mapping: “Gimbal snaps/overshoots” \(\rightarrow\) SR1; “Tracking lost” \(\rightarrow\) SR3; “UI unavailable/unclear” \(\rightarrow\) UI alarms, SOP, tests.
  \item Each sprint closes with: (a) updated FMEA row(s), (b) test artifacts linked to SR IDs, (c) residual risk review.
\end{itemize}

\subsection{Acceptance Criteria}
\begin{itemize}
  \item \textbf{SR1}: Max angular velocity/angle never exceeded in any commanded or fault path; evidence: logs + plotted limits across stress tests.
  \item \textbf{SR2}: Operators use PPE; SOP and checklist enforced; evidence: signed checklist + demo footage.
  \item \textbf{SR3}: From target-loss events, system recovers or safely recenters within \(T_{\max}\) seconds in \(\geq 95\%\) of trials.
\end{itemize}

\subsection{Post-Capstone Backlog (Future Work)}
\begin{enumerate}
  \item Hardware redundancy (limit sensors, power rails), formal E-stop certification.
  \item Model optimization (TensorRT INT8), thermal profiling, enclosure/EMI review.
  \item Expanded UI: guided wizards, operator training mode, automated pre-flight.
  \item Data logging for incident analysis; automated regression test farm.
  \item Security hardening: signed releases, SBOM, dependency scanning.
\end{enumerate}
\newpage{}

\section*{Appendix --- Reflection}

\wss{Not required for CAS 741}

\input{../Reflection.tex}

\begin{enumerate}
  \item What went well while writing this deliverable?
  \item What pain points did you experience during this deliverable, and how did you
        resolve them?
  \item Which of your listed risks had your team thought of before this deliverable,
        and which did you think of while doing this deliverable? For the latter ones
        (ones you thought of while doing the Hazard Analysis), how did they come about?
  \item Other than the risk of physical harm (some projects may not have any
        appreciable risks of this form), list at least 2 other types of risk in
        software products. Why are they important to consider?
\end{enumerate}

\end{document}
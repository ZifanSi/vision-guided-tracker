\documentclass[11pt]{article}
\pdfinfoomitdate=1
\pdftrailerid{}

% Preamble
\usepackage{array}                % <-- required for \newcolumntype and \arraybackslash
\setlength{\tabcolsep}{3pt}       % shrink inter-column padding (default is 6pt)
\renewcommand{\arraystretch}{1.15}

\usepackage{booktabs}
\usepackage{tabularx}
\usepackage{hyperref}
\usepackage{longtable}
\usepackage{ragged2e}
\usepackage[hidelinks]{hyperref}
\renewcommand{\arraystretch}{1.2}
\usepackage[margin=1in]{geometry}
\newcolumntype{R}[1]{>{\RaggedRight\arraybackslash}p{#1}}


\hypersetup{
    colorlinks=true,       % false: boxed links; true: colored links
    linkcolor=red,          % color of internal links (change box color with linkbordercolor)
    citecolor=green,        % color of links to bibliography
    filecolor=magenta,      % color of file links
    urlcolor=cyan           % color of external links
}

\title{Hazard Analysis\\\progname}

\author{\authname}

\date{}

\input{../Comments}
%% Common Parts

\newcommand{\progname}{RoCam} % PUT YOUR PROGRAM NAME HERE
\newcommand{\authname}{Team \#3, SpaceY
  \\ Zifan Si
  \\ Jianqing Liu
  \\ Mike Chen
  \\ Xiaotian Lou} % AUTHOR NAMES                  

\usepackage{hyperref}
\hypersetup{colorlinks=true, linkcolor=blue, citecolor=blue, filecolor=blue,
  urlcolor=blue, unicode=false}
\urlstyle{same}

\usepackage{indentfirst}
\usepackage{graphicx}

\usepackage{titling}

\pretitle{\begin{center}\includegraphics[width=0.5\textwidth]{../../assets/logo/black.png}\\[0.75em]\LARGE}
    \posttitle{\par\end{center}}

\usepackage[letterpaper, portrait, margin=1in]{geometry}

\usepackage{placeins}
\usepackage{float}

\begin{document}

\maketitle
\thispagestyle{empty}

~\newpage

\begin{table}[hp]
  \caption{Revision History} \label{TblRevisionHistory}
  \begin{tabularx}{\textwidth}{llX}
    \toprule
    \textbf{Date} & \textbf{Developer(s)} & \textbf{Change}        \\
    \midrule
    Date1         & Shike Chen              & Initial draft \\
    Date2         & Shike Chen              & Detailed draft \\
    \bottomrule
  \end{tabularx}
\end{table}

~\newpage

\tableofcontents

~\newpage

\wss{You are free to modify this template.}

\section{Introduction}
Hazard is generally defined as a harm or potential harm or effect caused by a 
system which may lead to personal injury, property damage, environmental damage 
etc. As an integrated system with both hardware and software components, hazards 
may arise due to improper design, malfunction of the system, unexpected user
behavior, etc. Therefore, it is critical to identify potential hazards and 
mitigate them in the early stage of the project. 
In the case of the RoCam project, the potential hazards could cause physical 
injury of the user or malfunction of a larger system.

\section{Scope and Purpose of Hazard Analysis}

The hazard analysis will identify any direct or potential hazards caused by the 
system which includes hazardous materials, design, user safety, etc. 
The hazard analysis will also address the general hazardous effects of this system 
which includes hazards effects in the events of malfunction, misuse or environmental
damage. However, since the project is open source, any use case that is beyond the
scope of this project will not be included in this hazard analysis.

Such hazard analysis will not only used to identify potential hazards caused by the 
system, but also to minimize risks related in unforeseen cases. The purpose of this 
hazard analysis is to properly address the potential harm and risks caused by the 
product to the user and the external environment.

\section{System Boundaries and Components}

\subsection{Hardware Components}
\subsubsection{Motion control modules}
The motion control module consists of a custom-designed PCB board utilizing an STM32 
microcontroller. This module is responsible for controlling the movement of the camera
gimbal. It receives commands from our Computer Vision module and translates them into
precise motor movements in real time. 
\subsubsection{Camera gimbal}
The camera gimbal used in this project is a existing Camera Gimbal designed by the 
McMaster Rocketry team. Since our team is not responsible for the design of the gimbal, 
we can only estimate the potential hazards based on similar products in the market. 
\subsubsection{Jetson Orin Nano}
The Jetson Orin Nano is small, lightweight AI computer designed by NVIDIA. It will
be used to deploy our computer vision moodule and communicate directly with the motion
control module.
\subsubsection{camera}
The camera is a off the shelf camera that is compatible with the Jetson Orin Nano. It 
is able to provide 4k video input in real time. 

\subsection{Software Components}
\subsubsection{Computer Vision Module}
Our computer vision module will be developed in Python. The base architecture will be 
torch and pandas. It will be accerlerated with TensorRT in Jetson to achieve real time
stable performance. 
\subsubsection{User Interface Module}
The user interface module will be developed using React. It will provide a web-based
interface for users to interact with the system, view camera feeds, and control the
camera gimbal.

\wss{Dividing the system into components will help you brainstorm the hazards.
  You shouldn't do a full design of the components, just get a feel for the major
  ones.  For projects that involve hardware, the components will typically include
  each individual piece of hardware.  If your software will have a database, or an
  important library, these are also potential components.}

\section{Critical Assumptions}

\subsection{Assumptions about the Operating Environment}
The system will not operating in any extreme environment (e.g., extreme temperature, 
acidic environments, etc.). The system will be operating within 10 to 25 degree Celsius.
The system will operate with a stable power supply. 
\subsection{Assumptions about the User}
Anyone above 14 years old who uses electronic and exposed to software interface on a
daily basis for over 2 years with good intent. 
\subsection{Assumptions about general of the shelf hardware components}
All off-the-shelf hardware components will function as advertised and
will be compatible with our system. This includes the camera, gimbal, Jetson Orin Nano
etc. 
\subsection{Assumptions about the software libraries}
All software libraries used will not have any unknown security vulnerabilities.

\wss{These assumptions that are made about the software or system.  You should
  minimize the number of assumptions that remove potential hazards.  For instance,
  you could assume a part will never fail, but it is generally better to include
  this potential failure mode.}

\section{Failure Mode and Effect Analysis}

\begin{longtable}{|R{2.2cm}|R{3.2cm}|R{3.0cm}|R{3.0cm}|R{3.2cm}|R{1.6cm}|R{1.2cm}|}
  \caption{FMEA} \\
  \hline
  \multicolumn{7}{|c|}{\textbf{Failure Mode and Effects Analysis}}\\
  \multicolumn{7}{|l|}{\textbf{System}: Audio to sheet music generator}\\
  \multicolumn{7}{|l|}{\textbf{Phase/Mode}: System Requirements}\\
  \hline
  \textbf{Design Function} & \textbf{Failure Modes} & \textbf{Effects of Failure} &
  \textbf{Causes of Failure} & \textbf{Recommended Action} & \textbf{SR} & \textbf{Ref.}\\
  \hline
  \endfirsthead

  \hline
  \multicolumn{7}{|l|}{\emph{Continued from previous page}}\\
  \hline
  \textbf{Design Function} & \textbf{Failure Modes} & \textbf{Effects of Failure} &
  \textbf{Causes of Failure} & \textbf{Recommended Action} & \textbf{SR} & \textbf{Ref.}\\
  \hline
  \endhead

  \hline
  \multicolumn{7}{|r|}{\emph{Continued on next page}}\\
  \hline
  \endfoot

  \hline
  \endlastfoot

  % ===== Rows (edit/extend as needed) =====
  Generate sheet music
  & Sheet music flashes on the screen
  & Potential trigger for users with epilepsy
  & Incorrect rendering or scrolling through the sheet music
  & Limit or disable visual flicker; provide user setting
  & \hyperref[PR-SC1]{PR-SC1}
  & \\ \cline{2-7}

  & A bad microphone causes notes to be off due to pitch drift
  & Embarrassment for the user
  & Microphone exhibits pitch drift (mechanical issue)
  & Troubleshooting steps; suggest higher-quality microphone
  & N/A
  & \\ \hline

  Process audio
  & Audio improperly recorded
  & Loss of musical performance
  & Microphone error
  & Notify user about microphone issue; retry guidance
  & N/A
  & \\ \cline{4-7}

  &  &  & Signal processing error
  & Notify user a processing error occurred
  & \href{https://github.com/emilyperica/ScoreGen/blob/main/docs/SRS/SRS.pdf}{FR-SP4}
  & \\ \hline

  File functions
  & Files outside scope of the application are deleted
  & Possible system failure if critical files removed
  & Improper file I/O actions
  & Restrict permissions; sandbox; confirmations
  & \hyperref[S-P1]{S-P1}
  & \\ \cline{2-7}

  & Files outside scope of the application are modified
  & System instability / data loss
  & Improper file I/O actions
  & Same mitigations as above
  & \hyperref[S-P1]{S-P1}
  & \\ \hline

  Output audio
  & Overly loud audio emitted
  & Hearing damage / startle
  & Miscalibrated volume path
  & Output limiter; safe defaults; pre-playback check
  & N/A
  & \\ \cline{3-7}

  &  & User is surprised and has a health incident
  & Miscalibrated volume path
  & Same as above
  & N/A
  & \\ \hline

  Application runtime
  & App crashes unexpectedly
  & User loses progress
  & Power outage, OOM error, overheating, malware
  & Autosave (with consent); crash recovery; logging
  & \href{https://github.com/emilyperica/ScoreGen/blob/main/docs/SRS/SRS.pdf}{PR-RFT3}
  & \\ \hline

\end{longtable}

\wss{Include your FMEA table here. This is the most important part of this document.}
\wss{The safety requirements in the table do not have to have the prefix SR.
  The most important thing is to show traceability to your SRS. You might trace to
  requirements you have already written, or you might need to add new
  requirements.}
\wss{If no safety requirement can be devised, other mitigation strategies can be
  entered in the table, including strategies involving providing additional
  documentation, and/or test cases.}

\section{Safety and Security Requirements}

\wss{Newly discovered requirements.  These should also be added to the SRS.  (A
  rationale design process how and why to fake it.)}

\section{Roadmap}

\wss{Which safety requirements will be implemented as part of the capstone timeline?
  Which requirements will be implemented in the future?}

\newpage{}

\section*{Appendix --- Reflection}

\wss{Not required for CAS 741}

\input{../Reflection.tex}

\begin{enumerate}
  \item What went well while writing this deliverable?
  \item What pain points did you experience during this deliverable, and how did you
        resolve them?
  \item Which of your listed risks had your team thought of before this deliverable,
        and which did you think of while doing this deliverable? For the latter ones
        (ones you thought of while doing the Hazard Analysis), how did they come about?
  \item Other than the risk of physical harm (some projects may not have any
        appreciable risks of this form), list at least 2 other types of risk in
        software products. Why are they important to consider?
\end{enumerate}

\end{document}
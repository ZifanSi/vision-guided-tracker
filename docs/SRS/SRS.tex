% THIS DOCUMENT IS FOLLOWS THE VOLERE TEMPLATE BY Suzanne Robertson and James Robertson
% ONLY THE SECTION HEADINGS ARE PROVIDED
%
% Initial draft from https://github.com/Dieblich/volere
%
% Risks are removed because they are covered by the Hazard Analysis
\documentclass[12pt]{article}
\pdfinfoomitdate=1
\pdftrailerid{}

\usepackage{booktabs}
\usepackage{tabularx}
\usepackage{hyperref}
\usepackage{enumitem}
\hypersetup{
    bookmarks=true,         % show bookmarks bar?
    colorlinks=true,        % false: boxed links; true: colored links
    linkcolor=red,          % color of internal links (change box color with linkbordercolor)
    citecolor=green,        % color of links to bibliography
    filecolor=magenta,      % color of file links
    urlcolor=cyan           % color of external links
}

\newcommand{\lips}{\textit{Insert your content here.}}

\input{../Comments}
%% Common Parts

\newcommand{\progname}{RoCam} % PUT YOUR PROGRAM NAME HERE
\newcommand{\authname}{Team \#3, SpaceY
  \\ Zifan Si
  \\ Jianqing Liu
  \\ Mike Chen
  \\ Xiaotian Lou} % AUTHOR NAMES                  

\usepackage{hyperref}
\hypersetup{colorlinks=true, linkcolor=blue, citecolor=blue, filecolor=blue,
  urlcolor=blue, unicode=false}
\urlstyle{same}

\usepackage{indentfirst}
\usepackage{graphicx}

\usepackage{titling}

\pretitle{\begin{center}\includegraphics[width=0.5\textwidth]{../../assets/logo/black.png}\\[0.75em]\LARGE}
    \posttitle{\par\end{center}}

\usepackage[letterpaper, portrait, margin=1in]{geometry}

\usepackage{placeins}
\usepackage{float}

\begin{document}

\title{Software Requirements Specification for \progname: High Performance Vision-Guided Rocket Tracker}
\author{\authname}
\date{\today}

\maketitle

~\newpage

\tableofcontents

~\newpage

\section*{Revision History}

\begin{tabularx}{\textwidth}{p{3cm}p{2cm}X}
  \toprule {\textbf{Date}} & {\textbf{Version}} & {\textbf{Notes}} \\
  \midrule
  Date 1                   & 1.0                & Notes            \\
  Date 2                   & 1.1                & Notes            \\
  \bottomrule
\end{tabularx}

~\\

~\newpage
\section{Purpose of the Project}
\subsection{User Business}

Student and amateur rocketry teams, launch organizers, and technical judges
need reliable, high-fidelity visual evidence of rocket flights to analyze
staging behavior, parachute deployment, and flight anomalies that occur at
extreme speeds and altitudes. Manual camera operation is impractical in these
conditions, and existing tools (manual PTZ/GPS-centric systems) often fail to
maintain visual lock on small, fast-moving vehicles. This project addresses
this gap with an autonomous, vision-based tracking camera designed to lock onto
and follow rockets from launch through landing, providing real-time and
recorded footage for post-flight analysis and event operations.

\subsection{Goals of the Project}

\textbf{Purpose:}

Develop a software+hardware stack (the system) that is able to command a gimbal
with a camera mounted on it to track and record model rocket launches.

\textbf{Advantage:}

The system provides insights into the flight performance by having high quality
flight footage.

\textbf{Measure:}

Deploy the system at small-scale model rocket launch site with apogee less than
200 meters. The system should provide higher quality flight footage than the
manual camera operation.

\section{Stakeholders}

\subsection{Client}

The client of this project is McMaster Rocketry Team. they are the primary end
users who will deploy the system during launches.

\subsection{Customer}

Apart from our client, we also identified the following customers that may be
interested in using the project:

\begin{enumerate}
  \item \textbf{Model Rocketry Event Organizers}: Benefit from
        reliable tracking for live streaming of rocket flights.
  \item \textbf{Model Rocketry Safety Officers}: Benefit from
        live monitoring of parachute deployment and landing location
        of the rockets.
  \item \textbf{Aerospace Engineers and Researchers}: Benefit from
        high-quality flight footage to validate models, improve rocket
        designs, and support experimental research.
\end{enumerate}

\subsection{Other Stakeholders}

\begin{enumerate}
  \item \textbf{Dr. Shahin Sirouspour (Supervisor)}: Provides
        technical guidance, project oversight, and mentorship. Ensures
        the project aligns with academic standards, engineering best
        practices, and capstone deliverable expectations.
\end{enumerate}

\subsection{Hands-On Users of the Project}

Here are the users that will be interacting with the product directly.

\textbf{Role: Camera Operator}

The camera operator controls the system through the UI, starts/stops
recordings, manages tracking states, and monitors preview. The camera operator
is assumed to have a decent level of familiarity with technology, and have been
trained on the use of the product.

\textbf{Role: Live Stream Technician}

The live stream technician manages the video feed from the system to the live
streaming equipments. The live stream technician is assumed to have a high
level of familiarity with live streaming software and hardware.

\textbf{Role: Flight Performance Analyst}

The flight performance analyst is a member of the rocketry team that
reconstructs flight profiles from sensor data and camera footage to understand
how the rocket behaves during ascent and descent. The flight performance
analyst is assumed to have a high level of familiarity with rocket flight
dynamics and software used for analysis.

\textbf{Role: Live Stream / Recordings Viewers}

People who view the live stream or recorded videos. They can be anyone from the
general public that are interested in the launch, or the team members
themselves. They are assumed to have a basic level of familiarity with
technology.

\subsection{Personas}

\subsubsection*{Alexis Andrade}

\textbf{Age}: 28

\textbf{Role}: Camera Operator

Alexis works as an industrial automation technician in Hamilton, Ontario, where
she designs and maintains robotic assembly systems for a manufacturing firm.
Outside of work, her real passion lies in high-speed imaging and experimental
photography — a hobby that naturally led her to volunteer as a staff camera
operator at university rocketry competitions. She loves the mix of engineering
precision and adrenaline that comes with capturing a successful rocket launch.

She lives with her partner and their energetic border collie, Pixel. Alexis
spends her weekends hiking the Bruce Trail, refurbishing vintage lenses, or
volunteering at local maker fairs. Her favorite food is falafel, and she always
packs a thermos of strong coffee for early launch mornings. She listens to
synthwave and retro electronic music on long drives to test sites, finding it
helps her stay calm and focused.

Alexis is confident with technology but prefers systems that are reliable and
straightforward. She values hands-on work, hates clunky software interfaces,
and believes in simplicity over flashiness. “If I need a manual,” she likes to
say, “the interface is wrong.” Her cool temperament and technical precision
make her a trusted member of the launch operations crew.

\subsubsection*{Brandon Brooks}

\textbf{Age}: 23

\textbf{Role}: Flight Performance Analyst

Brandon is in his final year of mechanical engineering at a Canadian university
and serves as a flight performance analyst for his rocketry team. His specialty
is reconstructing flight profiles from sensor data and camera footage to
understand how the rocket behaves during ascent and descent. He joined the team
for the engineering challenge but stayed because he loves the mix of theory,
experimentation, and teamwork.

Originally from Calgary, Brandon now lives in a small off-campus apartment
filled with model rockets, whiteboards, and a curious cat named Fourier. He's
known for his patience and meticulous documentation — traits that make him the
go-to person for post-launch analysis. When he's not studying or coding MATLAB
scripts, he's cooking ramen, taking photos, or biking along Lake Ontario.

Brandon listens to ambient and post-rock while he works, loves strong coffee,
and dislikes chaotic testing days where data logging gets overlooked. His
attitude toward technology is analytical but practical: he values transparency,
well-documented systems, and data that “speaks for itself.” For him, every
successful flight is a story written in numbers and motion.

\subsubsection*{Christopher Chou}

\textbf{Age}: 59

\textbf{Role}: Event Organizer / Live Stream Technician

Christopher is a retired broadcast engineer who spent three decades working in
television production for CBC Toronto. After retirement, he moved to Burlington
and found a new passion in volunteer work, helping organize live streams and
event logistics for student rocketry competitions. For him, it's the perfect
blend of his lifelong love of broadcasting and his fascination with aerospace
technology.

He's married with two grown children who live abroad, and he often jokes that
the rocketry community has become his “third family.” Christopher loves cooking
Cantonese comfort food — especially steamed dumplings and congee — and enjoys
jazz and classic rock in equal measure. He spends his spare time restoring old
audio equipment, gardening, and taking long weekend drives along the Niagara
Escarpment.

Christopher's approach to technology is deeply pragmatic: he respects
automation but insists that every system should have a manual override. He
likes tidy wiring, clear documentation, and software that doesn't crash
midstream. He dislikes unnecessary complexity and “updates that fix what wasn't
broken.” To younger team members, he's both a mentor and the steady hand who
keeps the broadcast running when chaos hits.
\subsection{Priorities Assigned to Users}

\textbf{Key Users}: Camera Operator, Live Stream Technician

\textbf{Secondary User}: Flight Performance Analyst, Live Stream / Recordings Viewers, Event Organizer

\subsection{User Participation}

Through out the development process, the development team will continue to
iterate on the requirements based on feedback from these users:

\begin{enumerate}
  \item \textbf{Camera Operator}: At least three field tests will be conducted
        to gather feedback from the camera operator. Each field test will last
        half a day to a day.
  \item \textbf{Live Stream Technician}: At least two meetings will be conducted
        to gather feedback from the live stream technician.
\end{enumerate}

\subsection{Maintenance Users and Service Technicians}

Due to the nature of this project as a capstone requirement, there are
currently no expected maintenance users.

\section{Mandated Constraints}
\subsection{Solution Constraints}

\begin{itemize}[leftmargin=*]
  \item[SC-1] \emph{The system must use computer vision for rocket detection and
          tracking}\\[2mm]
        \textbf{Rationale:} There is already a camera connected to the system for recording, using computer vision for detection and tracking is the most efficient way to do it. \\
        \textbf{Fit Criterion:} The system should be able to detect and track a rocket based only on the camera feed.
\end{itemize}

\subsection{Implementation Environment of the Current System}

There are no constrains on the hardware platform the system need to use.

\subsection{Partner or Collaborative Applications}

The system is expected to be used with a gimbal with a camera mounted on it.
The system will command the gimbal to track and record model rocket launches.

\subsection{Off-the-Shelf Software}

There are no constrains on the off-the-shelf software the system need to use.

\subsection{Anticipated Workplace Environment}

The system should be designed to operate in a large-scale outdoor launch
environment, for example, the Launch Canada launch site in Timmins, Ontario, Or
Spaceport America in New Mexico.

The tracking camera will be operated at outdoor launch sites, typically in wide
open fields or designated rocket ranges. Launches occur only under clear, sunny
conditions with few or no clouds.

Additionally, users often wear hearing protection devices to mitigate high
acoustic noise from tools, filling tanks, and generators.

Launch sites are typically in a remote area, with limited or no internet
access.

\subsection{Schedule Constraints}

\begin{itemize}[leftmargin=*]
  \item[SHC-1] \emph{The project shall be completed by April 2026, with interim
          deadlines for key milestones such as Proof of Concept (November 2025) and the
          final demonstration (March 2026).}\\[2mm]
        \textbf{Rationale:} These deadlines are based on the academic timeline and the expectations of the capstone course.\\
        \textbf{Fit Criterion:} All project components must be completed and fully functional by the final demonstration in March 2026.
\end{itemize}

\subsection{Budget Constraints}

\begin{itemize}[leftmargin=*]
  \item[BC-1] \emph{The project shall not exceed the budget of 500CAD.}\\[2mm]
        \textbf{Rationale:} Limited by the capstone project requirement.\\
        \textbf{Fit Criterion:} The sum of the costs of the hardware components required for the project must be less than 500CAD.
\end{itemize}

\subsection{Enterprise Constraints}

\begin{itemize}[leftmargin=*]
  \item[EC-1] \emph{The product shall be built to comply with the standards of McMaster
          University's capstone project requirements and academic integrity policies.}\\[2mm]
        \textbf{Rationale:} The project is part of the university's curriculum and must adhere to its standards.\\
        \textbf{Fit Criterion:} The product must meet the requirements specified by the course syllabus and project advisor.
\end{itemize}

\section{Naming Conventions and Terminology}
\subsection{Glossary of All Terms, Including Acronyms, Used by Stakeholders
  involved in the Project}

\begin{enumerate}
  \item \textbf{The System}: The software+hardware stack developed for this project that is able to command a gimbal with camera mounted on it to track and record a rocket flight. (the gimbal and the camera are not a part of the system)
  \item \textbf{Gimbal}: A gimbal is a motorized device that rotates based on the commands sent to it.
  \item \textbf{Rocket}: Refers to a model rocket.
  \item \textbf{Ascent}: The stage in a model rocket flight where the rocket is ascending.
  \item \textbf{Descent}: The stage in a model rocket flight where the rocket has reached its apogee and is descending.
\end{enumerate}

\section{Relevant Facts And Assumptions}
\subsection{Relevant Facts}

None.

\subsection{Business Rules}

None.

\subsection{Assumptions}

\begin{itemize}[leftmargin=*]
  \item[AS-1] The system will not be connected to the public internet; thus, no
        authentication is required.
  \item[AS-2] The system will only be physically accessed by authorized personnel.
  \item[AS-3] Only one rocket will be flying at a time.
  \item[AS-4] The gimbal connected to the system has adequate performance to track the
        rocket (e.g., it can turn fast enough and has enough range of motion).
  \item[AS-5] 110V AC power is available at the launch site.
  \item[AS-6] It will always be sunny at the launch site, because the launch would be
        cancelled in case of bad weather.
\end{itemize}

\section{The Scope of the Work}
\subsection{The Current Situation}

Currently, the process of recording and analyzing rocket flights relies on
manual camera operators positioned at safe distances from the launch pad. These
operators attempt to visually follow the rocket during ascent and descent using
consumer or semi-professional video cameras mounted on tripods or pan-tilt
systems.

This approach suffers from several limitations:

\textbf{Accuracy:} Human reaction time and limited field of view make it nearly
impossible to keep fast-moving rockets (often exceeding Mach speeds) centered
in frame. The footage is frequently lost during liftoff or staging events,
leaving gaps in post-flight analysis.

\textbf{Safety Constraints:} Because a human operator must be physically present, camera
placement is restricted to safe zones, which may not provide optimal viewing
angles. Cameras cannot always be positioned where the best line-of-sight
exists.

\textbf{Operational Overhead:} Each launch requires trained operators, setup time, and
coordination with the launch control team. Fatigue, stress, and environmental
conditions (sun glare, wind, dust) further degrade performance.

\textbf{Data Quality:} The video produced by manual operators is often shaky,
inconsistently framed, and lacking synchronization with other flight data
(e.g., telemetry or gimbal angle). This reduces its value for engineering
analysis and competition reporting.

\subsection{The Context of the Work}

\hyperref[img:context-of-work]{Figure 1} identifies the scope of the
investigation necessary in order to discover the requirements related to the
work of tracking rockets. The model shows the detailed area of investigation
and how it connects to the adjacent systems.

\begin{figure}[H]
  \centering
  \includegraphics[width=\textwidth,height=\textheight,keepaspectratio]{../Images/context_of_work.png}
  \caption{Context of Work Diagram}
  \label{img:context-of-work}
\end{figure}

\subsection{Work Partitioning}

The \hyperref[tab:work-part]{Table 1} identifies all business events happening
in the real world that affect the work of tracking rockets.

\begin{table}[H]
  \centering
  \setlength\extrarowheight{5mm}
  \begin{tabularx}{\textwidth}{cp{1.5in}X}
    \toprule \textbf{No.} & \textbf{Event Name}                                  &
    \textbf{Input / Output}                                                        \\
    \midrule
    BUC 1                 & Output live video                                    &
    (in) Video feed from camera \newline
    (out) Transmitting the video to the live stream
    \\
    BUC 2                 & Launch countdown starts                              &
    (in) Vocal count down from Mission Control \newline
    (out) Start recording the rocket flight
    \\
    BUC 3                 & Rocket takes flight                                  &
    (in) Video feed from camera \newline
    (out) Keeps pointing the camera to the rocket
    \\
    BUC 4                 & Rocket completes flight                              &
    (in) Video feed from camera \newline
    (out) Stop tracking the rocket \newline
    (out) Save the video  \newline
    (out) Update mission control on flight status
    \\
    BUC 5                 & Flight performance analyst wants to view the footage &
    (in) Request from Flight performance analyst \newline
    (out) Exported video files and logs
    \\
    \bottomrule
  \end{tabularx}
  \caption{Work Partitioning of System}
  \label{tab:work-part}
\end{table}

\subsection{Specifying a Business Use Case (BUC)}

Each event listed in \hyperref[tab:work-part]{Table 1} is expanded into an
individual business use case (BUC) which describes business logic in detail.

~\\

\textbf{BUC 1: Output live video.}

\textbf{Trigger:} None, always active.

\textbf{Interested Stakeholders:} Model Rocketry Event Organizers

\textbf{Preconditions:} Camera is powered on.

\textbf{Main Flow of Steps:}
\begin{enumerate}
  \item The camera operator connect the camera to the live streaming equipments.
\end{enumerate}

\textbf{Outcome:} The live stream receives a video feed from the camera.

~\\

\textbf{BUC 2: Launch countdown starts.}

\textbf{Trigger:} Mission Control begins the countdown to launch.

\textbf{Interested Stakeholders:} Model Rocketry Event Organizers.

\textbf{Preconditions:} Camera is powered on and ready to record.

\textbf{Main Flow of Steps:}
\begin{enumerate}
  \item The camera operator points the camera to the rocket
  \item The camera operator starts recording.
\end{enumerate}

\textbf{Outcome:} Recording of the rocket flight is started.

~\\

\textbf{BUC 3: Rocket takes flight.}

\textbf{Trigger:} Rocket ignition and liftoff are observed.

\textbf{Interested Stakeholders:} Model Rocketry Event Organizers, Aerospace Engineers and Researchers.

\textbf{Preconditions:} Camera is pointing at the rocket and recording.

\textbf{Main Flow of Steps:}
\begin{enumerate}
  \item The camera operator keeps pointing the camera to the rocket.
  \item If the rocket stages, the camera operator points the camera to the next stage.
\end{enumerate}

\textbf{Outcome:} The camera remains pointed at the rocket throughout the flight.

~\\

\textbf{BUC 4: Rocket completes flight.}

\textbf{Trigger:} Rocket landing is observed.

\textbf{Interested Stakeholders:} Model Rocketry Event Organizers, Model Rocketry Safety Officers.

\textbf{Preconditions:} Camera is pointing at the rocket and recording.

\textbf{Main Flow of Steps:}
\begin{enumerate}
  \item The camera operator confirms end of flight based on callout or observation.
  \item The camera operator ends the recording.
  \item The camera operator stops pointing the camera to the rocket.
  \item The camera operator communicates the flight status to Mission Control.
\end{enumerate}

\textbf{Outcome:} Video and logs are safely stored and the flight status is communicated to Mission Control.

~\\

\textbf{BUC 5: Flight performance analyst wants to view the footage.}

\textbf{Trigger:} Flight performance analyst requests to view the footage.

\textbf{Interested Stakeholders:} Aerospace Engineers and Researchers.

\textbf{Preconditions:} None.

\textbf{Main Flow of Steps:}
\begin{enumerate}
  \item The flight performance analyst provides date and time of the flight.
  \item The camera operator looks up the flight in the database.
  \item If the flight is found, the camera operator exports the video files and logs.
  \item If the flight is not found, the camera operator informs the flight performance
        analyst that the flight is not found.
\end{enumerate}

\textbf{Outcome:} The flight performance analyst receives the video files and logs.

\section{Business Data Model and Data Dictionary}
\subsection{Business Data Model}

The business data model consists of a list of launches, and for each launch,
the video footage and logs associated with the launch.

\begin{figure}[H]
  \centering
  \includegraphics[width=\textwidth,height=\textheight,keepaspectratio]{../Images/business_data_model.png}
  \caption{Business Data Model}
  \label{img:business-data-model}
\end{figure}

\subsection{Data Dictionary}

\begin{table}[H]
  \centering
  \setlength\extrarowheight{5mm}
  \begin{tabularx}{\textwidth}{lXp{1in}}
    \toprule \textbf{Name} & \textbf{Content}               &
    \textbf{Type}                                             \\
    \midrule
    Recording Session      & date + time                    &
    Class
    \\
    Video Footage          & Recorded video from the camera &
    Class
    \\
    Log                    & timestamp + gimbal angle       &
    Class
    \\
    \bottomrule
  \end{tabularx}
  \caption{Data Dictionary}
  \label{tab:data-dictionary}
\end{table}

\section{The Scope of the Product}
\subsection{Product Boundary}

The product boundary defines which parts of the work are handled by the system
and which parts remain the responsibility of human actors.

In \hyperref[img:product-boundary]{Figure 3}, The product use cases (PUCs) are
the ellipses inside the boundary. Each PUC is labeled with one or more BUCs
that it came from.

\FloatBarrier
\begin{figure}[h]
  \centering
  \includegraphics[width=\textwidth,height=\textheight,keepaspectratio]{../Images/product_boundary.png}
  \caption{Product Boundary}
  \label{img:product-boundary}
\end{figure}
\FloatBarrier

\subsection{Product Use Case Table}

The following \hyperref[tab:product-use-case-table]{Table 2} summarizes the
primary use cases of the system.

\begin{table}[H]
  \centering
  \setlength\extrarowheight{5mm}
  \begin{tabularx}{\textwidth}{p{0.4in}p{0.4in}p{1.25in}p{1in}X}
    \toprule \textbf{BUC No.}       & \textbf{PUC No.}        & \textbf{PUC Name}              &
    \textbf{Actors}                 & \textbf{Input / Output}                                    \\
    \midrule
    1                               & 1                       & Live video output              &
    Camera, Live Stream Technician
                                    &
    (in) Video feed from camera \newline
    (out) Live video output                                                                      \\
    2                               & 2.1                     & Point camera to rocket         &
    Camera Operator, Camera, Gimbal
                                    &
    (in) Manual gimbal adjustment commands from camera operator \newline
    (out) Movement commands to the gimbal                                                        \\
    2                               & 2.2                     & Start video recording          &
    Camera Operator, Camera
                                    &
    (in) Start video recording command \newline
    (out) Video recording started                                                                \\
    2                               & 2.3                     & Start tracking rocket          &
    Camera Operator, Camera, Gimbal
                                    &
    (in) Start tracking rocket command \newline
    (out) System in armed state                                                                  \\

    3                               & 3                       & Automated rocket tracking      &
    Camera, Gimbal                  &
    (in) Video feed from camera \newline
    (out) Movement commands to gimbal                                                            \\
    4                               & 4.1                     & Stop video recording           &
    Camera Operator, Camera         &
    (in) Stop video recording command \newline
    (out) Video recording stopped and saved                                                      \\
    4                               & 4.2                     & Stop tracking rocket           &
    Camera Operator, Camera, Gimbal &
    (in) Stop tracking rocket command \newline
    (out) Gimbal stops moving \newline
    (out) System in idle state                                                                   \\
    5                               & 5                       & View recorded footage and logs &
    Camera Operator                 &
    (in) request to download footage and logs \newline
    (out) Footage and logs downloaded                                                            \\

    \bottomrule
  \end{tabularx}
  \caption{Product Use Case Table}
  \label{tab:product-use-case-table}
\end{table}

\subsection{Individual Product Use Cases (PUC's)}

something something

~\\

\textbf{PUC 1: Live video output.}

\textbf{Actors:} Camera, Live Stream Technician.

\textbf{Preconditions:} The system is powered on and has finished initialization.

\textbf{Main Flow of Steps:}
\begin{enumerate}
  \item The camera operator connects the system to the live streaming equipments.
\end{enumerate}

\textbf{Outcome:} The live streaming equipments receive a video feed from the system.

~\\

\textbf{PUC 2.1: Point camera to rocket.}

\textbf{Actors:} Camera Operator, Camera, Gimbal.

\textbf{Preconditions:} The system is in idle state.

\textbf{Main Flow of Steps:}
\begin{enumerate}
  \item The camera operator monitors the camera preview on the UI.
  \item The camera operator selects the direction to rotate the gimbal on the UI.
  \item The system sends commands to the gimbal to rotate to the selected direction.
  \item The camera operator confirms the gimbal is rotating to the selected direction
        through the preview.
\end{enumerate}

\textbf{Outcome:} The gimbal is rotated to the selected direction.

~\\

\textbf{PUC 2.2: Start video recording.}

\textbf{Actors:} Camera Operator, Camera.

\textbf{Preconditions:} The system is powered on and has finished initialization; not currently recording.

\textbf{Main Flow of Steps:}
\begin{enumerate}
  \item The camera operator clicks the "Start Recording" button on the UI.
  \item The system starts to record the camera feed and additional logs (e.g. angle of
        the gimbal, temperature of the camera, etc.).
  \item The UI displays a recording indicator and the elapsed recording time.
\end{enumerate}

\textbf{Outcome:} Video recording starts and logs are written to storage.

~\\

\textbf{PUC 2.3: Start tracking rocket.}

\textbf{Actors:} Camera Operator, Camera, Gimbal.

\textbf{Preconditions:} System is in idle state.

\textbf{Main Flow of Steps:}
\begin{enumerate}
  \item The camera operator selects the "Arm" button on the UI.
  \item The system enters the armed state.
  \item The UI displays a "Armed" indicator.
\end{enumerate}

\textbf{Outcome:} The system enters the armed state and starts to look for rockets.

~\\

\textbf{PUC 3: Automated rocket tracking.}

\textbf{Actors:} Camera, Gimbal.

\textbf{Preconditions:} System is in armed state.

\textbf{Main Flow of Steps:}
\begin{enumerate}
  \item The system processes incoming frames to detect moving rockets.
  \item (Event) The system detects a moving rocket.
  \item The system enters the tracking state.
  \item The UI displays a "Tracking" indicator.
  \item The system sends movement commands to the gimbal to keep the rocket centered in
        the frame.
  \item If the rocket is lost, the system returns to idle state.
\end{enumerate}

\textbf{Outcome:} The gimbal follows the rocket and keeps it near the frame center until the rocket is lost.

~\\

\textbf{PUC 4.1: Stop video recording.}

\textbf{Actors:} Camera Operator, Camera.

\textbf{Preconditions:} The system is powered on and has finished initialization; currently recording.

\textbf{Main Flow of Steps:}
\begin{enumerate}
  \item The camera operator clicks the "Stop Recording" button on the UI.
  \item The system stops recording the camera feed and additional logs (e.g. angle of
        the gimbal, temperature of the camera, etc.).
  \item The system saves the video and logs to storage.
  \item The UI stops displaying the recording indicator.

\end{enumerate}

\textbf{Outcome:} Video recording stops.

~\\

\textbf{PUC 4.2: Stop tracking rocket.}

\textbf{Actors:} Camera Operator, Camera, Gimbal.

\textbf{Preconditions:} System is in armed or tracking state.

\textbf{Main Flow of Steps:}
\begin{enumerate}
  \item The camera operator clicks the "Stop Tracking" button on the UI.
  \item The system returns to idle state.
  \item The system stops sending movement commands to the gimbal.
  \item The UI displays a "Idle" indicator.
\end{enumerate}

\textbf{Outcome:} The system returns to idle state.

~\\

\textbf{PUC 5: View recorded footage and logs.}

\textbf{Actors:} Camera Operator.

\textbf{Preconditions:} The system is in idle state.

\textbf{Main Flow of Steps:}
\begin{enumerate}
  \item The camera operator opens the recordings page in the UI.
  \item The camera operator filters or selects the desired files by date/time.
  \item The camera operator downloads the files to their device.
\end{enumerate}

\textbf{Outcome:} The desired footage and logs are downloaded for offline viewing and analysis.

\subsection{System State Diagram}

The following is a state diagram of the system.

\FloatBarrier
\begin{figure}[h]
  \centering
  \includegraphics[width=\textwidth,height=\textheight,keepaspectratio]{../Images/state_diagram.png}
  \caption{System State Diagram}
  \label{img:state-diagram}
\end{figure}
\FloatBarrier

\section{Functional Requirements}
\subsection{Functional Requirements}

\begin{itemize}[leftmargin=*]

  \item[FR-1] \emph{The system must acquire live video stream from the connected
          camera.}\\[2mm]
        \textbf{Rationale:} Video acquisition is the first step of the vision-guided pipeline and enables all downstream processing.\\
        \textbf{Fit Criterion:} The system successfully captures frames at at least 1080p 60fps and passes them to the CV pipeline.\\
        \textbf{Source PUC:} PUC 1, PUC 3 \\
        \textbf{Priority:} High

  \item[FR-2] \emph{The system must send real-time movement commands to the connected
          gimbal.}\\[2mm]
        \textbf{Rationale:} Sending movement commands to gimbal is how the system controls the gimbal.\\
        \textbf{Fit Criterion:} Movement commands are sent to the gimbal at least once per video frame received.\\
        \textbf{Source PUC:} PUC2.1, PUC 3 \\
        \textbf{Priority:} High

  \item[FR-3] \emph{The system must detect moving rocket from the video stream in real
          time.}\\[2mm]
        \textbf{Rationale:} Real time rocket detection is the first step of the tracking process.\\
        \textbf{Fit Criterion:} The system can detect moving rockets from the video stream at 60fps.\\
        \textbf{Source PUC:} PUC 3 \\
        \textbf{Priority:} High

  \item[FR-4] \emph{If armed, the system must automatically enter the tracking state
          when a moving rocket is detected.}\\[2mm]
        \textbf{Rationale:} Entering the tracking state means the system is commanding the gimbal to keep the rocket centered in the frame.\\
        \textbf{Fit Criterion:} The system enters the tracking state when a moving rocket is detected.\\
        \textbf{Source PUC:} PUC 2.3, PUC 3 \\
        \textbf{Priority:} High

  \item[FR-5] \emph{If the system is in the tracking state and the rocket is lost, the
          system must automatically return to the idle state.}\\[2mm]
        \textbf{Rationale:} Automatically reverting to idle prevents unnecessary gimbal movement and prepares the system for future launches or manual intervention.\\
        \textbf{Fit Criterion:} Upon loss of the rocket target, the system transitions from tracking to idle state without manual input.\\
        \textbf{Source PUC:} PUC 3 \\
        \textbf{Priority:} High

  \item[FR-6] \emph{When in the idle state, the system must allow the user to manually
          control the gimbal.}\\[2mm]
        \textbf{Rationale:} Manual gimbal control is necessary for arming the camera to the launch pad before the rocket is launched.\\
        \textbf{Fit Criterion:} In idle mode, the user can issue manual pan/tilt commands that are reflected in real-time by the gimbal hardware.\\
        \textbf{Source PUC:} PUC 2.1 \\
        \textbf{Priority:} High

  \item[FR-7] \emph{When in the tracking state, and the rocket is not in view, the
          system must autonomously attempt re-acquisition, before declaring the rocket is
          lost.}\\[2mm]
        \textbf{Rationale:} Rocket may be temporarily obstructed; re-acquisition ensures robustness of the tracking process.\\
        \textbf{Fit Criterion:} The system waits for 2 seconds before declaring the rocket is lost.\\
        \textbf{Source PUC:} PUC 3 \\
        \textbf{Priority:} Medium

  \item[FR-8] \emph{When in the tracking state, the system must autonomously keep the
          rocket within the frame.}\\[2mm]
        \textbf{Rationale:} Accurate rocket tracking is the core function of the system.\\
        \textbf{Fit Criterion:} Rocket remains within visible in the frame.\\
        \textbf{Source PUC:} PUC 3 \\
        \textbf{Priority:} High

  \item[FR-9] \emph{The system must begin outputting video through HDMI as soon as it
          is initialized.}\\[2mm]
        \textbf{Rationale:} Immediate HDMI output ensures that event video is available to stakeholders from the earliest possible moment.\\
        \textbf{Fit Criterion:} Upon system initialization, an HDMI-connected display immediately receives a live video signal from the camera without requiring additional user action.\\
        \textbf{Source PUC:} PUC 1 \\
        \textbf{Priority:} High

  \item[FR-10] \emph{The HDMI video output must have a text overlay that displays the
          current gimbal tilt and pan angle.}\\[2mm]
        \textbf{Rationale:} Overlaying the gimbal angles on the video stream provides immediate feedback to operators and analysts and aids in diagnostics and post-flight analysis.\\
        \textbf{Fit Criterion:} The HDMI output includes a text label showing current tilt and pan angles, updating once per video frame.\\
        \textbf{Source PUC:} PUC 1 \\
        \textbf{Priority:} Medium

  \item[FR-11] \emph{The system must display runtime information including system state
          and camera preview to the user.}\\[2mm]
        \textbf{Rationale:} Seeing the system status and camera preview is necessary for the user to control the system.\\
        \textbf{Fit Criterion:} The interface updates system status and preview in real time with at least 15fps.\\
        \textbf{Source PUC:} PUC 2.1 \\
        \textbf{Priority:} High

  \item[FR-12] \emph{The system must only enter armed mode when explicitly instructed
          by the user.}\\[2mm]
        \textbf{Rationale:} Prevents accidental or unintended transitions to armed state, ensuring safe and predictable operation for all users.\\
        \textbf{Fit Criterion:} The system transitions from idle to armed mode only in response to a clear, user-initiated command, and never automatically or implicitly.\\
        \textbf{Source PUC:} PUC 2.3 \\
        \textbf{Priority:} High

  \item[FR-13] \emph{The system must allow users to record video and logs associated
          with the video for later review.}\\[2mm]
        \textbf{Rationale:} Recording supports debugging, validation, and demonstration.\\
        \textbf{Fit Criterion:} Users can start and stop recording from the UI.\\
        \textbf{Source PUC:} PUC 2.2, PUC 4.1 \\
        \textbf{Priority:} High

  \item[FR-14] \emph{The system must provide management functionalities for recorded
          videos and logs (list, delete, download).}\\[2mm]
        \textbf{Rationale:} Reviewing and managing captured sessions supports performance analysis and documentation.\\
        \textbf{Fit Criterion:} Users can view the list of recorded videos and logs, and perform delete/download actions.\\
        \textbf{Source PUC:} PUC 5 \\
        \textbf{Priority:} High

\end{itemize}
\section{Look and Feel Requirements}
\subsection{Appearance Requirements}
\begin{itemize}[leftmargin=*]
  \item[AR-1] \emph{The user interface shall be designed for higher readability in
          outdoor environments}\\[2mm]
        \textbf{Rationale:} The system is going to be used in outdoor environments\\
        \textbf{Fit Criterion:} Test users report no issues with readability in outdoor environments

  \item[AR-2] \emph{All UI elements required for tracking operation shall be visible in
          one page without scrolling on a 1080p display.}\\[2mm]
        \textbf{Rationale:} Operators need immediate awareness of system
        status under high-pressure launch operations.\\
        \textbf{Fit Criterion:} On a 1920x1080 monitor, all UI elements required for tracking operation are visible without scrolling
\end{itemize}

\subsection{Style Requirements}
\begin{itemize}[leftmargin=*]
  \item[SR-1] \emph{The user interface shall appear professional and reliable}\\[2mm]
        \textbf{Rationale:} The system is controlling a physical gimbal and it is important to gain the trust of the users.\\
        \textbf{Fit Criterion:} More than 80\% of Test users report the system appears professional and reliable
\end{itemize}

\section{Usability and Humanity Requirements}
\subsection{Ease of Use Requirements}
\begin{itemize}[leftmargin=*]
  \item[EZ-1] \emph{Units in the user interface shall be consistent}\\[2mm]
        \textbf{Rationale:} Consistency reduces operator error during high-pressure launch operations.\\
        \textbf{Fit Criterion:} A style audit of UI readouts finds no unit inconsistencies.

  \item[EZ-2] \emph{All user interface interactions shall provide immediate feedback to
          the user.}\\[2mm]
        \textbf{Rationale:} Immediate feedback reassures users that the system has registered their actions and helps prevent confusion or repeated actions during critical operations.\\
        \textbf{Fit Criterion:} For all UI actions, a visible confirmation appears within 0.2 seconds of user input in usability tests.

  \item[EZ-3] \emph{The system shall rely on visual indicators for system status and
          alerts.}\\[2mm]
        \textbf{Rationale:} The work environment may be noisy or require hearing protection, so visual cues ensure users do not miss important updates or warnings.\\
        \textbf{Fit Criterion:} All status and alerts are presented visually in the UI.
  \item[EZ-4] \emph{The product shall help the user to avoid making mistakes.}\\[2mm]
        \textbf{Rationale:} Because the system is connected to a physical gimbal, a user error might cause significant damage.\\
        \textbf{Fit Criterion:} Every potentially dangerous command presents a confirmation prompt that must be acknowledged before proceeding.
\end{itemize}

\subsection{Personalization and Internationalization Requirements}
\begin{itemize}[leftmargin=*]
  \item[PI-1] \emph{The user interface shall support both metric and imperial units}\\[2mm]
        \textbf{Rationale:} Allowing the user to pick the unit system with which they are most familiar helps reduce user mistakes. \\
        \textbf{Fit Criterion:} Unit preferences can be selected and applied to the UI.
  \item[PI-2] \emph{The user interface shall support both English and French
          languages}\\[2mm]
        \textbf{Rationale:} Bilingual support meets accessibility and inclusiveness standards for Canadian stakeholders and allows operation by users from different linguistic backgrounds.\\
        \textbf{Fit Criterion:} Users can select either English or French, and all UI text is presented in the chosen language.
\end{itemize}

\subsection{Learning Requirements}
\begin{itemize}[leftmargin=*]
  \item[LR-1] \emph{The system shall be easy to learn for users with a reasonable level
          of technological familiarity, provided they have access to the instruction
          manual.}\\[2mm]
        \textbf{Rationale:} Reducing the learning curve for technically literate users allows for faster and safer adoption, especially in time-pressured launch scenarios.\\
        \textbf{Fit Criterion:} In usability tests, new users who have read the manual are able to successfully perform all product use cases within 20 minutes without external assistance.

\end{itemize}

\subsection{Understandability and Politeness Requirements}
\begin{itemize}[leftmargin=*]
  \item[UPR-1] \emph{The system shall use symbols and words that are naturally
          understandable by the user community.}\\[2mm]
        \textbf{Rationale:} Using familiar terminology and icons reduces cognitive load and misinterpretation, especially during high-pressure tasks.\\
        \textbf{Fit Criterion:} User testing confirms that 90\% of surveyed users recognize and correctly interpret all UI symbols and labels without additional explanation.

  \item[UPR-2] \emph{The system should only display information required for normal use
          cases on the main page (i.e., hide debug information).}\\[2mm]
        \textbf{Rationale:} Restricting the main interface to essential information keeps the UI clear and reduces distraction or confusion for non-technical users.\\
        \textbf{Fit Criterion:} Reviews confirm that only operational and safety-relevant details are visible to end users during normal workflows; diagnostic/debug fields require an explicit action to reveal.

  \item[UPR-3] \emph{All confirmation dialogs should explain why an action could
          potentially be dangerous.}\\[2mm]
        \textbf{Rationale:} Providing context in confirmation dialogs educates users, prevents accidental execution of risky commands, and reinforces safe operational habits when interacting with hardware.\\
        \textbf{Fit Criterion:} All confirmation dialogs for safety-critical actions include a clear, specific explanation of risk, as verified by interface inspection.
\end{itemize}

\subsection{Accessibility Requirements}
\begin{itemize}[leftmargin=*]
  \item[AR-1] \emph{Status indicators shall remain distinguishable under common
          color-vision deficiencies.}\\[2mm]
        \textbf{Rationale:} Ensures legibility for diverse users and conditions.\\
        \textbf{Fit Criterion:} UI passes color-vision simulations.
\end{itemize}

\section{Performance Requirements}
\subsection{Speed and Latency Requirements}
\begin{itemize}[leftmargin=*]
  \item[SLR-1] \emph{camera-to-gimbal latency shall be low enough to enable continuous
          rocket tracking.}\\[2mm]
        \textbf{Rationale:} If the latency is too high, the gimbal will not ba able to keep up with the rocket\\
        \textbf{Fit Criterion:} Field tests confirm the latency is low enough to enable continuous rocket tracking.

  \item[SLR-2] \emph{The system shall be able to handle 1080p 60fps camera feed}\\[2mm]
        \textbf{Rationale:} The camera feed is the input to the system\\
        \textbf{Fit Criterion:} The system is able to handle 1080p 60fps camera feed.

  \item[SLR-3] \emph{The system shall be able to output 1080p 60fps video via HDMI}\\[2mm]
        \textbf{Rationale:} The HDMI output is the output of the system\\
        \textbf{Fit Criterion:} The system is able to output 1080p 60fps video via HDMI.
\end{itemize}

\subsection{Safety-Critical Requirements}
\begin{itemize}[leftmargin=*]
  \item[SCR-1] \emph{The system must allow the user to manually exit armed or tracking
          mode, transitioning the system to idle mode.}\\[2mm]
        \textbf{Rationale:} Manual override is necessary to ensure safety, provide user control in unexpected circumstances, and to prepare the system for the next operation.\\
        \textbf{Fit Criterion:} When the user issues an exit command during armed or tracking mode, the system transitions to idle mode within 0.5 second and ceases any automated gimbal movement.\\
        \textbf{Source PUC:} PUC 4.2 \\
        \textbf{Priority:} High
\end{itemize}

\subsection{Precision or Accuracy Requirements}
\begin{itemize}[leftmargin=*]
  \item[PAR-1] \emph{In the recorded footage and the live stream, the rocket shall be
          centered in the frame}\\[2mm]
        \textbf{Rationale:} Stable footage is important for the analysis of the flight\\
        \textbf{Fit Criterion:} The user is satisfied with the stability of the footage.
\end{itemize}

\subsection{Robustness or Fault-Tolerance Requirements}
\begin{itemize}[leftmargin=*]
  \item[RFR-1] \emph{The system shall report all errors to the user interface for
          timely notification and resolution.}\\[2mm]
        \textbf{Rationale:} Immediate visibility of errors ensures users can respond quickly, improving system safety and reliability.\\
        \textbf{Fit Criterion:} Any error condition, including hardware faults, communication losses, and software exceptions, generates a visible and descriptive message on the user interface as verified during failure injection testing.
  \item[RFR-2] \emph{The system shall immediately stop all automated operations and
          enter a safe state if any unrecoverable error occurs.}\\[2mm]
        \textbf{Rationale:} Rapid cessation of activity prevents damage to equipment and ensures user safety when continued operation is unsafe.\\
        \textbf{Fit Criterion:} Simulated unrecoverable errors (such as critical hardware failures or fatal software exceptions) cause the system to halt actuation and notify the user interface within 1 second.
\end{itemize}

\subsection{Capacity Requirements}
\begin{itemize}[leftmargin=*]
  \item[CR-1] \emph{The system must provide sufficient storage capacity to record at
          least 100 launches.}\\[2mm]
        \textbf{Rationale:} Ensures adequate capacity for a full series of events without requiring immediate offloading, supporting typical event schedules and operational continuity.\\
        \textbf{Fit Criterion:} System verifies available storage prior to use; test runs confirm that at least 100 launches are recorded and preserved without storage exhaustion.
\end{itemize}

\subsection{Scalability or Extensibility Requirements}

None.

\subsection{Longevity Requirements}

None.

\section{Operational and Environmental Requirements}
\subsection{Expected Physical Environment}
- noisy
- bright
- wide temperature range
\begin{itemize}[leftmargin=*]
  \item[EPE-1] \emph{The system shall operate normally—without overheating, system
          instability, or degraded performance—within the expected deployment
          environments.}\\[2mm]
        \textbf{Rationale:} Outdoor launch sites can present challenging ambient conditions; hardware and software must tolerate these environments reliably.\\
        \textbf{Fit Criterion:} During field trials at representative sites, the system remains stable and fully functional through a typical launch day, with no evidence of thermal shutdown or operating errors due to the environment.
\end{itemize}

\subsection{Wider Environment Requirements}

None.

\subsection{Requirements for Interfacing with Adjacent Systems}

- must use HDMI
- must provide a flexible interface for gimbal hardware

\begin{itemize}[leftmargin=*]
  \item[OER-INT-1] \emph{Provide an HDMI program feed and an IP stream compatible with
          common switchers/encoders; remote control range shall be $\geq$ 100\,m.}\\
        \textbf{Rationale:} Integrates with broadcast rigs and supports distant safe
        operation.\\ \textbf{Fit Criterion:} Verified ingest into OBS/capture cards;
        stable remote control over a measured 100\,m link.
\end{itemize}

\subsection{Productization Requirements}

None.

\subsection{Release Requirements}

- one release will be made by the end of the project
\begin{itemize}[leftmargin=*]
  \item[OER-REL-1] \emph{Each public release shall include a parameter sheet and links
          to system, network, and support documentation.}\\ \textbf{Rationale:} Speeds
        onboarding of new crew and stakeholders.\\ \textbf{Fit Criterion:} Release
        bundle contains the parameter sheet and referenced chapters.
\end{itemize}

\section{Maintainability and Support Requirements}

Not applicable, no features or changes are expected to be made after the
project is completed.

\subsection{Supportability Requirements}

Not applicable.

\subsection{Adaptability Requirements}

Not applicable, we are free to pick any hardware platform.

\section{Security Requirements}
\subsection{Access Requirements}

Not applicable, we assume that the system will not be connected to the public
internet, and only physically accessible by authorized personnel.

\subsection{Integrity Requirements}
\begin{itemize}[leftmargin=*]
  \item[SEC-IT-1] \emph{The system shall validate all user inputs to prevent invalid
          commands from being sent to the gimbal.}\\ \textbf{Rationale:} Prevents
        accidental or malicious input errors that could result in unsafe or unintended
        gimbal motion.\\ \textbf{Fit Criterion:} All gimbal control inputs from the
        user interface are checked for validity and range prior to execution; test
        cases confirm that malformed or out-of-range commands are rejected and not
        transmitted to the gimbal.
\end{itemize}

\subsection{Privacy Requirements}

None.

\subsection{Audit Requirements}
\begin{itemize}[leftmargin=*]
  \item[SEC-AU-1] \emph{The system shall log all operations performed by the user,
          recording each action with a timestamp.}\\ \textbf{Rationale:} Comprehensive
        logging increases traceability, supports troubleshooting, and ensures
        accountability by maintaining an accurate record of user activity.\\
        \textbf{Fit Criterion:} For every operation executed via the user interface,
        the system appends an entry to the log including the nature of the operation,
        the user (if applicable), and an accurate timestamp; audit tests confirm no
        user action proceeds unlogged.
\end{itemize}

\subsection{Immunity Requirements}

Not applicable, we assume that the system will not be connected to the public
internet, and only physically accessible by authorized personnel.

\section{Cultural Requirements}
\begin{itemize}[leftmargin=*]
  \item[CUL-1] \emph{The system must avoid using colours or symbols that could be
          culturally sensitive or offensive.}\\ \textbf{Rationale:} Ensuring cultural
        sensitivity in design helps avoid alienating or offending users from diverse
        backgrounds, which is critical for global acceptance and usability.\\
        \textbf{Fit Criterion:} Conduct a cultural review to ensure that all icons and
        colours used in the tool are neutral and universally acceptable.
  \item[CUL-2] \emph{The system must not include content that could be considered
          culturally insensitive.}\\ \textbf{Rationale:} Avoiding culturally insensitive
        content ensures that the tool is respectful and inclusive, fostering a positive
        user experience across different cultures.\\ \textbf{Fit Criterion:} A cultural
        sensitivity review is conducted to ensure all content is appropriate for a
        global audience.
\end{itemize}

\section{Compliance Requirements}
\subsection{Legal Requirements}
\begin{itemize}[leftmargin=*]
  \item[CMP-LG-1] \emph{The system must respect user privacy by avoiding the collection
          of any personal or identifiable data.}\\ \textbf{Rationale:} Ensuring privacy
        builds user trust and minimizes risks associated with handling sensitive
        information, even when specific privacy laws are not targeted.\\ \textbf{Fit
          Criterion:} The system avoids collecting or storing personal user data,
        including information about user activities and profiles, and operates entirely
        within the user's local environment.
\end{itemize}

\subsection{Standards Compliance Requirements}

None.

\section{Open Issues}

- is it possible to reliably track a small rocket fast enough given the current technology?

\section{Off-the-Shelf Solutions}
\subsection{Ready-Made Products}

- there are no off-the-shelf solutions that is capable of tracking a small and fast moving rocket.

\subsection{Reusable Components}

- computer vision algorithms

\subsection{Products That Can Be Copied}

- UI can take inspiration from other tracking systems (surveillance cameras, drones)

\section{New Problems}
\subsection{Effects on the Current Environment}

- the system shouldn't have any negative effects on the current environment

\subsection{Effects on the Installed Systems}

- The system interfaces with the existing live stream equipment using HDMI. No negative effects are expected because HDMI is an industry standard interface.

\subsection{Potential User Problems}

- If the system lost the rocket, the live stream technician will not be happy

\subsection{Follow-Up Problems}
\lips

\section{Tasks}
\subsection{Project Planning}
\lips

\subsection{Planning of the Development Phases}
\lips

\section{Migration to the New Product}
\subsection{Requirements for Migration to the New Product}

- no migration is needed

\subsection{Data That Has to be Modified or Translated for the New System}

- no migration is needed

\section{Costs}
\lips
\section{User Documentation and Training}
\subsection{User Documentation Requirements}
\lips
\subsection{Training Requirements}
\lips

\section{Waiting Room}

not applicable?

\section{Ideas for Solution}

- webpage for UI

\newpage{}
\section*{Appendix --- Reflection}

\input{../Reflection.tex}

\input{../SRS_Reflection.tex}

\end{document}
\documentclass[12pt, titlepage]{article}
\pdfinfoomitdate=1
\pdftrailerid{}

\usepackage{amsmath, mathtools}

\usepackage[round]{natbib}
\usepackage{amsfonts}
\usepackage{amssymb}
\usepackage{graphicx}
\usepackage{colortbl}
\usepackage{xr}
\usepackage{hyperref}
\usepackage{longtable}
\usepackage{xfrac}
\usepackage{tabularx}
\usepackage{float}
\usepackage{siunitx}
\usepackage{booktabs}
\usepackage{multirow}
\usepackage[section]{placeins}
\usepackage{caption}
\usepackage{fullpage}

\hypersetup{
bookmarks=true,     % show bookmarks bar?
colorlinks=true,       % false: boxed links; true: colored links
linkcolor=red,          % color of internal links (change box color with linkbordercolor)
citecolor=blue,      % color of links to bibliography
filecolor=magenta,  % color of file links
urlcolor=cyan          % color of external links
}

\usepackage{array}

\externaldocument{../../SRS/SRS}

\input{../../Comments}
%% Common Parts

\newcommand{\progname}{RoCam} % PUT YOUR PROGRAM NAME HERE
\newcommand{\authname}{Team \#3, SpaceY
  \\ Zifan Si
  \\ Jianqing Liu
  \\ Mike Chen
  \\ Xiaotian Lou} % AUTHOR NAMES                  

\usepackage{hyperref}
\hypersetup{colorlinks=true, linkcolor=blue, citecolor=blue, filecolor=blue,
  urlcolor=blue, unicode=false}
\urlstyle{same}

\usepackage{indentfirst}
\usepackage{graphicx}

\usepackage{titling}

\pretitle{\begin{center}\includegraphics[width=0.5\textwidth]{../../assets/logo/black.png}\\[0.75em]\LARGE}
    \posttitle{\par\end{center}}

\usepackage[letterpaper, portrait, margin=1in]{geometry}

\usepackage{placeins}
\usepackage{float}

\begin{document}

\title{Module Interface Specification for \progname{}}

\author{\authname}

\date{\today}

\maketitle

\pagenumbering{roman}

\section{Revision History}

\begin{tabularx}{\textwidth}{p{3cm}p{2cm}X}
  \toprule {\bf Date} & {\bf Version} & {\bf Notes} \\
  \midrule
  Date 1              & 1.0           & Notes       \\
  Date 2              & 1.1           & Notes       \\
  \bottomrule
\end{tabularx}

~\newpage

\section{Symbols, Abbreviations and Acronyms}

See SRS Documentation at:

\href{https://github.com/ZifanSi/vision-guided-tracker/blob/main/docs/SRS/SRS.pdf}{https://github.com/ZifanSi/vision-guided-tracker/blob/main/docs/SRS/SRS.pdf}

\wss{Also add any additional symbols, abbreviations or acronyms}

\newpage

\tableofcontents

\newpage

\pagenumbering{arabic}

\section{Introduction}

The following document details the Module Interface Specifications for Rocam:
High Performance Vision-Guided Rocket Tracker.

Complementary documents include the System Requirement Specifications and
Module Guide. The full documentation and implementation can be found at:

\href{https://github.com/ZifanSi/vision-guided-tracker}{https://github.com/ZifanSi/vision-guided-tracker}

\section{Notation}

\wss{You should describe your notation.  You can use what is below as
  a starting point.}

The structure of the MIS for modules comes from \citet{HoffmanAndStrooper1995},
with the addition that template modules have been adapted from
\cite{GhezziEtAl2003}. The mathematical notation comes from Chapter 3 of
\citet{HoffmanAndStrooper1995}. For instance, the symbol := is used for a
multiple assignment statement and conditional rules follow the form $(c_1
  \Rightarrow r_1 | c_2 \Rightarrow r_2 | ... | c_n \Rightarrow r_n )$.

The following table summarizes the primitive data types used by \progname.

\begin{center}
  \renewcommand{\arraystretch}{1.2}
  \noindent
  \begin{tabular}{l l p{7.5cm}}
    \toprule
    \textbf{Data Type} & \textbf{Notation} & \textbf{Description}                                             \\
    \midrule
    character          & char              & a single symbol or digit                                         \\
    integer            & $\mathbb{Z}$      & a number without a fractional component in (-$\infty$, $\infty$) \\
    natural number     & $\mathbb{N}$      & a number without a fractional component in [1, $\infty$)         \\
    real               & $\mathbb{R}$      & any number in (-$\infty$, $\infty$)                              \\
    \bottomrule
  \end{tabular}
\end{center}

\noindent
The specification of \progname \ uses some derived data types: sequences, strings, and
tuples. Sequences are lists filled with elements of the same data type. Strings
are sequences of characters. Tuples contain a list of values, potentially of
different types. In addition, \progname \ uses functions, which
are defined by the data types of their inputs and outputs. Local functions are
described by giving their type signature followed by their specification.

\section{Module Decomposition}

The following table is taken directly from the Module Guide document for this
project.

\begin{table}[h!]
  \centering
  \begin{tabular}{p{0.3\textwidth} p{0.6\textwidth}}
    \toprule
    \textbf{Level 1} & \textbf{Level 2}          \\
    \midrule

    Jetson Module    & Gimbal Abstraction Module \\ & Computer Vision Module \\  &
    Tracking Module                              \\ & Output Video Module \\ & Recording Module \\  & State
    Management Module                            \\ & API Gateway Module \\

    \midrule

    UI Module        & Preview Module            \\ & Manual Control Module \\  & Recording Management
    Module                                       \\ & Configuration Module\\ \bottomrule

  \end{tabular}
  \caption{Module Hierarchy}
  \label{TblMH}
\end{table}

\newpage
~\newpage

\section{MIS of Gimbal Abstraction Module} \label{mGimbal} \wss{Use labels for
  cross-referencing}

\wss{You can reference SRS labels, such as R\ref{R_Inputs}.}

\wss{It is also possible to use \LaTeX for hypperlinks to external documents.}

\subsection{Module}

Gimbal

\wss{Short name for the module}

\subsection{Uses}

This module does not use any other modules.

\subsection{Syntax}

\subsubsection{Exported Constants}

This module does not have any exported constants.

\subsubsection{Exported Access Programs}

\begin{center}
  \begin{tabular}{p{3cm} p{4cm} p{4cm} p{3.5cm}}
    \hline
    \textbf{Name}        & \textbf{In}         & \textbf{Out}        & \textbf{Exceptions}      \\
    \hline
    move\_deg            & tilt: f32, pan: f32 & -                   & gimbalCommunicationError \\
    \hline
    measure\_deg         & -                   & tilt: f32, pan: f32 & gimbalCommunicationError \\
    \hline
    control\_arm\_led    & enabled: bool       & -                   & gimbalCommunicationError \\
    \hline
    control\_status\_led & enabled: bool       & -                   & gimbalCommunicationError \\
    \hline
  \end{tabular}
\end{center}

\subsection{Semantics}

\subsubsection{State Variables}

\begin{itemize}
  \item presistent connection to the gimbal
\end{itemize}

\wss{Not all modules will have state variables.  State variables give the module
  a memory.}

\subsubsection{Environment Variables}

\begin{itemize}
  \item This module interacts with an external gimbal device.
\end{itemize}

\wss{This section is not necessary for all modules.  Its purpose is to capture
  when the module has external interaction with the environment, such as for a
  device driver, screen interface, keyboard, file, etc.}

\subsubsection{Assumptions}

\wss{Try to minimize assumptions and anticipate programmer errors via
  exceptions, but for practical purposes assumptions are sometimes appropriate.}

\subsubsection{Access Routine Semantics}

\noindent move\_deg(tilt: f32, pan: f32):
\begin{itemize}
  \item transition: sends the tilt and pan angles to the gimbal
  \item output: None
  \item exception: gimbalCommunicationError
\end{itemize}

\noindent measure\_deg():
\begin{itemize}
  \item transition: retrieves the tilt and pan angle measurements from the gimbal
  \item output: (tilt: f32, pan: f32)
  \item exception: gimbalCommunicationError
\end{itemize}

\noindent control\_arm\_led(enabled: bool):
\begin{itemize}
  \item transition: controls the arm LED on the gimbal
  \item output: None
  \item exception: gimbalCommunicationError
\end{itemize}

\noindent control\_status\_led(enabled: bool):
\begin{itemize}
  \item transition: controls the status LED on the gimbal
  \item output: None
  \item exception: gimbalCommunicationError
\end{itemize}

\wss{A module without environment variables or state variables is unlikely to
  have a state transition.  In this case a state transition can only occur if
  the module is changing the state of another module.}

\wss{Modules rarely have both a transition and an output.  In most cases you
  will have one or the other.}

\subsubsection{Local Functions}

\wss{As appropriate} \wss{These functions are for the purpose of specification.
  They are not necessarily something that is going to be implemented
  explicitly.  Even if they are implemented, they are not exported; they only
  have local scope.}

\section{MIS of Computer Vision Module} \label{mCV} \wss{Use labels for
  cross-referencing}

\wss{You can reference SRS labels, such as R\ref{R_Inputs}.}

\wss{It is also possible to use \LaTeX for hypperlinks to external documents.}

\subsection{Module}

CV

\wss{Short name for the module}

\subsection{Uses}

This module does not use any other modules.

\subsection{Syntax}

\subsubsection{Exported Constants}

\begin{itemize}
  \item WIDTH: width of the video feed (in pixels)
  \item HEIGHT: height of the video feed (in pixels)
\end{itemize}

\subsubsection{Exported Access Programs}

\begin{center}
  \begin{tabular}{p{3cm} p{4cm} p{4cm} p{3.5cm}}
    \hline
    \textbf{Name}         & \textbf{In} & \textbf{Out}   & \textbf{Exceptions}        \\
    \hline
    get\_frame            & -           & frame          & cameraError                \\
    \hline
    get\_rocket\_location & frame       & x: f32, y: f32 & noRocketFound, cameraError \\
    \hline
  \end{tabular}
\end{center}

\subsection{Semantics}

\subsubsection{State Variables}

\begin{itemize}
  \item computer vision model
  \item presistent connection to the camera sensor
\end{itemize}

\wss{Not all modules will have state variables.  State variables give the module
  a memory.}

\subsubsection{Environment Variables}

\begin{itemize}
  \item This module interacts with an external camera sensor.
\end{itemize}

\wss{This section is not necessary for all modules.  Its purpose is to capture
  when the module has external interaction with the environment, such as for a
  device driver, screen interface, keyboard, file, etc.}

\subsubsection{Assumptions}

\wss{Try to minimize assumptions and anticipate programmer errors via
  exceptions, but for practical purposes assumptions are sometimes appropriate.}

\subsubsection{Access Routine Semantics}

\noindent get\_frame():
\begin{itemize}
  \item transition: retrieves a frame from the camera sensor
  \item output: frame
  \item exception: cameraError
\end{itemize}

\noindent get\_rocket\_location(frame):
\begin{itemize}
  \item transition: analyzes the frame to find the rocket
  \item output: (x: f32, y: f32)
  \item exception: noRocketFound, cameraError
\end{itemize}

\wss{A module without environment variables or state variables is unlikely to
  have a state transition.  In this case a state transition can only occur if
  the module is changing the state of another module.}

\wss{Modules rarely have both a transition and an output.  In most cases you
  will have one or the other.}

\subsubsection{Local Functions}

\wss{As appropriate} \wss{These functions are for the purpose of specification.
  They are not necessarily something that is going to be implemented
  explicitly.  Even if they are implemented, they are not exported; they only
  have local scope.}

\section{MIS of Tracking Module} \label{mTracking} \wss{Use labels for
  cross-referencing}

\wss{You can reference SRS labels, such as R\ref{R_Inputs}.}

\wss{It is also possible to use \LaTeX for hypperlinks to external documents.}

\subsection{Module}

Tracking

\wss{Short name for the module}

\subsection{Uses}

This module uses the Computer Vision Module (\ref{mCV}) and the Gimbal
Abstraction Module (\ref{mGimbal}).

\subsection{Syntax}

\subsubsection{Exported Constants}

This module does not have any exported constants.

\subsubsection{Exported Access Programs}

\begin{center}
  \begin{tabular}{p{3cm} p{4cm} p{4cm} p{3.5cm}}
    \hline
    \textbf{Name} & \textbf{In} & \textbf{Out} & \textbf{Exceptions} \\
    \hline
    step          & -           & -            & -                   \\
    \hline
  \end{tabular}
\end{center}

\subsection{Semantics}

\subsubsection{State Variables}

\begin{itemize}
  \item PID parameters for controlling the gimbal
\end{itemize}

\wss{Not all modules will have state variables.  State variables give the module
  a memory.}

\subsubsection{Environment Variables}

None

\wss{This section is not necessary for all modules.  Its purpose is to capture
  when the module has external interaction with the environment, such as for a
  device driver, screen interface, keyboard, file, etc.}

\subsubsection{Assumptions}

\wss{Try to minimize assumptions and anticipate programmer errors via
  exceptions, but for practical purposes assumptions are sometimes appropriate.}

\subsubsection{Access Routine Semantics}

\noindent step():
\begin{itemize}
  \item transition:
        \begin{enumerate}
          \item Retrieve the current gimbal angle from the Gimbal Abstraction Module.
          \item Get the location of the rocket from the Computer Vision Module.
          \item Calculate the desired gimbal angle using the PID algorithm.
          \item Send the desired gimbal angle to the Gimbal Abstraction Module.
        \end{enumerate}
  \item output: None
  \item exception: None
\end{itemize}

\wss{A module without environment variables or state variables is unlikely to
  have a state transition.  In this case a state transition can only occur if
  the module is changing the state of another module.}

\wss{Modules rarely have both a transition and an output.  In most cases you
  will have one or the other.}

\subsubsection{Local Functions}

\wss{As appropriate} \wss{These functions are for the purpose of specification.
  They are not necessarily something that is going to be implemented
  explicitly.  Even if they are implemented, they are not exported; they only
  have local scope.}

\section{MIS of Output Video Module} \label{mOutputVideo} \wss{Use labels for
  cross-referencing}

\wss{You can reference SRS labels, such as R\ref{R_Inputs}.}

\wss{It is also possible to use \LaTeX for hypperlinks to external documents.}

\subsection{Module}

videoOut

\wss{Short name for the module}

\subsection{Uses}

\subsection{Syntax}

\subsubsection{Exported Constants}

\subsubsection{Exported Access Programs}

\begin{center}
  \begin{tabular}{p{3cm} p{4cm} p{4cm} p{3.5cm}}
    \hline
    \textbf{Name} & \textbf{In} & \textbf{Out} & \textbf{Exceptions} \\
    \hline
    output\_frame & states      & -            & outputDeviceError   \\
    \hline
  \end{tabular}
\end{center}

\subsection{Semantics}

\subsubsection{State Variables}

None

\wss{Not all modules will have state variables.  State variables give the module
  a memory.}

\subsubsection{Environment Variables}

This module displays the frame on the screen connected to the Jetson.

\wss{This section is not necessary for all modules.  Its purpose is to capture
  when the module has external interaction with the environment, such as for a
  device driver, screen interface, keyboard, file, etc.}

\subsubsection{Assumptions}

\wss{Try to minimize assumptions and anticipate programmer errors via
  exceptions, but for practical purposes assumptions are sometimes appropriate.}

\subsubsection{Access Routine Semantics}

\noindent output\_frame(states):
\begin{itemize}
  \item transition:
        \begin{enumerate}
          \item Retrieve the frame from the Computer Vision Module.
          \item Retrieve the location of the rocket from the Computer Vision Module.
          \item Crop the frame so the rocket is in the center of the frame.
          \item Overlay the states on the cropped frame.
          \item Display the cropped frame on the screen connected to the Jetson.
        \end{enumerate}
  \item output: None
  \item exception: outputDeviceError
\end{itemize}

\wss{A module without environment variables or state variables is unlikely to
  have a state transition.  In this case a state transition can only occur if
  the module is changing the state of another module.}

\wss{Modules rarely have both a transition and an output.  In most cases you
  will have one or the other.}

\subsubsection{Local Functions}

\wss{As appropriate} \wss{These functions are for the purpose of specification.
  They are not necessarily something that is going to be implemented
  explicitly.  Even if they are implemented, they are not exported; they only
  have local scope.}

\section{MIS of Recording Module} \label{mRecording} \wss{Use labels for
  cross-referencing}

\wss{You can reference SRS labels, such as R\ref{R_Inputs}.}

\wss{It is also possible to use \LaTeX for hypperlinks to external documents.}

\subsection{Module}

recording

\wss{Short name for the module}

\subsection{Uses}

\subsection{Syntax}

\subsubsection{Exported Constants}

\subsubsection{Exported Access Programs}

\begin{center}
  \begin{tabular}{p{3cm} p{4cm} p{4cm} p{3.5cm}}
    \hline
    \textbf{Name}     & \textbf{In}      & \textbf{Out}                & \textbf{Exceptions} \\
    \hline
    start\_recording  & -                & -                           & recordingError      \\
    \hline
    stop\_recording   & -                & -                           & recordingError      \\
    \hline
    list\_recordings  & -                & recordings: list[Recording] &                     \\
    \hline
    delete\_recording & recordingId: str & -                           &                     \\
    \hline
  \end{tabular}
\end{center}

\subsection{Semantics}

\subsubsection{State Variables}

\begin{itemize}
  \item recording status
\end{itemize}

\wss{Not all modules will have state variables.  State variables give the module
  a memory.}

\subsubsection{Environment Variables}

This module interacts with the file system to record the video and log files.

\wss{This section is not necessary for all modules.  Its purpose is to capture
  when the module has external interaction with the environment, such as for a
  device driver, screen interface, keyboard, file, etc.}

\subsubsection{Assumptions}

\wss{Try to minimize assumptions and anticipate programmer errors via
  exceptions, but for practical purposes assumptions are sometimes appropriate.}

\subsubsection{Access Routine Semantics}

\noindent start\_recording():
\begin{itemize}
  \item transition: starts recording the video and log files
  \item output: None
  \item exception: recordingError
\end{itemize}

\noindent stop\_recording():
\begin{itemize}
  \item transition: stops recording the video and log files
  \item output: None
  \item exception: recordingError
\end{itemize}

\noindent list\_recordings():
\begin{itemize}
  \item transition: retrieves the list of recordings from the file system
  \item output: recordings: list[Recording]
  \item exception: None
\end{itemize}

\noindent delete\_recording(recordingId: str):
\begin{itemize}
  \item transition: deletes the recording from the file system
  \item output: None
  \item exception: None
\end{itemize}

\wss{A module without environment variables or state variables is unlikely to
  have a state transition.  In this case a state transition can only occur if
  the module is changing the state of another module.}

\wss{Modules rarely have both a transition and an output.  In most cases you
  will have one or the other.}

\subsubsection{Local Functions}

\wss{As appropriate} \wss{These functions are for the purpose of specification.
  They are not necessarily something that is going to be implemented
  explicitly.  Even if they are implemented, they are not exported; they only
  have local scope.}

\section{MIS of State Management Module} \label{mStateManagement} \wss{Use labels for
  cross-referencing}

\wss{You can reference SRS labels, such as R\ref{R_Inputs}.}

\wss{It is also possible to use \LaTeX for hypperlinks to external documents.}

\subsection{Module}

stateManagement

\wss{Short name for the module}

\subsection{Uses}

This module uses the Recording Module (\ref{mRecording}), Tracking Module
(\ref{mTracking}), and Output Video Module (\ref{mOutputVideo}).

\subsection{Syntax}

\subsubsection{Exported Constants}

This module does not have any exported constants.

\subsubsection{Exported Access Programs}

\begin{center}
  \begin{tabular}{p{3cm} p{4cm} p{4cm} p{3.5cm}}
    \hline
    \textbf{Name}   & \textbf{In} & \textbf{Out} & \textbf{Exceptions} \\
    \hline
    arm             & -           & -            & -                   \\
    \hline
    disarm          & -           & -            & -                   \\
    \hline
    manual\_control & direction   & -            & -                   \\
    \hline
  \end{tabular}
\end{center}

\subsection{Semantics}

\subsubsection{State Variables}

\begin{itemize}
  \item armed status
\end{itemize}

\wss{Not all modules will have state variables.  State variables give the module
  a memory.}

\subsubsection{Environment Variables}

None

\wss{This section is not necessary for all modules.  Its purpose is to capture
  when the module has external interaction with the environment, such as for a
  device driver, screen interface, keyboard, file, etc.}

\subsubsection{Assumptions}

\wss{Try to minimize assumptions and anticipate programmer errors via
  exceptions, but for practical purposes assumptions are sometimes appropriate.}

\subsubsection{Access Routine Semantics}

\noindent arm():
\begin{itemize}
  \item transition:
        \begin{enumerate}
          \item Set the armed state variable to true.
          \item Start a loop in the background, which does the following until the armed state variable is set to false:
          \begin{enumerate}
            \item Call "get\_frame" from the Computer Vision Module.
            \item Call "get\_rocket\_location" from the Computer Vision Module.
            \item Call "step" from the Tracking Module to adjust the gimbal to keep the rocket in the center of the frame.
            \item Call "output\_frame" from the Output Video Module to display the frame on the screen.
          \end{enumerate}
        \end{enumerate}
  \item output: None
  \item exception: None
\end{itemize}

\noindent disarm():
\begin{itemize}
  \item transition: sets the armed state variable to false
  \item output: None
  \item exception: None
\end{itemize}

\noindent manual\_control(direction):
\begin{itemize}
  \item transition: adjusts the gimbal to the given direction if the armed state variable is false
  \item output: None
  \item exception: None
\end{itemize}

\wss{A module without environment variables or state variables is unlikely to
  have a state transition.  In this case a state transition can only occur if
  the module is changing the state of another module.}

\wss{Modules rarely have both a transition and an output.  In most cases you
  will have one or the other.}

\subsubsection{Local Functions}

\wss{As appropriate} \wss{These functions are for the purpose of specification.
  They are not necessarily something that is going to be implemented
  explicitly.  Even if they are implemented, they are not exported; they only
  have local scope.}

\section{MIS of API Gateway Module} \label{mAPI} \wss{Use labels for
  cross-referencing}

\wss{You can reference SRS labels, such as R\ref{R_Inputs}.}

\wss{It is also possible to use \LaTeX for hypperlinks to external documents.}

\subsection{Module}

apiGateway

\wss{Short name for the module}

\subsection{Uses}

This module uses the State Management Module (\ref{mStateManagement}) .

\subsection{Syntax}

\subsubsection{Exported Constants}

This module does not have any exported constants.

\subsubsection{Exported Access Programs}

\begin{center}
  \begin{tabular}{p{3cm} p{4cm} p{4cm} p{3.5cm}}
    \hline
    \textbf{Name} & \textbf{In} & \textbf{Out} & \textbf{Exceptions} \\
    \hline
    start\_server & -           & -            & -                   \\
    \hline
  \end{tabular}
\end{center}

\subsection{Semantics}

\subsubsection{State Variables}

None

\wss{Not all modules will have state variables.  State variables give the module
  a memory.}

\subsubsection{Environment Variables}

None

\wss{This section is not necessary for all modules.  Its purpose is to capture
  when the module has external interaction with the environment, such as for a
  device driver, screen interface, keyboard, file, etc.}

\subsubsection{Assumptions}

\wss{Try to minimize assumptions and anticipate programmer errors via
  exceptions, but for practical purposes assumptions are sometimes appropriate.}

\subsubsection{Access Routine Semantics}

\noindent start\_server():
\begin{itemize}
  \item transition: starts the api server and listens for requests from the web interface.
  \item output: None
  \item exception: None
\end{itemize}

\wss{A module without environment variables or state variables is unlikely to
  have a state transition.  In this case a state transition can only occur if
  the module is changing the state of another module.}

\wss{Modules rarely have both a transition and an output.  In most cases you
  will have one or the other.}

\subsubsection{Local Functions}

\wss{As appropriate} \wss{These functions are for the purpose of specification.
  They are not necessarily something that is going to be implemented
  explicitly.  Even if they are implemented, they are not exported; they only
  have local scope.}

\section{MIS of Preview Module} \label{mPreview} \wss{Use labels for
  cross-referencing}

\wss{You can reference SRS labels, such as R\ref{R_Inputs}.}

\wss{It is also possible to use \LaTeX for hypperlinks to external documents.}

\subsection{Module}

preview

\wss{Short name for the module}

\subsection{Uses}

This module uses the API Gateway Module (\ref{mAPI}).

\subsection{Syntax}

\subsubsection{Exported Constants}

This module does not have any exported constants.

\subsubsection{Exported Access Programs}

This module does not have any exported access programs

\subsection{Semantics}

\subsubsection{State Variables}

None

\wss{Not all modules will have state variables.  State variables give the module
  a memory.}

\subsubsection{Environment Variables}

\begin{itemize}
  \item This module shows the preview on the screen.
\end{itemize}

\wss{This section is not necessary for all modules.  Its purpose is to capture
  when the module has external interaction with the environment, such as for a
  device driver, screen interface, keyboard, file, etc.}

\subsubsection{Assumptions}

\wss{Try to minimize assumptions and anticipate programmer errors via
  exceptions, but for practical purposes assumptions are sometimes appropriate.}

\subsubsection{Access Routine Semantics}


\wss{A module without environment variables or state variables is unlikely to
  have a state transition.  In this case a state transition can only occur if
  the module is changing the state of another module.}

\wss{Modules rarely have both a transition and an output.  In most cases you
  will have one or the other.}

\subsubsection{Local Functions}

\wss{As appropriate} \wss{These functions are for the purpose of specification.
  They are not necessarily something that is going to be implemented
  explicitly.  Even if they are implemented, they are not exported; they only
  have local scope.}

\section{MIS of Manual Control Module} \label{mManualControl} \wss{Use labels for
  cross-referencing}

\wss{You can reference SRS labels, such as R\ref{R_Inputs}.}

\wss{It is also possible to use \LaTeX for hypperlinks to external documents.}

\subsection{Module}

manualControl

\wss{Short name for the module}

\subsection{Uses}

This module uses the State Management Module (\ref{mStateManagement}).

\subsection{Syntax}

\subsubsection{Exported Constants}

This module does not have any exported constants.

\subsubsection{Exported Access Programs}

This module does not have any exported access programs

\subsection{Semantics}

\subsubsection{State Variables}

None

\wss{Not all modules will have state variables.  State variables give the module
  a memory.}

\subsubsection{Environment Variables}

\begin{itemize}
  \item This module shows the manual control interface on the screen.
\end{itemize}

\wss{This section is not necessary for all modules.  Its purpose is to capture
  when the module has external interaction with the environment, such as for a
  device driver, screen interface, keyboard, file, etc.}

\subsubsection{Assumptions}

\wss{Try to minimize assumptions and anticipate programmer errors via
  exceptions, but for practical purposes assumptions are sometimes appropriate.}

\subsubsection{Access Routine Semantics}

\wss{A module without environment variables or state variables is unlikely to
  have a state transition.  In this case a state transition can only occur if
  the module is changing the state of another module.}

\wss{Modules rarely have both a transition and an output.  In most cases you
  will have one or the other.}

\subsubsection{Local Functions}

\wss{As appropriate} \wss{These functions are for the purpose of specification.
  They are not necessarily something that is going to be implemented
  explicitly.  Even if they are implemented, they are not exported; they only
  have local scope.}
\section{MIS of Recording Management Module} \label{mRecordingManagement} \wss{Use labels for
  cross-referencing}

\wss{You can reference SRS labels, such as R\ref{R_Inputs}.}

\wss{It is also possible to use \LaTeX for hypperlinks to external documents.}

\subsection{Module}

recordingManagement

\wss{Short name for the module}

\subsection{Uses}

This module uses the Recording Module (\ref{mRecording}).

\subsection{Syntax}

\subsubsection{Exported Constants}

This module does not have any exported constants.

\subsubsection{Exported Access Programs}

This module does not have any exported access programs

\subsection{Semantics}

\subsubsection{State Variables}

\begin{itemize}
  \item recording list
\end{itemize}

\wss{Not all modules will have state variables.  State variables give the module
  a memory.}

\subsubsection{Environment Variables}

\begin{itemize}
  \item This module manages the recordings.
\end{itemize}

\wss{This section is not necessary for all modules.  Its purpose is to capture
  when the module has external interaction with the environment, such as for a
  device driver, screen interface, keyboard, file, etc.}

\subsubsection{Assumptions}

\wss{Try to minimize assumptions and anticipate programmer errors via
  exceptions, but for practical purposes assumptions are sometimes appropriate.}

\subsubsection{Access Routine Semantics}

\wss{A module without environment variables or state variables is unlikely to
  have a state transition.  In this case a state transition can only occur if
  the module is changing the state of another module.}

\wss{Modules rarely have both a transition and an output.  In most cases you
  will have one or the other.}

\subsubsection{Local Functions}

\wss{As appropriate} \wss{These functions are for the purpose of specification.
  They are not necessarily something that is going to be implemented
  explicitly.  Even if they are implemented, they are not exported; they only
  have local scope.}
\section{MIS of Configuration Module} \label{mConfiguration} \wss{Use labels for
  cross-referencing}

\wss{You can reference SRS labels, such as R\ref{R_Inputs}.}

\wss{It is also possible to use \LaTeX for hypperlinks to external documents.}

\subsection{Module}

configuration

\wss{Short name for the module}

\subsection{Uses}

This module uses the Recording Management Module (\ref{mRecordingManagement}).

\subsection{Syntax}

\subsubsection{Exported Constants}

This module does not have any exported constants.

\subsubsection{Exported Access Programs}

This module does not have any exported access programs

\subsection{Semantics}

\subsubsection{State Variables}

\begin{itemize}
  \item configuration settings
\end{itemize}

\wss{Not all modules will have state variables.  State variables give the module
  a memory.}

\subsubsection{Environment Variables}

\begin{itemize}
  \item This module manages the configuration settings.
\end{itemize}

\wss{This section is not necessary for all modules.  Its purpose is to capture
  when the module has external interaction with the environment, such as for a
  device driver, screen interface, keyboard, file, etc.}

\subsubsection{Assumptions}

\wss{Try to minimize assumptions and anticipate programmer errors via
  exceptions, but for practical purposes assumptions are sometimes appropriate.}

\subsubsection{Access Routine Semantics}

\wss{A module without environment variables or state variables is unlikely to
  have a state transition.  In this case a state transition can only occur if
  the module is changing the state of another module.}

\wss{Modules rarely have both a transition and an output.  In most cases you
  will have one or the other.}

\subsubsection{Local Functions}

\wss{As appropriate} \wss{These functions are for the purpose of specification.
  They are not necessarily something that is going to be implemented
  explicitly.  Even if they are implemented, they are not exported; they only
  have local scope.}

\newpage

\bibliographystyle {plainnat}
\bibliography {../../../refs/References}

\newpage

\section{Appendix} \label{Appendix}

\wss{Extra information if required}

\newpage{}

\section*{Appendix --- Reflection}

\wss{Not required for CAS 741 projects}

The information in this section will be used to evaluate the team members on
the graduate attribute of Problem Analysis and Design.

\input{../../Reflection.tex}

\begin{enumerate}
  \item What went well while writing this deliverable?
  \item What pain points did you experience during this deliverable, and how did you
        resolve them?
  \item Which of your design decisions stemmed from speaking to your client(s) or a
        proxy (e.g. your peers, stakeholders, potential users)? For those that were
        not, why, and where did they come from?
  \item While creating the design doc, what parts of your other documents (e.g.
        requirements, hazard analysis, etc), it any, needed to be changed, and why?
  \item What are the limitations of your solution? Put another way, given unlimited
        resources, what could you do to make the project better? (LO\_ProbSolutions)
  \item Give a brief overview of other design solutions you considered. What are the
        benefits and tradeoffs of those other designs compared with the chosen design?
        From all the potential options, why did you select the documented design?
        (LO\_Explores)
\end{enumerate}

\end{document}
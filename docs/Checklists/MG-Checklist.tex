\documentclass[12pt]{article}
\pdfinfoomitdate=1
\pdftrailerid{}

\usepackage{hyperref}
\hypersetup{colorlinks=true,
    linkcolor=blue,
    citecolor=blue,
    filecolor=blue,
    urlcolor=blue,
    unicode=false}
\urlstyle{same}

\usepackage{enumitem,amssymb}
\newlist{todolist}{itemize}{2}
\setlist[todolist]{label=$\square$}
\usepackage{pifont}
\newcommand{\cmark}{\ding{51}}%
\newcommand{\xmark}{\ding{55}}%
\newcommand{\done}{\rlap{$\square$}{\raisebox{2pt}{\large\hspace{1pt}\cmark}}%
\hspace{-2.5pt}}
\newcommand{\wontfix}{\rlap{$\square$}{\large\hspace{1pt}\xmark}}

\begin{document}

\title{MG Checklist}
\author{Spencer Smith}
\date{\today}

\maketitle

% Show an item is done by   \item[\done] Frame the problem
% Show an item will not be fixed by   \item[\wontfix] profit

\begin{itemize}

  \item Follows writing checklist (full checklist provided in a separate document)
        \begin{todolist}
          \item \LaTeX{} points
          \item Structure
          \item Spelling, grammar, attention to detail
          \item Avoid low information content phrases
          \item Writing style
          \item Hyperlinks should be done properly (\texttt{\textbackslash ref})
        \end{todolist}

  \item Module Decomposition
        \begin{todolist}
          \item One module one secret (unless an explicit exception is made, with a good
          reason) - all ``and''s should be checked.
          \item The uses relation is a hierarchy.
          \item Secrets are nouns (generally).
          \item Traceability matrix between modules and requirements shows every requirement is
          satisfied by at least on module
          \item Traceability matrix between modules and requirements shows that every module is
          used to satisfy at least one requirement
          \item Traceability matrix between likely changes and modules shows a one to one
          mapping, or, if this is not the case, explains the exceptions to this rule.
          \item Level 1 of the decomposition by secrets shows: Hardware-Hiding,
          Behaviour-Hiding and Software Decision Hiding.
          \item Behaviour-Hiding modules are related to the requirements
          \item Software-Decision hiding modules are concepts that need to be introduced, but
          are not detailed in the requirements
          \item Each Software Decision Hiding module is used by at least one Behaviour-Hiding
          Module (if this isn't the case, an explanation should be provided)
          \item Uses relation is not confused with a data flow chart. If you can imagine an
          ``import B'' statement in the code for module A, then module A uses module B.
          \item The arrow in the uses relation points from module A to module B when module A
          uses module B
          \item Anticipated changes are a superset of the likely changes in the SRS
          \item If there is a ``control'' module, it should be at the top of the hierarchy
          \item Ideally the uses relation is drawn with all uses arrows pointing down, with
          clear layers for the hierarchy
        \end{todolist}

  \item MG quality
        \begin{todolist}
          \item Follow template
          \item Low coupling
          \item Satisfies information hiding
          \item Figures can be zoomed in on (pdf better than bitmap for zooming)
        \end{todolist}
\end{itemize}

\end{document}

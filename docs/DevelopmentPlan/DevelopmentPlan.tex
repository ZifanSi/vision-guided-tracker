\documentclass{article}

\usepackage{booktabs}
\usepackage{tabularx}

\title{Development Plan\\\progname}

\author{\authname}

\date{}

\input{../Comments}
%% Common Parts

\newcommand{\progname}{RoCam} % PUT YOUR PROGRAM NAME HERE
\newcommand{\authname}{Team \#3, SpaceY
  \\ Zifan Si
  \\ Jianqing Liu
  \\ Mike Chen
  \\ Xiaotian Lou} % AUTHOR NAMES                  

\usepackage{hyperref}
\hypersetup{colorlinks=true, linkcolor=blue, citecolor=blue, filecolor=blue,
  urlcolor=blue, unicode=false}
\urlstyle{same}

\usepackage{indentfirst}
\usepackage{graphicx}

\usepackage{titling}

\pretitle{\begin{center}\includegraphics[width=0.5\textwidth]{../../assets/logo/black.png}\\[0.75em]\LARGE}
    \posttitle{\par\end{center}}

\usepackage[letterpaper, portrait, margin=1in]{geometry}

\usepackage{placeins}
\usepackage{float}

\begin{document}

\maketitle

\begin{table}[hp]
  \caption{Revision History} \label{TblRevisionHistory}
  \begin{tabularx}{\textwidth}{llX}
    \toprule
    \textbf{Date}  & \textbf{Developer(s)} & \textbf{Change}                               \\
    \midrule
    Sept. 10, 2025 & Mike Chen             & Proof of Concept Plan and Expected Technology \\
    Sept. 13, 2025 & Jianqing Liu          & Team Member Roles and Communication Plan      \\
    Sept. 13, 2025 & Jianqing Liu          & Expected Technology and Coding Standard       \\
    \bottomrule
  \end{tabularx}
\end{table}

\newpage{}

\wss{Put your introductory blurb here.  Often the blurb is a brief roadmap of
  what is contained in the report.}

\wss{Additional information on the development plan can be found in the
  \href{https://gitlab.cas.mcmaster.ca/courses/capstone/-/blob/main/Lectures/L02b_POCAndDevPlan/POCAndDevPlan.pdf?ref_type=heads}
  {lecture slides}.}

\section{Confidential Information?}

\wss{State whether your project has confidential information from industry, or
  not.  If there is confidential information, point to the agreement you have in
  place.}

\wss{For most teams this section will just state that there is no confidential
  information to protect.}
\section{IP to Protect}

\wss{State whether there is IP to protect.  If there is, point to the agreement.
  All students who are working on a project that requires an IP agreement are also
  required to sign the ``Intellectual Property Guide Acknowledgement.''}

\section{Copyright License}

\wss{What copyright license is your team adopting.  Point to the license in your
  repo.}

\section{Team Meeting Plan}

\wss{How often will you meet? where?}

\wss{If the meeting is a physical location (not virtual), out of an abundance of
  caution for safety reasons you shouldn't put the location online}

\wss{How often will you meet with your industry advisor?  when?  where?}

\wss{Will meetings be virtual?  At least some meetings should likely be
  in-person.}

\wss{How will the meetings be structured?  There should be a chair for all meetings.  There should be an agenda for all meetings.}

\section{Team Communication Plan}

\begin{itemize}
  \item \textbf{Discord}: Used for general team communication, informal discussions, quick updates, and coordination of meetings. Team members will use Discord for day-to-day conversations, sharing progress updates, asking general questions, and maintaining team cohesion.
  \item \textbf{GitHub}: Used for all code-related discussions and project management. GitHub Issues will be used to track bugs, feature requests, and tasks. GitHub Pull Requests will be used for code reviews and discussions related to specific code changes. Any technical discussions related to implementation details, code quality, or specific issues will be conducted through GitHub's commenting system on the relevant issue or pull request.
\end{itemize}

\wss{Issues on GitHub should be part of your communication plan.}

\section{Team Member Roles}

\wss{You should identify the types of roles you anticipate, like notetaker,
  leader, meeting chair, reviewer.  Assigning specific people to those roles is
  not necessary at this stage.  In a student team the role of the individuals will
  likely change throughout the year.}

\begin{itemize}
  \item \textbf{Project Manager}: Oversees project timeline, coordinates tasks between team members, manages deliverables and deadlines, and serves as primary point of contact with supervisor and stakeholders.
  \item \textbf{Meeting Chair}: Leads team meetings, prepares agendas, ensures discussions stay on track, facilitates decision-making, and manages meeting time effectively.
  \item \textbf{Notetaker}: Records meeting minutes, tracks action items and decisions, maintains documentation of team discussions, and distributes meeting summaries to all members.
  \item \textbf{Quality Assurance}: Reviews code, documentation, and deliverables for quality and consistency, conducts testing and validation, ensures adherence to coding standards and project requirements, and manages the review process for all team outputs.
\end{itemize}

\section{Workflow Plan}

\begin{itemize}
  \item How will you be using git, including branches, pull request, etc.?
  \item How will you be managing issues, including template issues, issue
        classification, etc.?
  \item Use of CI/CD
\end{itemize}

\section{Project Decomposition and Scheduling}

\begin{itemize}
  \item How will you be using GitHub projects?
  \item Include a link to your GitHub project
\end{itemize}

\wss{How will the project be scheduled?  This is the big picture schedule, not
  details. You will need to reproduce information that is in the course outline
  for deadlines.}

\section{Proof of Concept Demonstration Plan}

The following are planned steps of POC:

\begin{enumerate}
  \item Have initial image acquired from a camera, where the target is stationary.
  \item Activate tracing mode of our system, it will segment and detect multiple moving
        objects in the image.
        \begin{itemize}
          \item The segmentation and moving object detection will be done using Computer Vision
                techniques.
          \item The Computer Vision model will be deployed on a Jetson Nano.
        \end{itemize}
  \item user select a stationary or moving object as the target.
  \item As the target moves, the system will keep it at the center of the image for
        smooth tracking.
        \begin{itemize}
          \item The system will be able to handle occlusion and loss of target.
          \item user can manually re-select the target if needed.
          \item the real time control of the camera will be done using a STM32 microcontroller.
        \end{itemize}
\end{enumerate}

The following is a list of primary risks to consider from the POC:
\begin{enumerate}
  \item The computer vision system may not be able to process images at a fast enough
        rate.
        \begin{itemize}
          \item If this occurs, we will try to optimize the existing model, consider using a
                more powerful board, lower frame rate, or consider using traditional
                algorithmic approach and technique, where it detects motion by comparing image
                pixel wise.
          \item We also reserve the option of use models trained for specific set of objects to
                increase the speed of the model.
        \end{itemize}
  \item The STM32 may not be able to deliver the real time control to the camera.
        \begin{itemize}
          \item If this occurs, we will consider using a more powerful microcontroller, or
                using a different control technique.
        \end{itemize}
  \item The Integration of the frontend, backend, computer vision model, and
        microcontroller may be more difficult than expected.
        \begin{itemize}
          \item If this occurs, we will consider using a more powerful integrated micro
                computer or deploy via cloud computation, redesign our control flow, to make
                integration easier.
        \end{itemize}
\end{enumerate}

Other smaller risks to consider:
\begin{enumerate}
  \item UI Useability issues: The user interface may not be intuitive or easy to use,
        leading to user frustration or errors.
        \begin{itemize}
          \item {Potential Solution}: Conduct user testing and gather feedback to improve the interface design.
        \end{itemize}
\end{enumerate}

\section{Expected Technology}

\wss{What programming language or languages do you expect to use?  What external
  libraries?  What frameworks?  What technologies.  Are there major components of
  the implementation that you expect you will implement, despite the existence of
  libraries that provide the required functionality.  For projects with machine
  learning, will you use pre-trained models, or be training your own model?  }

\wss{The implementation decisions can, and likely will, change over the course
  of the project.  The initial documentation should be written in an abstract way;
  it should be agnostic of the implementation choices, unless the implementation
  choices are project constraints.  However, recording our initial thoughts on
  implementation helps understand the challenge level and feasibility of a
  project.  It may also help with early identification of areas where project
  members will need to augment their training.}

\wss{git, GitHub and GitHub projects should be part of your technology.}

% Topics to discuss include the following:

% \begin{itemize}
%   \item Specific programming language
%   \item Specific libraries
%   \item Pre-trained models
%   \item Specific linter tool (if appropriate)
%   \item Specific unit testing framework
%   \item Investigation of code coverage measuring tools
%   \item Specific plans for Continuous Integration (CI), or an explanation that CI is
%         not being done
%   \item Specific performance measuring tools (like Valgrind), if appropriate
%   \item Tools you will likely be using?
% \end{itemize}

\begin{itemize}
  \item Motion Control
        \begin{itemize}
          \item STM32 Microcontroller
          \item Rust Programming Language
          \item Linter: rust-clippy
          \item Unit Testing: Rust built-in
          \item Code Coverage: grcov
          \item Hardware Abstraction Layer: embassy-rs
          \item Debugging: probe-rs
        \end{itemize}
  \item Computer Vision
        \begin{itemize}
          \item Nvidia Jetson
          \item Nvidia Jetpack
          \item Language: Python
          \item Libraries: OpenCV, NumPy, Matplotlib, Torch
          \item Open Source Models: YOLO series, SAM, various ViTs
          \item Linter: pylint
        \end{itemize}
  \item Web App
        \begin{itemize}
          \item Web Server: Flask
          \item React
          \item Typescript
          \item Linter: eslint
          \item End-to-end Testing: Cypress
        \end{itemize}
  \item All the above will use GitHub Actions for CI
  \item Development Tools
        \begin{itemize}
          \item VSCode
          \item Git
          \item Github
        \end{itemize}
\end{itemize}

\section{Coding Standard}

\begin{itemize}
  \item Rust:
        \href{https://en.wikipedia.org/wiki/The_Power_of_10:_Rules_for_Developing_Safety-Critical_Code}{The
          Power of 10 Rules}
  \item Python:
        Follow PEP 8 / PEP 257 / PEP 484; use Black for formatting and pylint and Ruff for static analysis; \\
        use pytest + coverage for tests ($\geq 50\%$) on Python 3.9–3.11 across Linux, macOS, and Windows;; \\
        perform SAST with Bandit; all of the above are enforced via GitHub Actions.\\
  \item Typescript:
        \href{https://typescript-eslint.io/packages/typescript-eslint}{typescript-eslint}

\end{itemize}

\wss{What coding standard will you adopt?}

\newpage{}

\section*{Appendix --- Reflection}

\wss{Not required for CAS 741}

\input{../Reflection.tex}

\begin{enumerate}
  \item Why is it important to create a development plan prior to starting the project?
  \item In your opinion, what are the advantages and disadvantages of using CI/CD?
  \item What disagreements did your group have in this deliverable, if any, and how did
        you resolve them?
\end{enumerate}

\newpage{}

\section*{Appendix --- Team Charter}

\wss{borrows from
  \href{https://engineering.up.edu/industry_partnerships/files/team-charter.pdf}
  {University of Portland Team Charter}}

\subsection*{External Goals}

\wss{What are your team's external goals for this project? These are not the
  goals related to the functionality or quality fo the project.  These are the
  goals on what the team wishes to achieve with the project.  Potential goals are
  to win a prize at the Capstone EXPO, or to have something to talk about in
  interviews, or to get an A+, etc.}

\subsection*{Attendance}

\subsubsection*{Expectations}

\wss{What are your team's expectations regarding meeting attendance (being on
  time, leaving early, missing meetings, etc.)?}

\subsubsection*{Acceptable Excuse}

\wss{What constitutes an acceptable excuse for missing a meeting or a deadline?
  What types of excuses will not be considered acceptable?}

\subsubsection*{In Case of Emergency}

\wss{What process will team members follow if they have an emergency and cannot
  attend a team meeting or complete their individual work promised for a team
  deliverable?}

\subsection*{Accountability and Teamwork}

\subsubsection*{Quality}

\wss{What are your team's expectations regarding the quality
  of team members' preparation for team meetings and the quality of the
  deliverables that members bring to the team?}

\subsubsection*{Attitude}

\wss{What are your team's expectations regarding team members' ideas,
  interactions with the team, cooperation, attitudes, and anything else regarding
  team member contributions?  Do you want to introduce a code of conduct?  Do you
  want a conflict resolution plan?  Can adopt existing codes of conduct.}

\subsubsection*{Stay on Track}

\wss{What methods will be used to keep the team on track? How will your team
  ensure that members contribute as expected to the team and that the team
  performs as expected? How will your team reward members who do well and manage
  members whose performance is below expectations?  What are the consequences for
  someone not contributing their fair share?}

\wss{You may wish to use the project management metrics collected for the TA and
  instructor for this.}

\wss{You can set target metrics for attendance, commits, etc.  What are the
  consequences if someone doesn't hit their targets?  Do they need to bring the
  coffee to the next team meeting?  Does the team need to make an appointment with
  their TA, or the instructor?  Are there incentives for reaching targets early?}

\subsubsection*{Team Building}

\wss{How will you build team cohesion (fun time, group rituals, etc.)? }

\subsubsection*{Decision Making}

\wss{How will you make decisions in your group? Consensus?  Vote? How will you
  handle disagreements? }

\end{document}
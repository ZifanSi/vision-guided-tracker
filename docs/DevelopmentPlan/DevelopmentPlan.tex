\documentclass{article}

\usepackage{booktabs}
\usepackage{tabularx}

\title{Development Plan\\\progname}

\author{\authname}

\date{}

\input{../Comments}
%% Common Parts

\newcommand{\progname}{RoCam} % PUT YOUR PROGRAM NAME HERE
\newcommand{\authname}{Team \#3, SpaceY
  \\ Zifan Si
  \\ Jianqing Liu
  \\ Mike Chen
  \\ Xiaotian Lou} % AUTHOR NAMES                  

\usepackage{hyperref}
\hypersetup{colorlinks=true, linkcolor=blue, citecolor=blue, filecolor=blue,
  urlcolor=blue, unicode=false}
\urlstyle{same}

\usepackage{indentfirst}
\usepackage{graphicx}

\usepackage{titling}

\pretitle{\begin{center}\includegraphics[width=0.5\textwidth]{../../assets/logo/black.png}\\[0.75em]\LARGE}
    \posttitle{\par\end{center}}

\usepackage[letterpaper, portrait, margin=1in]{geometry}

\usepackage{placeins}
\usepackage{float}

\begin{document}

\maketitle

\begin{table}[hp]
  \caption{Revision History} \label{TblRevisionHistory}
  \begin{tabularx}{\textwidth}{llX}
    \toprule
    \textbf{Date}  & \textbf{Developer(s)} & \textbf{Change}                                 \\
    \midrule
    Sept. 10, 2025 & Mike Chen             & Proof-of-Concept Plan and Expected Technology   \\
    Sept. 13, 2025 & Jianqing Liu          & Team Member Roles and Communication Plan        \\
    Sept. 13, 2025 & Jianqing Liu          & Expected Technology and Coding Standard         \\
    Sept. 20, 2025 & Jianqing Liu          & Team Meeting Plan, Workflow Plan and Scheduling \\
    Sept. 21, 2025 & Jianqing Liu          & Misc changes and expedited workflow plan        \\
    \bottomrule
  \end{tabularx}
\end{table}

\newpage{}

This document outlines the development plan for RoCam, a high performance
vision-guided rocket tracker.

\wss{Put your introductory blurb here.  Often the blurb is a brief roadmap of
  what is contained in the report.}

\wss{Additional information on the development plan can be found in the
  \href{https://gitlab.cas.mcmaster.ca/courses/capstone/-/blob/main/Lectures/L02b_POCAndDevPlan/POCAndDevPlan.pdf?ref_type=heads}
  {lecture slides}.}

\section{Confidential Information?}

\wss{State whether your project has confidential information from industry, or
  not.  If there is confidential information, point to the agreement you have in
  place.}

No confidential information to protect.

\wss{For most teams this section will just state that there is no confidential
  information to protect.}
\section{IP to Protect}

\wss{State whether there is IP to protect.  If there is, point to the agreement.
  All students who are working on a project that requires an IP agreement are also
  required to sign the ``Intellectual Property Guide Acknowledgement.''}

No IP to protect.

\section{Copyright License}

\wss{What copyright license is your team adopting.  Point to the license in your
  repo.}

This project adopts the MIT license, which is available at this
\href{https://github.com/ZifanSi/vision-guided-tracker/blob/main/LICENSE}{link}.

\section{Team Meeting Plan}

The team will meet from 7:00–8:00 PM every Tuesday and Saturday. Meetings will
be held online using Discord. If an in-person session is required for hardware
testing, a separate time and location will be scheduled in advance of each
event.

Each team meeting will be structured as follows:

\begin{enumerate}
  \item The agenda for each meeting will be posted as a GitHub issue ahead of time.
  \item Each team member will share task updates (progress, difficulties).
  \item Discuss and distribute new tasks to be worked on.
  \item (Saturdays only) Draft an email for the supervisor summarizing weekly progress and any questions.
\end{enumerate}

Communication with the supervisor will include a weekly update email and
optional online or in-person meetings if either party requests them.

\wss{How often will you meet? where?}

\wss{If the meeting is a physical location (not virtual), out of an abundance of
  caution for safety reasons you shouldn't put the location online}

\wss{How often will you meet with your industry advisor?  when?  where?}

\wss{Will meetings be virtual?  At least some meetings should likely be
  in-person.}

\wss{How will the meetings be structured?  There should be a chair for all meetings.  There should be an agenda for all meetings.}

\section{Team Communication Plan}

\begin{itemize}
  \item \textbf{Discord}: General team communication, informal discussions, quick updates, and meetings.
  \item \textbf{GitHub}: Code-related discussions, project management, and meeting notes.
  \item \textbf{Zoom}: Meetings with supervisor.
  \item \textbf{Email}: Weekly progress updates to supervisor.

\end{itemize}

\wss{Issues on GitHub should be part of your communication plan.}

\section{Team Member Roles}

\wss{You should identify the types of roles you anticipate, like notetaker,
  leader, meeting chair, reviewer.  Assigning specific people to those roles is
  not necessary at this stage.  In a student team the role of the individuals will
  likely change throughout the year.}

\begin{itemize}
  \item \textbf{Project Manager}: Oversees project timeline, coordinates tasks between team members, manages deliverables and deadlines, and serves as primary point of contact with supervisor and stakeholders.
  \item \textbf{Meeting Chair}: Leads team meetings, prepares agendas, ensures discussions stay on track, facilitates decision-making, and manages meeting time effectively.
  \item \textbf{Notetaker}: Records meeting minutes, tracks action items and decisions, maintains documentation of team discussions, and distributes meeting summaries to all members.
  \item \textbf{Quality Assurance}: Reviews code, documentation, and deliverables for quality and consistency, conducts testing and validation, ensures adherence to coding standards and project requirements, and manages the review process for all team outputs.
\end{itemize}

\section{Workflow Plan}

\subsection{Normal Features}

\begin{enumerate}
  \item Planning
        \begin{enumerate}
          \item Create an issue in GitHub Projects under "Backlog" using an appropriate
                template (bug or enhancement), and assign it to the correct subproject.
          \item Backlog issues will be discussed during meetings to refine scope and
                requirements.
          \item If the issue is approved for development, assign an owner and a deadline, then
                move it into "Todo".
        \end{enumerate}
  \item Developing
        \begin{enumerate}
          \item The assignee will work on the task in a new branch:
                [main-contributor-name]/[feature-name].
          \item Move the issue into "In Progress".
          \item Create a pull request once the code is ready for review. The pull request
                should reference the original issue.
          \item Request a review from at least one team member and ping them on Discord.
        \end{enumerate}
  \item Reviewing
        \begin{enumerate}
          \item Move the issue into "In Review".
          \item The reviewer may comment or commit directly to the feature branch.
          \item The reviewer approves and merges the pull request.
          \item Delete the feature branch after the pull request is merged.
          \item Move the issue into "Done".
        \end{enumerate}
\end{enumerate}

\subsection{Minor Changes (Expedited Workflow)}

For small, low-risk tasks (e.g., fixing typos or minor UI adjustments), the
normal workflow may be skipped to reduce overhead. Creating an issue on the
GitHub Projects board is not required. Instead:
\begin{enumerate}
  \item Open a pull request against the main branch with "minor" label attached.
  \item Obtain at least one reviewer approval before merging.
  \item Delete the feature branch after the pull request is merged.
\end{enumerate}

\section{Project Decomposition and Scheduling}

This project is decomposed into the following subprojects:

\begin{itemize}
  \item \textbf{Web App}: responsible for remote management.
  \item \textbf{CV Pipeline}: responsible for locating the rocket.
  \item \textbf{Motion Control}: responsible for controlling gimbal movement.
\end{itemize}

All code for the subprojects, along with documentation, is centralized in a
single monorepo \href{https://github.com/ZifanSi/vision-guided-tracker}{here}.
GitHub Projects is used for project management and can be accessed
\href{https://github.com/users/ZifanSi/projects/1}{here}.

While development will be broken down into smaller features with individual
deadlines, the overall project will follow the major deadlines below.

\FloatBarrier
\begin{table}[h]
  \caption{Major Deliverables} \label{TblMajorDeliverables}
  \begin{tabularx}{\textwidth}{llX}
    \toprule
    \textbf{Date}    & \textbf{Deliverable}                          & \textbf{Files}                                                                                                                                   \\
    \midrule
    Sept. 22, 2025   & Problem Statement, POC Plan, Development Plan & \href{https://github.com/ZifanSi/vision-guided-tracker/blob/main/docs/ProblemStatementAndGoals/ProblemStatement.pdf}{Problem Statement} \newline
    \href{https://github.com/ZifanSi/vision-guided-tracker/blob/main/docs/DevelopmentPlan/DevelopmentPlan.pdf}{POC and Development Plan}                                                                                \\
    Oct. 6, 2025     & Req. Doc. and Hazard Analysis Revision 0      & \href{https://github.com/ZifanSi/vision-guided-tracker/blob/main/docs/SRS/SRS.pdf}{Req. Doc.}    \newline
    \href{https://github.com/ZifanSi/vision-guided-tracker/blob/main/docs/HazardAnalysis/HazardAnalysis.pdf}{Hazard Analysis}                                                                                           \\
    Oct. 27, 2025    & V\&V Plan Revision 0                          & \href{https://github.com/ZifanSi/vision-guided-tracker/blob/main/docs/VnVPlan/VnVPlan.pdf}{V\&V Plan}                                            \\
    Nov. 10, 2025    & Design Document Revision -1                   & \href{https://github.com/ZifanSi/vision-guided-tracker/blob/main/docs/Design/README.md}{Design Document}                                         \\
    Nov. 17-28, 2025 & Proof of Concept Demonstration                &                                                                                                                                                  \\
    Jan. 19, 2026    & Design Document Revision 0                    & \href{https://github.com/ZifanSi/vision-guided-tracker/blob/main/docs/Design/README.md}{Design Document}                                         \\
    Feb. 2-13, 2026  & Revision 0 Demonstration                      &                                                                                                                                                  \\
    Mar. 9, 2026     & V\&V Report and Extras Revision 0             & \href{https://github.com/ZifanSi/vision-guided-tracker/blob/main/docs/VnVReport/VnVReport.pdf}{V\&V Report}   \newline
    \href{https://github.com/ZifanSi/vision-guided-tracker/tree/main/docs/Extras}{Extras}
    \\
    Mar. 23-29, 2026 & Final Demonstration (Revision 1)              &                                                                                                                                                  \\
    TBD              & EXPO Demonstration                            &                                                                                                                                                  \\
    Apr. 6, 2026     & Final Documentation (Revision 1)              &                                                                                                                                                  \\
    \bottomrule
  \end{tabularx}
\end{table}
\FloatBarrier

\wss{How will the project be scheduled?  This is the big picture schedule, not
  details. You will need to reproduce information that is in the course outline
  for deadlines.}

\section{Proof-of-Concept Demonstration Plan}

The following are the planned steps of the POC:

\begin{enumerate}
  \item Acquire an initial image from a camera with a stationary target.
  \item Activate the system's tracking mode. It will segment and detect multiple moving
        objects in the image.
        \begin{itemize}
          \item Segmentation and moving-object detection will use computer vision techniques.
          \item The computer vision model will be deployed on a Jetson Nano.
        \end{itemize}
  \item The user selects a stationary or moving object as the target.
  \item As the target moves, the system keeps it centered in the image for smooth
        tracking.
        \begin{itemize}
          \item The system will handle occlusion and temporary loss of the target.
          \item The user can manually reselect the target if needed.
          \item Real-time camera control will be implemented using an STM32 microcontroller.
        \end{itemize}
\end{enumerate}

The following is a list of primary risks to consider for the POC:
\begin{enumerate}
  \item The computer vision system may not process images at a sufficient frame rate.
        \begin{itemize}
          \item If this occurs, we will optimize the existing model, consider using a more
                powerful board, lower the frame rate, or use a traditional algorithmic approach
                that detects motion via pixel-wise image comparison.
          \item We also reserve the option to use models trained for a specific set of objects
                to increase throughput.
        \end{itemize}
  \item The STM32 may not deliver real-time control to the camera.
        \begin{itemize}
          \item If this occurs, we will consider using a more powerful microcontroller, or a
                different control technique.
        \end{itemize}
  \item Integration of the frontend, backend, computer vision model, and
        microcontroller may be more difficult than expected.
        \begin{itemize}
          \item If this occurs, we will consider using a more powerful integrated single-board
                computer, deploying via cloud computation, or redesigning our control flow to
                simplify integration.
        \end{itemize}
\end{enumerate}

Other smaller risks to consider:
\begin{enumerate}
  \item UI usability issues: The user interface may not be intuitive or easy to use,
        leading to user frustration or errors.
        \begin{itemize}
          \item {Potential solution}: Conduct user testing and gather feedback to improve the interface design.
        \end{itemize}
\end{enumerate}

\section{Expected Technology}

\wss{What programming language or languages do you expect to use?  What external
  libraries?  What frameworks?  What technologies.  Are there major components of
  the implementation that you expect you will implement, despite the existence of
  libraries that provide the required functionality.  For projects with machine
  learning, will you use pre-trained models, or be training your own model?  }

\wss{The implementation decisions can, and likely will, change over the course
  of the project.  The initial documentation should be written in an abstract way;
  it should be agnostic of the implementation choices, unless the implementation
  choices are project constraints.  However, recording our initial thoughts on
  implementation helps understand the challenge level and feasibility of a
  project.  It may also help with early identification of areas where project
  members will need to augment their training.}

\wss{git, GitHub and GitHub projects should be part of your technology.}

% Topics to discuss include the following:

% \begin{itemize}
%   \item Specific programming language
%   \item Specific libraries
%   \item Pre-trained models
%   \item Specific linter tool (if appropriate)
%   \item Specific unit testing framework
%   \item Investigation of code coverage measuring tools
%   \item Specific plans for Continuous Integration (CI), or an explanation that CI is
%         not being done
%   \item Specific performance measuring tools (like Valgrind), if appropriate
%   \item Tools you will likely be using?
% \end{itemize}

\begin{itemize}
  \item Motion Control
        \begin{itemize}
          \item STM32 Microcontroller
          \item Language: Rust
          \item Framework: embassy-rs
          \item Formatting: rustfmt
          \item Linter: rust-clippy
          \item Unit Testing: Rust built-in
          \item Code Coverage: grcov
        \end{itemize}
  \item Computer Vision
        \begin{itemize}
          \item NVIDIA Jetson
          \item NVIDIA JetPack
          \item Language: Python
          \item Libraries: OpenCV, NumPy, Matplotlib, Torch
          \item Open Source Models: Ultralytics YOLO, SAM, various ViTs
          \item Formatting: Black
          \item Linter: pylint and ruff
          \item Unit Testing: pytest
          \item Code Coverage: coverage
          \item Containerized using Docker
        \end{itemize}
  \item Web App
        \begin{itemize}
          \item Web Server: Flask
          \item Language: TypeScript
          \item Framework: React
          \item Formatting: prettier
          \item Linter: eslint
          \item End-to-end Testing: Cypress
        \end{itemize}
  \item All of the above will use GitHub Actions for CI.
  \item Development Tools
        \begin{itemize}
          \item VS Code
          \item PyCharm
          \item Git
          \item GitHub
        \end{itemize}
\end{itemize}

\section{Coding Standard}

\begin{itemize}
  \item Rust:
        \href{https://en.wikipedia.org/wiki/The_Power_of_10:_Rules_for_Developing_Safety-Critical_Code}{The
          Power of 10 Rules}
  \item Python: \href{https://peps.python.org/pep-0008/}{PEP 8}
  \item TypeScript:
        \href{https://typescript-eslint.io/packages/typescript-eslint}{typescript-eslint}

\end{itemize}

\wss{What coding standard will you adopt?}

\newpage{}

\section*{Appendix --- Reflection}

\wss{Not required for CAS 741}

\input{../Reflection.tex}

\begin{enumerate}
  \item Why is it important to create a development plan prior to starting the project?
  \item In your opinion, what are the advantages and disadvantages of using CI/CD?
  \item What disagreements did your group have in this deliverable, if any, and how did
        you resolve them?
\end{enumerate}

\newpage{}

\section*{Appendix --- Team Charter}

\wss{borrows from
  \href{https://engineering.up.edu/industry_partnerships/files/team-charter.pdf}
  {University of Portland Team Charter}}

\subsection*{External Goals}

\wss{What are your team's external goals for this project? These are not the
  goals related to the functionality or quality fo the project.  These are the
  goals on what the team wishes to achieve with the project.  Potential goals are
  to win a prize at the Capstone EXPO, or to have something to talk about in
  interviews, or to get an A+, etc.}

\subsection*{Attendance}

\subsubsection*{Expectations}

\wss{What are your team's expectations regarding meeting attendance (being on
  time, leaving early, missing meetings, etc.)?}

\subsubsection*{Acceptable Excuse}

\wss{What constitutes an acceptable excuse for missing a meeting or a deadline?
  What types of excuses will not be considered acceptable?}

\subsubsection*{In Case of Emergency}

\wss{What process will team members follow if they have an emergency and cannot
  attend a team meeting or complete their individual work promised for a team
  deliverable?}

\subsection*{Accountability and Teamwork}

\subsubsection*{Quality}

\wss{What are your team's expectations regarding the quality
  of team members' preparation for team meetings and the quality of the
  deliverables that members bring to the team?}

\subsubsection*{Attitude}

\wss{What are your team's expectations regarding team members' ideas,
  interactions with the team, cooperation, attitudes, and anything else regarding
  team member contributions?  Do you want to introduce a code of conduct?  Do you
  want a conflict resolution plan?  Can adopt existing codes of conduct.}

\subsubsection*{Stay on Track}

\wss{What methods will be used to keep the team on track? How will your team
  ensure that members contribute as expected to the team and that the team
  performs as expected? How will your team reward members who do well and manage
  members whose performance is below expectations?  What are the consequences for
  someone not contributing their fair share?}

\wss{You may wish to use the project management metrics collected for the TA and
  instructor for this.}

\wss{You can set target metrics for attendance, commits, etc.  What are the
  consequences if someone doesn't hit their targets?  Do they need to bring the
  coffee to the next team meeting?  Does the team need to make an appointment with
  their TA, or the instructor?  Are there incentives for reaching targets early?}

\subsubsection*{Team Building}

\wss{How will you build team cohesion (fun time, group rituals, etc.)? }

\subsubsection*{Decision Making}

\wss{How will you make decisions in your group? Consensus?  Vote? How will you
  handle disagreements? }

\end{document}

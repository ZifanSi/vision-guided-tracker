\documentclass{article}
\pdfinfoomitdate=1
\pdftrailerid{}

\usepackage{booktabs}
\usepackage{tabularx}

\title{Development Plan\\\progname}

\author{\authname}

\date{}

\input{../Comments}
%% Common Parts

\newcommand{\progname}{RoCam} % PUT YOUR PROGRAM NAME HERE
\newcommand{\authname}{Team \#3, SpaceY
  \\ Zifan Si
  \\ Jianqing Liu
  \\ Mike Chen
  \\ Xiaotian Lou} % AUTHOR NAMES                  

\usepackage{hyperref}
\hypersetup{colorlinks=true, linkcolor=blue, citecolor=blue, filecolor=blue,
  urlcolor=blue, unicode=false}
\urlstyle{same}

\usepackage{indentfirst}
\usepackage{graphicx}

\usepackage{titling}

\pretitle{\begin{center}\includegraphics[width=0.5\textwidth]{../../assets/logo/black.png}\\[0.75em]\LARGE}
    \posttitle{\par\end{center}}

\usepackage[letterpaper, portrait, margin=1in]{geometry}

\usepackage{placeins}
\usepackage{float}

\begin{document}

\maketitle

\begin{table}[hp]
  \caption{Revision History} \label{TblRevisionHistory}
  \begin{tabularx}{\textwidth}{llX}
    \toprule
    \textbf{Date}  & \textbf{Developer(s)} & \textbf{Change}                                 \\
    \midrule
    Sept. 10, 2025 & Mike Chen             & Proof-of-Concept Plan and Expected Technology   \\
    Sept. 13, 2025 & Jianqing Liu          & Team Member Roles and Communication Plan        \\
    Sept. 13, 2025 & Jianqing Liu          & Expected Technology and Coding Standard         \\
    Sept. 15, 2025 & Xiaotian Lou          & Coding Standard for Python                      \\
    Sept. 20, 2025 & Jianqing Liu          & Team Meeting Plan, Workflow Plan and Scheduling \\
    Sept. 21, 2025 & Xiaotian Lou          & Team Charter                                    \\
    Sept. 21, 2025 & Jianqing Liu          & Misc changes and expedited workflow plan        \\
    Sept. 22, 2025 & Xiaotian Lou          & change on Team Charter                          \\
    Sept. 22, 2025 & Zifan Si              & Fix development plan based on lecture slides    \\
    Sept. 22, 2025 & Xiaotian Lou          & update reflection                               \\
    Sept. 22, 2025 & Zifan Si              & Update reflection and fix format                \\
    \bottomrule
  \end{tabularx}
\end{table}

\newpage{}

This document outlines the development plan for RoCam, a high performance
vision-guided rocket tracker.

\wss{Put your introductory blurb here.  Often the blurb is a brief roadmap of
  what is contained in the report.}

\wss{Additional information on the development plan can be found in the
  \href{https://gitlab.cas.mcmaster.ca/courses/capstone/-/blob/main/Lectures/L02b_POCAndDevPlan/POCAndDevPlan.pdf?ref_type=heads}
  {lecture slides}.}

\section{Confidential Information?}

\wss{State whether your project has confidential information from industry, or
  not.  If there is confidential information, point to the agreement you have in
  place.}

No confidential information to protect.

\wss{For most teams this section will just state that there is no confidential
  information to protect.}
\section{IP to Protect}

\wss{State whether there is IP to protect.  If there is, point to the agreement.
  All students who are working on a project that requires an IP agreement are also
  required to sign the ``Intellectual Property Guide Acknowledgement.''}

No IP to protect.

\section{Copyright License}

\wss{What copyright license is your team adopting.  Point to the license in your
  repo.}

This project adopts the MIT license, which is available at this
\href{https://github.com/ZifanSi/vision-guided-tracker/blob/main/LICENSE}{link}.

\section{Team Meeting Plan}

The team will meet from 7:00–8:00 PM every Tuesday and Saturday. Meetings will
be held online using Discord. If an in-person session is required for hardware
testing, a separate time and location will be scheduled in advance of each
event.

Each team meeting will be structured as follows:

\begin{enumerate}
  \item The agenda for each meeting will be posted as a GitHub issue ahead of time.
  \item Each team member will share task updates (progress, difficulties).
  \item Discuss and distribute new tasks to be worked on.
  \item (Saturdays only) Draft an email for the supervisor summarizing weekly progress and any questions.
\end{enumerate}

Communication with the supervisor will include a weekly update email and
optional online or in-person meetings if either party requests them.

% Justification for P4
\textcolor{blue}{Justification:} Meeting twice a week gives us enough touchpoints
to stay aligned without eating into development time. By keeping careful notes
and sharing them after each meeting, everyone stays up to date—even if someone
can’t attend—and we avoid the need for extra meetings that would slow progress.

\wss{How often will you meet? where?}

\wss{If the meeting is a physical location (not virtual), out of an abundance of
  caution for safety reasons you shouldn't put the location online}

\wss{How often will you meet with your industry advisor?  when?  where?}

\wss{Will meetings be virtual?  At least some meetings should likely be
  in-person.}

\wss{How will the meetings be structured?  There should be a chair for all meetings.  There should be an agenda for all meetings.}

\section{Team Communication Plan}

\begin{itemize}
  \item \textbf{Discord}: General team communication, informal discussions, quick updates, and meetings.
  \item \textbf{GitHub}: Code-related discussions, project management, and meeting notes.
  \item \textbf{Zoom}: Meetings with supervisor.
  \item \textbf{Email}: Weekly progress updates to supervisor.

\end{itemize}

% Justification for P5
\textcolor{blue}{Justification:} These four channels cover everything we need without
overlap—fast chat (Discord), structured task tracking (GitHub), live supervisor
discussions (Zoom), and formal updates (Email). Using more tools would only add
confusion and split conversations across too many places, while using fewer
would leave gaps in how we share information.

\wss{Issues on GitHub should be part of your communication plan.}

\section{Team Member Roles}

\wss{You should identify the types of roles you anticipate, like notetaker,
  leader, meeting chair, reviewer.  Assigning specific people to those roles is
  not necessary at this stage.  In a student team the role of the individuals will
  likely change throughout the year.}
  
The following roles have been defined for the team. These roles are not fixed and may
rotate as the project progresses to balance workload and skill development.
However, an initial assignment has been made based on each member’s strengths and
availability.

\begin{itemize}
  \item \textbf{Project Manager}: Oversees project timeline, coordinates tasks between team members, manages deliverables and deadlines, and serves as primary point of contact with supervisor and stakeholders.
  \item \textbf{Meeting Chair}: Leads team meetings, prepares agendas, ensures discussions stay on track, facilitates decision-making, and manages meeting time effectively.
  \item \textbf{Notetaker}: Records meeting minutes, tracks action items and decisions, maintains documentation of team discussions, and distributes meeting summaries to all members.
  \item \textbf{Quality Assurance}: Reviews code, documentation, and deliverables for quality and consistency, conducts testing and validation, ensures adherence to coding standards and project requirements, and manages the review process for all team outputs.
\end{itemize}

% Justification for P6
\textcolor{blue}{Justification:} These four roles cover everything our team needs:
someone to coordinate and manage progress (Project Manager), someone to run
organized meetings (Meeting Chair), someone to keep accurate records (Notetaker),
and someone to check the quality of our work (Quality Assurance). Together,
they cover planning, communication, documentation, and review, so nothing
important is left out.

\section{Workflow Plan}

\subsection{Normal Features}

\begin{enumerate}
  \item Planning
        \begin{enumerate}
          \item Create an issue in GitHub Projects under "Backlog" using an appropriate
                template (bug or enhancement), and assign it to the correct subproject.
          \item Backlog issues will be discussed during meetings to refine scope and
                requirements.
          \item If the issue is approved for development, assign an owner and a deadline, then
                move it into "Todo".
        \end{enumerate}
  \item Developing
        \begin{enumerate}
          \item The assignee will work on the task in a new branch:
                [main-author-name]/[feature-name].
          \item Move the issue into "In Progress".
          \item (optional) If the author wants to get feedback on the code before all the
                changes are complete, they can create a draft pull request and request for
                review.
          \item Create a pull request once the code is ready for review. The pull request
                should reference the original issue.
          \item Request a review from at least one team member and ping them on Discord.
        \end{enumerate}
  \item Reviewing
        \begin{enumerate}
          \item Move the issue into "In Review".
          \item The reviewer may comment or commit directly to the feature branch.
          \item The reviewer approves and merges the pull request.
          \item Delete the feature branch after the pull request is merged.
          \item Move the issue into "Done".
        \end{enumerate}
\end{enumerate}

\subsection{Minor Changes (Expedited Workflow)}

For small, low-risk tasks (e.g., fixing typos or minor UI adjustments), the
normal workflow may be skipped to reduce overhead. Creating an issue on the
GitHub Projects board is not required. Instead:
\begin{enumerate}
  \item Open a pull request against the main branch with "minor" label attached.
  \item Obtain at least one reviewer approval before merging.
  \item Delete the feature branch after the pull request is merged.
\end{enumerate}

% Justification for P7
\textcolor{blue}{Justification:} This workflow is enough to keep our work organized
and traceable without being too heavy. Every task starts as an issue so nothing
is forgotten, branches keep code changes separate, and pull requests with reviews
make sure mistakes get caught early. Using GitHub Projects ties everything together,
so planning, coding, and reviewing all stay in one place. More complicated processes
would just slow us down, while skipping steps would risk bugs and confusion.

\section{Project Decomposition and Scheduling}

This project is decomposed into the following subprojects:

\begin{itemize}
  \item \textbf{Web App}: responsible for remote management.
  \item \textbf{CV Pipeline}: responsible for locating the rocket.
  \item \textbf{Motion Control}: responsible for controlling gimbal movement.
\end{itemize}

All code for the subprojects, along with documentation, is centralized in a
single monorepo \href{https://github.com/ZifanSi/vision-guided-tracker}{here}.
GitHub Projects is used for project management and can be accessed
\href{https://github.com/users/ZifanSi/projects/1}{here}.

While development will be broken down into smaller features with individual
deadlines, the overall project will follow the major deadlines below.

\FloatBarrier
\begin{table}[h]
  \caption{Major Deliverables} \label{TblMajorDeliverables}
  \begin{tabularx}{\textwidth}{llX}
    \toprule
    \textbf{Date}    & \textbf{Deliverable}                          & \textbf{Files}                                                                                                                                   \\
    \midrule
    Sept. 22, 2025   & Problem Statement, POC Plan, Development Plan & \href{https://github.com/ZifanSi/vision-guided-tracker/blob/main/docs/ProblemStatementAndGoals/ProblemStatement.pdf}{Problem Statement} \newline
    \href{https://github.com/ZifanSi/vision-guided-tracker/blob/main/docs/DevelopmentPlan/DevelopmentPlan.pdf}{POC and Development Plan}                                                                                \\
    Oct. 6, 2025     & Req. Doc. and Hazard Analysis Revision 0      & \href{https://github.com/ZifanSi/vision-guided-tracker/blob/main/docs/SRS/SRS.pdf}{Req. Doc.}    \newline
    \href{https://github.com/ZifanSi/vision-guided-tracker/blob/main/docs/HazardAnalysis/HazardAnalysis.pdf}{Hazard Analysis}                                                                                           \\
    Oct. 27, 2025    & V\&V Plan Revision 0                          & \href{https://github.com/ZifanSi/vision-guided-tracker/blob/main/docs/VnVPlan/VnVPlan.pdf}{V\&V Plan}                                            \\
    Nov. 10, 2025    & Design Document Revision -1                   & \href{https://github.com/ZifanSi/vision-guided-tracker/blob/main/docs/Design/README.md}{Design Document}                                         \\
    Nov. 17-28, 2025 & Proof of Concept Demonstration                &                                                                                                                                                  \\
    Jan. 19, 2026    & Design Document Revision 0                    & \href{https://github.com/ZifanSi/vision-guided-tracker/blob/main/docs/Design/README.md}{Design Document}                                         \\
    Feb. 2-13, 2026  & Revision 0 Demonstration                      &                                                                                                                                                  \\
    Mar. 9, 2026     & V\&V Report and Extras Revision 0             & \href{https://github.com/ZifanSi/vision-guided-tracker/blob/main/docs/VnVReport/VnVReport.pdf}{V\&V Report}   \newline
    \href{https://github.com/ZifanSi/vision-guided-tracker/tree/main/docs/Extras}{Extras}
    \\
    Mar. 23-29, 2026 & Final Demonstration (Revision 1)              &                                                                                                                                                  \\
    TBD              & EXPO Demonstration                            &                                                                                                                                                  \\
    Apr. 6, 2026     & Final Documentation (Revision 1)              &                                                                                                                                                  \\
    \bottomrule
  \end{tabularx}
\end{table}
\FloatBarrier

% Justification for P8
\textcolor{blue}{Justification:} Splitting the work into Web App, CV Pipeline,
and Motion Control is enough because these three parts match the main technical
challenges of the project. Each subproject has clear boundaries, so team members
can work in parallel without confusion. Putting everything in a single monorepo
keeps the codebase consistent, and using GitHub Projects ties scheduling directly
to the actual code, so we don’t need extra tools that could complicate tracking.

\wss{How will the project be scheduled?  This is the big picture schedule, not
  details. You will need to reproduce information that is in the course outline
  for deadlines.}

\section{Proof-of-Concept Demonstration Plan}

The following are the planned steps of the POC:

\begin{enumerate}
  \item Acquire an initial image from a camera with a stationary target.
  \item Activate the system's tracking mode. It will segment and detect multiple moving
        objects in the image.
        \begin{itemize}
          \item Segmentation and moving-object detection will use computer vision techniques.
          \item The computer vision model will be deployed on a Jetson Nano.
        \end{itemize}
  \item The user selects a stationary or moving object as the target.
  \item As the target moves, the system keeps it centered in the image for smooth
        tracking.
        \begin{itemize}
          \item The system will handle occlusion and temporary loss of the target.
          \item The user can manually reselect the target if needed.
          \item Real-time camera control will be implemented using an STM32 microcontroller.
        \end{itemize}
\end{enumerate}

The following is a list of primary risks to consider for the POC:
\begin{enumerate}
  \item The computer vision system may not process images at a sufficient frame rate.
        \begin{itemize}
          \item If this occurs, we will optimize the existing model, consider using a more
                powerful board, lower the frame rate, or use a traditional algorithmic approach
                that detects motion via pixel-wise image comparison.
          \item We also reserve the option to use models trained for a specific set of objects
                to increase throughput.
        \end{itemize}
  \item The STM32 may not deliver real-time control to the camera.
        \begin{itemize}
          \item If this occurs, we will consider using a more powerful microcontroller, or a
                different control technique.
        \end{itemize}
  \item Integration of the frontend, backend, computer vision model, and
        microcontroller may be more difficult than expected.
        \begin{itemize}
          \item If this occurs, we will consider using a more powerful integrated single-board
                computer, deploying via cloud computation, or redesigning our control flow to
                simplify integration.
        \end{itemize}
\end{enumerate}

Other smaller risks to consider:
\begin{enumerate}
  \item UI usability issues: The user interface may not be intuitive or easy to use,
        leading to user frustration or errors.
        \begin{itemize}
          \item {Potential solution}: Conduct user testing and gather feedback to improve the interface design.
        \end{itemize}
\end{enumerate}

% Justification for P9
\textcolor{blue}{Justification:} These steps give us just enough to prove the idea works
without overloading the demo. Taking a picture, tracking movement, picking a target,
and keeping it in view are the core things we need to show. We also listed the biggest
risks so we’re ready if something goes wrong. This way the POC is doable, but it still
shows the hardest and most important parts of the system.

\section{Expected Technology}

\wss{What programming language or languages do you expect to use?  What external
  libraries?  What frameworks?  What technologies.  Are there major components of
  the implementation that you expect you will implement, despite the existence of
  libraries that provide the required functionality.  For projects with machine
  learning, will you use pre-trained models, or be training your own model?  }

\wss{The implementation decisions can, and likely will, change over the course
  of the project.  The initial documentation should be written in an abstract way;
  it should be agnostic of the implementation choices, unless the implementation
  choices are project constraints.  However, recording our initial thoughts on
  implementation helps understand the challenge level and feasibility of a
  project.  It may also help with early identification of areas where project
  members will need to augment their training.}

\wss{git, GitHub and GitHub projects should be part of your technology.}

% Topics to discuss include the following:

% \begin{itemize}
%   \item Specific programming language
%   \item Specific libraries
%   \item Pre-trained models
%   \item Specific linter tool (if appropriate)
%   \item Specific unit testing framework
%   \item Investigation of code coverage measuring tools
%   \item Specific plans for Continuous Integration (CI), or an explanation that CI is
%         not being done
%   \item Specific performance measuring tools (like Valgrind), if appropriate
%   \item Tools you will likely be using?
% \end{itemize}

\begin{itemize}
  \item Motion Control
        \begin{itemize}
          \item STM32 Microcontroller
          \item Language: Rust
          \item Framework: embassy-rs
          \item Formatting: rustfmt
          \item Linter: rust-clippy
          \item Unit Testing: Rust built-in
          \item Code Coverage: grcov
        \end{itemize}
  \item Computer Vision
        \begin{itemize}
          \item NVIDIA Jetson
          \item NVIDIA JetPack
          \item Language: Python
          \item Libraries: OpenCV, NumPy, Matplotlib, Torch
          \item Open Source Models: Ultralytics YOLO, SAM, various ViTs
          \item Formatting: Black
          \item Linter: pylint and ruff
          \item Unit Testing: pytest
          \item Code Coverage: coverage
          \item Containerized using Docker
        \end{itemize}
  \item Web App
        \begin{itemize}
          \item Web Server: Flask
          \item Language: TypeScript
          \item Framework: React
          \item Formatting: prettier
          \item Linter: eslint
          \item End-to-end Testing: Cypress
        \end{itemize}
  \item All of the above will use GitHub Actions for CI.
  \item Development Tools
        \begin{itemize}
          \item VS Code
          \item PyCharm
          \item Git
          \item GitHub
        \end{itemize}
\end{itemize}

\section{Coding Standard}

\begin{itemize}
  \item Rust:
        \href{https://en.wikipedia.org/wiki/The_Power_of_10:_Rules_for_Developing_Safety-Critical_Code}{The
          Power of 10 Rules}
  \item Python: \href{https://peps.python.org/pep-0008/}{PEP 8}
  \item TypeScript:
        \href{https://typescript-eslint.io/packages/typescript-eslint}{typescript-eslint}

\end{itemize}

% Justification for P11
\textcolor{blue}{Justification:} These standards are enough to keep our code clean
and consistent without adding extra rules that slow us down. The Power of 10 helps
make Rust code safer, PEP 8 is the common style guide for Python, and
typescript-eslint is widely used for TypeScript. Sticking to these well-known
guides means anyone reading or reviewing our code will find it easy to follow.

\wss{What coding standard will you adopt?}

\newpage{}

\section*{Appendix --- Reflection}

\wss{Not required for CAS 741}

\input{../Reflection.tex}

\begin{enumerate}
  \item Why is it important to create a development plan prior to starting the project?
  \item In your opinion, what are the advantages and disadvantages of using CI/CD?
  \item What disagreements did your group have in this deliverable, if any, and how did
        you resolve them?
\end{enumerate}

\subsection*{Development Plan Reflection (Hardware) --- Jianqing Liu}

\begin{enumerate}
  \item \textit{Why is it important to create a development plan prior to starting the project?}

        I'm responsible for drafting most of the sections of this deliverable except
        Proof-of-Concept Demonstration Plan and part of Expected Technology. One
        section I want to highlight is "7.2 Minor Changes (Expedited Workflow)". I got
        this idea from my time working in a coop position, where the high overhead of
        project management prevented me from fixing small issues I encountered during
        development, as the fix was unrelated to the feature I was working on. I hope
        this will not turn into a loophole where all the pull requests are labeled as
        "minor" to bypass the normal planning process.

        I also didn’t realize the usefulness of the development plan until I started
        working on it. It makes a lot of implicit assumptions explicit (especially the
        workflow plan and team meeting plan, which were only communicated verbally in
        my previous positions), and helps the team better collaborate.

  \item \textit{In your opinion, what are the advantages and disadvantages of using CI/CD?}

        The advantages of CI far outweigh the disadvantages. The most important use for
        it in my opinion is to enforce coding standards and catch regressions early.
        The only disadvantage I can think of is the extra overhead of setting up CI,
        but it is worth it.

        CD is a nice to have, but sometimes not feasible. For our project, since we
        need to deploy to actual hardware (Jetson and STM32), it is not feasible to set
        up CD. However, we can create a deployment script to simplify the process.

  \item \textit{What disagreements did your group have in this deliverable, if any, and how did you resolve them?}

        We did not have any disagreements in this deliverable.
\end{enumerate}

\subsection*{Development Plan Reflection (QA) --- Xiaotian Lou}

\begin{enumerate}
  \item \textit{Why is it important to create a development plan prior to starting the project?}

        I am the QA for this team, so having a clear plan is very important to me. A
        plan shows problems early and makes sure everyone knows their role. It also
        gives me a way to see when and how to check quality during the project. By
        connecting goals to tasks, timelines, and risks, the plan keeps the team on the
        same page and makes it easier to track progress.

  \item \textit{In your opinion, what are the advantages and disadvantages of using CI/CD?}

        From a QA view, CI/CD is very helpful. It runs tests on every change, catches
        bugs early, and makes sure the code stays consistent. This keeps the project
        stable and saves time later when fixing problems. The downside is that it takes
        effort to set up, and if the pipeline is not correct, it can slow things down.
        But overall, the benefits are much bigger than the drawbacks.

  \item \textit{What disagreements did your group have in this deliverable, if any, and how did you resolve them?}

        We did not have big disagreements. The only small issue is that sometimes the
        team is very optimistic, which can hide risks. To fix this, we added schedule
        buffers, milestone checkpoints, and a short risk log. This helps us stay
        positive but also realistic, so we are ready if something takes longer than
        expected.
\end{enumerate}

\subsection*{Development Plan Reflection (CV) --- Shike Chen}

\begin{enumerate}
  \item \textit{Why is it important to create a development plan prior to starting the project?}

        I have helped in drafting and revising the development plan. I realized the
        importance of the development plan as I was working on the proof of concept. In
        the process of writing the proof of concept, I went through many risks of the
        project that I had not thought of before. I believe the development plan is
        vital at the beginning of the project, as it helps the team identify the risks
        and feasibility of the project.

  \item \textit{In your opinion, what are the advantages and disadvantages of using CI/CD?}

        CI/CD is useful because it lets us add small changes, fixes, and updates
        quickly. For CV, this means things like models or scripts can be tested and
        merged right away, so we know they work. This makes the system stronger and
        easier to improve. The downside is that setup takes time, and if something bad
        gets into the main branch it can cause issues. But overall, the benefits are
        much bigger than the downsides.

  \item \textit{What disagreements did your group have in this deliverable, if any, and how did you resolve them?}

        We did not have any disagreements in this deliverable.

\end{enumerate}

\subsection*{Development Plan Reflection (Web) --- Zifan Si}

\begin{enumerate}
  \item \textit{Why is it important to create a development plan prior to starting the project?}

        I worked on the web side of the project, making sure the frontend and tech
        stack were included. While writing, I saw how much the web app depends on the
        backend on the Jetson. Things like speed, bandwidth, and deployment all affect
        how the frontend feels. Having a plan made these links clear early and kept the
        team on the same page. I also learned that web work here isn’t just about the
        frontend --- hardware and network limits change the design, so I had to think
        about the whole system, not just my part.

  \item \textit{In your opinion, what are the advantages and disadvantages of using CI/CD?}

        From the web side, CI is helpful because it can run tests on the React code and
        API calls, catching small bugs before they reach users. It also keeps the
        frontend consistent as new features are added. The downside is that setup takes
        time, and since we depend on the backend and Jetson hardware, a broken pipeline
        can also break the web side. Still, it makes development faster and more
        reliable overall.

  \item \textit{What disagreements did your group have in this deliverable, if any, and how did you resolve them?}

        We did not have big disagreements. The only question from my end was how much
        detail to write for the web app, like API design for a complex pipeline. Since
        a lot depends on backend performance, I decided to save those details for a
        later phase and focus on clear communication with the team.
\end{enumerate}

\newpage{}
\section*{Appendix --- Team Charter}

\wss{borrows from
  \href{https://github.com/ssm-lab/capstone--source-code-optimizer/blob/main/docs/DevelopmentPlan/DevelopmentPlan.pdf}
}

\subsection*{External Goals}

Our team's primary goal is to learn something practical and transferable skills
on full software development lifecycle that can be applied in the workforce,
including requirements engineering, UI design, API design, iterative
prototyping, and validation.

Additionally, we aim to create a project that we can confidently discuss in
interviews, demonstrating our ability to work on real-world problems. While we
focus on personal and professional growth, we also aim for an A+ as a
nice-to-have achievement.

\subsection*{Attendance}

\subsubsection*{\color{blue}{Expectations}}

Members commit to scheduled meetings, arrive on time, and participate fully. If
a member cannot attend, they will notify the team in advance, provide clear
reasons, and help coordinate a new time if full attendance is required. Repeat
lateness or unexplained absence will be addressed promptly to protect cadence.

\subsubsection*{\color{blue}Acceptable Excuse}

Acceptable reasons include emergencies, illness, family obligations, or other
significant duties, provided the team is informed as early as possible. Vague
or last minute reasons (e.g., ``forgot'') or avoidable conflicts are not
accepted, as they jeopardize schedule, trust, and safety readiness.

\subsubsection*{\color{blue}In Case of Emergency}

When emergencies prevent attendance or delivery, the member must notify the
team immediately via our primary channel (Discord). The team may redistribute
tasks or reschedule as needed. If a deadline is impacted, notify the team and
the instructor or TA promptly so proper arrangements are made without risking
progress or stakeholder expectations.
\subsection*{Accountability and Teamwork}

\subsubsection*{\color{blue}Quality}

Our team sets explicit expectations for meeting prep and deliverable quality,
with safety and reliability emphasized for field operations:

\begin{itemize}
  \item \textbf{Meeting Preparation}:
        \begin{itemize}
          \item Review materials and complete assigned work before meetings; bring concise,
                verifiable updates with blockers and proposed options.
          \item Keep discussions decision oriented; capture risks, assumptions, and next
                actions; timebox and create follow ups for deep dives.
          \item Maintain a decision log and lightweight RACI notes for key items.
        \end{itemize}

  \item \textbf{Deliverables Quality}:
        \begin{itemize}
          \item Meet standards of correctness, clarity, completeness, and reproducibility;
                include error and empty states in the UI.
          \item Align with API contracts and data schemas; document versions and backward
                compatibility expectations.
          \item Provide evidence: latency/FPS snapshots, basic load trials, and stability
                checks (e.g., 30+ min runs) with notes on limits.
        \end{itemize}

  \item \textbf{Verification and Safety Readiness}:
        \begin{itemize}
          \item Validate with staged tests: synthetic videos, recorded flights, and hardware in
                the loop where feasible.
          \item Track KPIs: end to end latency (p50/p95), FPS, target reacquisition time, crash
                free session rate, and UI task success.
          \item Record assumptions, known issues, and mitigations for field usage.
        \end{itemize}

  \item \textbf{Accountability and Feedback}:
        \begin{itemize}
          \item Own outcomes; escalate early if help or time is needed; provide timely handoffs
                and keep docs current.
          \item Welcome review feedback; address comments within 7 days unless a different SLA
                is agreed in advance.
        \end{itemize}

\end{itemize}

\noindent
By upholding these practices, meetings remain efficient and deliverables remain
professional, testable, and field minded.

\subsubsection*{\color{blue}Attitude}

We adopt the following \textbf{expectations} for contribution, interaction, and
cooperation to sustain a respectful and high performance culture:

\begin{itemize}
  \item \textbf{Respectful Communication}: Listen first; respond with concrete
        specifics; separate people from problems.
  \item \textbf{Open Collaboration}: Share ideas and risks early; offer support
        during integration and time critical milestones.
  \item \textbf{Accountability}: Meet timelines or replan early with options
        that protect scope, safety, and critical path.
  \item \textbf{Positive Attitude}: Stay solution oriented under constraints;
        help teammates recover when issues occur.
  \item \textbf{Commitment to Quality}: Prefer simple, observable solutions and
        incremental improvement over risky big bangs.
\end{itemize}

\noindent
We further adopt a concise \textbf{code of conduct}:

\begin{itemize}
  \item \textbf{Inclusivity}: Welcome diverse backgrounds and viewpoints; ensure
        space for every member to contribute.
  \item \textbf{Professionalism}: Act with integrity; avoid offensive language
        and behavior in all settings.
  \item \textbf{Collaboration and Feedback}: Keep critique about the work, not
        the person; make feedback specific and actionable.
  \item \textbf{No Tolerance for Harassment}: Any harassment is unacceptable
        and will be reported and handled immediately.
\end{itemize}

\noindent
Conflicts are handled via the following \textbf{resolution plan}:

\begin{enumerate}
  \item \textbf{Address the Issue Directly}: Involved parties first attempt a
        respectful, solution oriented discussion.
  \item \textbf{Mediation by a Neutral Member}: If unresolved, a neutral
        teammate facilitates and seeks common ground.
  \item \textbf{Escalation to Instructor/TA}: If still unresolved, we escalate
        for guidance and a final decision.
  \item \textbf{Follow Up and Monitoring}: Verify the resolution holds; refine
        norms if needed to prevent recurrence.
\end{enumerate}

By following these expectations and processes, we maintain a positive and
collaborative environment while delivering course outcomes.

\subsubsection*{\color{blue}Stay on Track}

To keep momentum and visibility while preparing for safe field demos, we will
use the methods below:

\begin{enumerate}
  \item \textbf{Regular Check ins and Progress Updates}: Hold \textit{weekly
          meetings} where each member shares progress, risks, and next steps; capture
        actions, owners, and due dates.
  \item \textbf{Performance and Reliability Metrics}:
        \begin{itemize}
          \item \textit{Attendance} tracked in GitHub Issues or the project board.
          \item \textit{Commits/PRs} for steady, reviewable progress with CI gates.
          \item \textit{Task on time rate} and defect turnaround time per milestone.
          \item \textit{Runtime KPIs}: latency, FPS, time to reacquire, stability.
        \end{itemize}
  \item \textbf{Rewards for High Performers}: Recognize members who meet or
        exceed expectations with public kudos and opportunities to lead desired
        submodules or integrations.
  \item \textbf{Managing Underperformance}: If a member underdelivers for more
        than 3 weeks:
        \begin{itemize}
          \item Start with a team conversation to understand obstacles, add pairing, and set
                concrete checkpoints.
          \item If issues persist, add guardrail tasks; in severe cases, meet with the TA or
                instructor.
        \end{itemize}
  \item \textbf{Consequences for Not Contributing}: When a member does not
        contribute fairly:
        \begin{itemize}
          \item Rebalance workload and expectations to protect the milestone.
          \item Escalate to the TA or instructor if patterns continue.
        \end{itemize}
  \item \textbf{Incentives for Meeting Targets Early}: Members who reliably
        meet or exceed targets receive first pick for next milestone tasks and
        additional leadership opportunities.
\end{enumerate}

\subsubsection*{\color{blue}Team Building}

We host a light, bi weekly hangout to build trust and informal bandwidth.
Examples include attending campus events, shared meals, or quick bubble tea
breaks at inclusive times. Optional pair programming or design jam sessions may
be used to accelerate learning and cohesion.

\subsubsection*{\color{blue}Decision Making}

We prefer consensus based decisions and ensure everyone has a chance to speak
before concluding. If consensus is not reachable in time, we vote with equal
weight per member and accept the majority decision. Summaries and action items
are posted within 24 hours in Discord and linked Issues.

\vspace{10pt}
\textit{To handle disagreements: address issues directly and respectfully.}

\begin{enumerate}
  \item Allow members to share concerns without interruption and record key points
        neutrally for transparency.
  \item Keep discussion focused on the topic, not on personal attributions or intent.
  \item When needed, appoint a neutral facilitator to guide the conversation toward a
        resolution within the timebox.
  \item If the issue persists, revisit project goals, safety constraints, and
        stakeholder needs; choose the option that best aligns with them.
\end{enumerate}

By applying these strategies, we preserve a collaborative environment and make
timely, transparent decisions aligned with a high quality tracking solution.

\end{document}
